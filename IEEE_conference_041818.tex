\documentclass[conference]{IEEEtran}
\IEEEoverridecommandlockouts
% The preceding line is only needed to identify funding in the first footnote. If that is unneeded, please comment it out.
\usepackage{cite}
\usepackage{amsmath,amssymb,amsfonts}
\usepackage{algorithmic}
\usepackage{graphicx}
\usepackage{textcomp}
\usepackage{xcolor}
\usepackage{amsmath,amssymb,amsthm}
\usepackage{xparse}
\usepackage{latexsym}
\usepackage{amsfonts}
\usepackage{graphicx}
\usepackage{txfonts}
\usepackage{wasysym}
\usepackage{enumitem}
\usepackage{adjustbox}
\usepackage{ragged2e}
\usepackage{tabularx}
\usepackage{changepage}
\usepackage{setspace}
\usepackage{hhline}
\usepackage{multicol}
\usepackage{float}
\usepackage{multirow}
\usepackage{makecell}
\usepackage{fancyhdr}
\usepackage[toc,page]{appendix}
\usepackage[utf8]{inputenc}
\usepackage[T1]{fontenc}
\usepackage{hyperref}
\hypersetup{
    colorlinks=true,
    linkcolor=blue,
    filecolor=magenta,      
    urlcolor=cyan,
}
\usepackage{isomath}
\usepackage{fixmath}
\usepackage{tikz}
\usepackage{textcomp}
\usepackage{epstopdf} %converting to PDF
\usepackage{upgreek}
\usepackage{mathtools}
\usepackage{xfrac}
\usepackage{lipsum}
\usepackage[colorinlistoftodos]{todonotes}
\usepackage[percent]{overpic}
%general:
%Box and color definitions:
%--------------------------
\newenvironment{ColorBoxedminipage}
{\begin{minipage}} {\end{minipage}}
%{\begin{Sbox}\begin{minipage}}
%{\end{minipage}\end{Sbox}\fcolorbox{Blue}{White}{\TheSbox}}

%General definitions:
%-------------------
\newcommand{\etal}{{\em {et al.}}}
\newcommand{\B}[1]{\mathbf{#1}}
\newcommand{\df}{\triangleq}
\newcommand{\norm}[1]{\left\Vert#1\right\Vert}
\newcommand{\abs}[1]{\left\vert#1\right\vert}
\newcommand{\RE}{\operatorname{Re}}
\newcommand{\IM}{\operatorname{Im}}
\newcommand{\sgma}[3]{\sum\limits_{{#1}={#2}}^{#3}}
\newcommand{\Brace}[1]{\left\{{#1}\right\}} %Braces
\newcommand{\Brack}[1]{\left({#1}\right)} %Brackets
\newcommand{\sBrack}[1]{\left[{#1}\right]} %square Brackets

%\newcommand{\ip}[2]{{\langle{#1},{#2}\rangle}} %inner-product
\newcommand{\ipLW}[3]{{\langle{#1},{#2}\rangle}_{{#3}}} %weighted inner-product

\newcommand{\Tr}[1]{Tr\Brack{#1}}
\newcommand{\Mtr}[2] %short notation for 2x1 Matrix.
{\begin{bmatrix}
  #1 \\
  #2
\end{bmatrix}}
\newcommand{\cMtr}[2] %short notation for 2x1 Matrix with curves.
{\left(
\begin{array}{c}
    {#1} \\
    {#2} \\
\end{array}
\right)}
\newcommand{\Mtrs}[2] %short notation for 2x1 Matrix star (adjoint)
{\begin{bmatrix}
  #1 &
  #2
\end{bmatrix}}
\newcommand{\Mtrt}[3] %short notation for 3x1 Matrix.
{\begin{bmatrix}
  #1 \\
  #2 \\
  #3
\end{bmatrix}}

\newcommand{\Cases}[4]{
\left\{
\begin{tabular}{lcl}
    $#1$ & $=$ & $#2$\\
    $#3$     & $=$ & $#4$
\end{tabular}
\right. }

\newcommand{\und}{\underline} %How lazy can I get?
\newcommand{\ovr}{\overline}
\newcommand{\conj}[1]{{#1}^\ast} %Conjugation


\newcommand{\er}[1]{{(\ref{#1})}} %equation reference

\newtheorem{Lemma}{Lemma}{}
\newtheorem{Prop}{Proposition}{}
\newtheorem{theorem}{Theorem}{}


\newenvironment{alg}[5]
{
\begin{figure}[htbp]
\begin{center}
\fbox{
  \begin{ColorBoxedminipage}{13cm}
%    \leftline{\color{Black}\bf {#1}}
    {#4}
   \end{ColorBoxedminipage}
   }
\end{center}
  \bcaptionff{#1}{#2}{}{#3}
  \label{#5}
\end{figure}
}{}

%Just body, caption and label.
\newenvironment{algo}[3]
{
\begin{figure}[htbp]
\begin{center}
\fbox{
  \begin{ColorBoxedminipage}{7.5cm}
%    \leftline{\color{Black}\bf {#1}}
    {#1}
   \end{ColorBoxedminipage}
   }
\end{center}
  \caption{#2}
  \label{#3}
\end{figure}
}{}

\newenvironment{BOX}[1]
{
\begin{center}
\fbox{
  \begin{ColorBoxedminipage}{16cm}
%    \leftline{\color{Black}\bf {#1}}
    {#1}
   \end{ColorBoxedminipage}
   }
\end{center}
}{}

\newcommand\vecnot[1]{\boldsymbol{#1}}
\newcommand\optvecnot[1]{\vecnot{#1}_{opt}}

\usepackage{amsmath}
\DeclareMathOperator*{\argmax}{arg\,max}
\DeclareMathOperator*{\argmin}{arg\,min}
\usepackage{subfig}
\usepackage{stfloats}
%%%%%%%%%%%%%%    aliases    %%%%%%%%%%%%%%
\newcommand{\Brace}[1]{\left\{{#1}\right\}}
\newcommand{\rBrace}[1]{\left({#1}\right)}
\newcommand{\lBrace}[1]{\left|{#1}\right|}
\newcommand{\vBrace}[1]{\left[{#1}\right]}
\newcommand{\cBrace}[1]{\left\{{#1}\right\}}
\newcommand{\dTheta}{\Delta\theta}
\newcommand{\dPhi}{\Delta\phi}
\newcommand{\dOmega}{\Delta\omega}
\newcommand{\dR}{\Delta{R}}
\newcommand{\dTau}{CHANGE_TO_DPHI}
% \newcommand{\D}[2]{\mathcal{D}\rBrace{#1,#2}}
% \newcommand{\Dp}[2]{\mathcal{D}^{#2}\rBrace{#1}}
\newcommand{\D}[2]{\text{D}\rBrace{#1,#2}}
\newcommand{\Dp}[2]{\text{D}^{#2}\rBrace{#1}}

\newcommand{\vd}{\vecnot{d}}
\newcommand{\vx}{\vecnot{x}}
\newcommand{\vAlpha}{\vecnot{\alpha}}
\newcommand{\vBeta}{\vecnot{\beta}}
\newcommand{\vdT}{\vd^{T}}
\newcommand{\vxT}{\vecnot{x}^{T}}
\newcommand{\vAlphaT}{\vAlpha^{T}}
\newcommand{\vBetaT}{\vBeta^{T}}
\newcommand{\vdH}{\vd^{H}}
\newcommand{\vxH}{\vecnot{x}^{H}}
\newcommand{\vAlphaH}{\vAlpha^{H}}
\newcommand{\vBetaH}{\vBeta^{H}}
\newcommand{\vEta}{\vecnot{\eta}}
\newcommand{\vEtaT}{\vEta^{T}}
\newcommand{\vEtaH}{\vEta^{H}}
%\newcommand{\F}[1]{#1^{\mathcal{F}}}
\newcommand{\F}[1]{\MakeUppercase{#1}}
\newcommand{\ePhi}[1]{\exp{\rBrace{#1j\phi}}}
\newcommand{\thetaD}{\theta_{\text{d}}}

\NewDocumentCommand{\evalat}{sO{\big}mm}{%
  \IfBooleanTF{#1}
   {\mleft. #3 \mright|_{#4}}
   {#3#2|_{#4}}%
}

\newcommand{\Steer}[1]{\vd_{#1}} 
\newcommand{\aTd}{\vAlpha^{T}\Steer{}} 
\newcommand{\bTd}{\vBeta^{T}\Steer{}}
\newcommand{\Hr}{\mathcal{H}}
\newcommand{\myTodo}[2]{\ifdefined\showTodo{\todo[#1]{#2}}\else\fi}
\newcommand{\myTodoNew}[2]{\ifdefined\showTodoNew{\todo[#1]{#2}}\else\fi}
% \newcommand{\coefSetName}{\text{CB}}
\newcommand{\coefSetName}{\text{DS}}


%%%%%%%%%%%%%%%%%%%%%%%%%%%%%%%%%%%%%%%%%%%%%%%%%%%%%%
%%%%%%%%%%%%%%      Document flags     %%%%%%%%%%%%%%%
%%%%%%%%%%%%%%%%%%%%%%%%%%%%%%%%%%%%%%%%%%%%%%%%%%%%%%
% \def\showDev{}
% \def\showTodo{}
\def\showTodoNew{}
\def\DEFIncludeAttenuation{}
\def\DEFInclueApplication{}
%%%%%%%%%%%%  Document Code starts here %%%%%%%%%%%%%%
\def\BibTeX{{\rm B\kern-.05em{\sc i\kern-.025em b}\kern-.08em
    T\kern-.1667em\lower.7ex\hbox{E}\kern-.125emX}}
\begin{document}

\title{Feedback based beamforming in the context of Localization
\thanks{Identify applicable funding agency here. If none, delete this.}
}

\author{\IEEEauthorblockN{1\textsuperscript{st} Given Name Surname}
\IEEEauthorblockA{\textit{dept. name of organization (of Aff.)} \\
\textit{name of organization (of Aff.)}\\
City, Country \\
email address}
\and
\IEEEauthorblockN{2\textsuperscript{nd} Given Name Surname}
\IEEEauthorblockA{\textit{dept. name of organization (of Aff.)} \\
\textit{name of organization (of Aff.)}\\
City, Country \\
email address}
\and
\IEEEauthorblockN{3\textsuperscript{rd} Given Name Surname}
\IEEEauthorblockA{\textit{dept. name of organization (of Aff.)} \\
\textit{name of organization (of Aff.)}\\
City, Country \\
email address}
\and
\IEEEauthorblockN{4\textsuperscript{th} Given Name Surname}
\IEEEauthorblockA{\textit{dept. name of organization (of Aff.)} \\
\textit{name of organization (of Aff.)}\\
City, Country \\
email address}
\and
\IEEEauthorblockN{5\textsuperscript{th} Given Name Surname}
\IEEEauthorblockA{\textit{dept. name of organization (of Aff.)} \\
\textit{name of organization (of Aff.)}\\
City, Country \\
email address}
\and
\IEEEauthorblockN{6\textsuperscript{th} Given Name Surname}
\IEEEauthorblockA{\textit{dept. name of organization (of Aff.)} \\
\textit{name of organization (of Aff.)}\\
City, Country \\
email address}
}

\maketitle

\begin{abstract}
A novel feedback based approach to beamforming in the context of localization, is presented. Performance analysis and comparison to traditional array processing show significant improvement. 
\end{abstract}

\begin{IEEEkeywords}
Spatial-IIR, spatial processing, array processing, beamforming, feedback based beamforming, localization, beam-pattern.
\end{IEEEkeywords}

\section{Introduction}
Array processing is a wide research area, combining spatial filtering, source localization, signal detection, source separation, manifold learning, feature extraction, and many more.
\par Spatial processing enables localizing a transmitting source \cite{skolnik2008radar}, blind separation of mixed signals \cite{Comon1994IndependentConcept}, improving signal to noise ratio (SNR) \cite{Frost1972AProcessing,verdu1998multiuser} and many more.
\par We focus on uniform linear arrays (ULA) due to their simplicity and the wide research that was conducted on them.
\par Trying to overcome the ULA spatial limitations 
\par One approach, involving different array geometries, examined minimum redundancy arrays \cite{Moffet1968Minimum-RedundancyArrays,Pillai1985AEstimation,UnnikrishnaPillai1987StatisticalMatrix} trying to reduce the spatial ambiguity through minimization of redundant inter-element spacing in order to increase the overall resolution. 
\par Another approach, commonly referenced as ``virtual arrays" \cite{Pal2010NestedFreedom,Chevalier2005OnProcessing,Mendel1999ApplicationsProcessing}\myTodo{inline}{are you citing known works? \\\textbf{(Yes, for example \cite{Chevalier2005OnProcessing} is cited ~200 times)}\\ I think Veydanathan \\\textbf{ADDED his work.}\\(hope that spelling correctly) did some works on this. Also Arye Yeredor from TAU} deals with the extraction of samples originated in sensors that do not really exist (i.e. relying on higher order statistics and manipulating multiple statistical cross-terms in order to estimate statistical characteristics of signals impinging in missing sensors). 
Using a similar approach, the $2q$-MUSIC algorithm \cite{Chevalier2006High-resolutionAlgorithm} \myTodo{inline}{not familiar... \\\textbf{Has around ~140 citations}\\}, enables the use of $N^{2q}$ ``virtual elements'', by calculating the $q$'th order statistics. \myTodo{inline}{\textbf{REPHRASED: and reduced}\\did not understand this last sentence. You did not explain what are nested arrays.}
\par A well known \cite{VanVeenBeamforming:Filtering} equivalence, which inspired the current paper, is the analogy between ULA spatial array processing of narrow-band signals and finite impulse response (FIR) temporal filtering. 
In the context of temporal signal processing, it is well known that for a given filter order $N$, in many cases, the infinite impulse response (IIR) filter leads to improved performance, as compared to FIR filter design of the same order. In particular, narrow transition regions, and low sidelobes can be achieved by implementing feedback based filtering. 
\par This naturally arises the question, ``what are the equivalent spatial domain processing methods which will be analogous to temporal IIR filtering? " 
\par A related work \cite{Wen2013ExtendingStructure} has also addressed this question. 
There, in the context of ULA, two approaches were considered. \myTodo{inline}{\textbf{DONE:}\\The next sentence is not clear. Please rephrase}
The first one was to estimate the time of arrival (TOA) difference between two consecutive sensors and to synthetically generate the recursive part of the IIR filter, entirely in the time-domain. 
The second approach suggested to consider overlapping subsets of one large ULA as finite approximation to an infinite array. Surely, the former approach heavily relies on the accuracy of the delay estimation and the latter approach does not achieve a recursive spatial response. In both cases, there is no true spatial feedback between the array and the source of interest.
\par Other works \cite{Madanayake2008AFilters,Madanayake2008ABeamformer} use the concept of $2D$ spatio-temporal plane wave representation (i.e. a straight line angled according to the DOA in the spatio-temporal plane), to design ultra-wide-band (UWB) filters \cite{L.Bruton1983HighlyPlanes} which both sample spatial snapshots of the signal and recursively process it in temporal domain. \myTodo{inline} {\textbf{REPHRASED}\\please explain (maybe later just to me) how UWB filtering is used}\myTodo{inline}{\textbf{Removed - not important - a way of implementation}\\what is that?}\myTodo{inline}{\textbf{DONE:}\\what are the drawbacks if these last papers?}
Here as well, the recursive part of the filter is obtained entirely in the temporal domain.  
\par The goal in this contribution, is to present a sensor array processing approach which achieves spatial-domain feedback-based processing, similarly to IIR filtering in the time domain. Opposed the previous works, we do not wish to estimate the inter-element delay and apply the recursive part in the time domain, but rather implement the feedback entirely in the spatial domain.
We focus on a localization problem, where our goal is to estimate the direction and the range of some target.
Opposed to classical array processing, we incorporate spatial feedback, show its equivalence to IIR filtering in case of a ULA, and analyze some key aspects of the proposed scheme. 

The spatial feedback between the array and the target is created by constantly re-transmitting a signal and its echoes between the array and the target. 
The initial stimulus can be generated at the target itself or at the location of the array. In the text to follow, we assume this is the latter. 

Also, similar to radar applications, the target can be passive and merely reflect the impinging signal, or to be cooperative, i.e. by receiving, enhancing the signal and re-transmitting it back to the array. In this work, we assume the former.

\par The outline of this paper is as follows. We first formulate the classic spatial beamforming setup in Sec.~\ref{sec:setup}. Then, in Sec.~\ref{sec_introduceFeedback}, we propose our novel feedback based architecture, and calculate its spatial response. In Sec.~ \ref{sec_FIM} we evaluate the Fisher Information Matrix (FIM) and discuss optimal choices of the array weights in order to  maximize the information in detecting the target's range and DOA. In Sec.~\ref{sec_Performance} we evaluate some key features of the proposed beamforming with feedback. Specifically, we compute the array beamwidth, its peak to sidelobe ratio and the array directivity, showing significant improvement compared to traditional beamforming without spatial feedback. 
In Sec.~\ref{sec_app} we simulate the proposed processing scheme, and emphasis its sensitivity to range errors. We then suggest a strategy which eliminates this sensitivity. Finally, concluding remarks are stated in Sec.~\ref{sec_conclusions}.
\myTodo{inline}{Here you should state the content of the paper. In Sec. \ref{sec:setup} we define our setup. In Sec. ... we analyze the suggested feedback based system. etc...}

\section{Ease of Use}

\subsection{Maintaining the Integrity of the Specifications}

The IEEEtran class file is used to format your paper and style the text. All margins, 
column widths, line spaces, and text fonts are prescribed; please do not 
alter them. You may note peculiarities. For example, the head margin
measures proportionately more than is customary. This measurement 
and others are deliberate, using specifications that anticipate your paper 
as one part of the entire proceedings, and not as an independent document. 
Please do not revise any of the current designations.

\section{Prepare Your Paper Before Styling}
Before you begin to format your paper, first write and save the content as a 
separate text file. Complete all content and organizational editing before 
formatting. Please note sections \ref{AA}--\ref{SCM} below for more information on 
proofreading, spelling and grammar.

Keep your text and graphic files separate until after the text has been 
formatted and styled. Do not number text heads---{\LaTeX} will do that 
for you.

\subsection{Abbreviations and Acronyms}\label{AA}
Define abbreviations and acronyms the first time they are used in the text, 
even after they have been defined in the abstract. Abbreviations such as 
IEEE, SI, MKS, CGS, ac, dc, and rms do not have to be defined. Do not use 
abbreviations in the title or heads unless they are unavoidable.

\subsection{Units}
\begin{itemize}
\item Use either SI (MKS) or CGS as primary units. (SI units are encouraged.) English units may be used as secondary units (in parentheses). An exception would be the use of English units as identifiers in trade, such as ``3.5-inch disk drive''.
\item Avoid combining SI and CGS units, such as current in amperes and magnetic field in oersteds. This often leads to confusion because equations do not balance dimensionally. If you must use mixed units, clearly state the units for each quantity that you use in an equation.
\item Do not mix complete spellings and abbreviations of units: ``Wb/m\textsuperscript{2}'' or ``webers per square meter'', not ``webers/m\textsuperscript{2}''. Spell out units when they appear in text: ``. . . a few henries'', not ``. . . a few H''.
\item Use a zero before decimal points: ``0.25'', not ``.25''. Use ``cm\textsuperscript{3}'', not ``cc''.)
\end{itemize}

\subsection{Equations}
Number equations consecutively. To make your 
equations more compact, you may use the solidus (~/~), the exp function, or 
appropriate exponents. Italicize Roman symbols for quantities and variables, 
but not Greek symbols. Use a long dash rather than a hyphen for a minus 
sign. Punctuate equations with commas or periods when they are part of a 
sentence, as in:
\begin{equation}
a+b=\gamma\label{eq}
\end{equation}

Be sure that the 
symbols in your equation have been defined before or immediately following 
the equation. Use ``\eqref{eq}'', not ``Eq.~\eqref{eq}'' or ``equation \eqref{eq}'', except at 
the beginning of a sentence: ``Equation \eqref{eq} is . . .''

\subsection{\LaTeX-Specific Advice}

Please use ``soft'' (e.g., \verb|\eqref{Eq}|) cross references instead
of ``hard'' references (e.g., \verb|(1)|). That will make it possible
to combine sections, add equations, or change the order of figures or
citations without having to go through the file line by line.

Please don't use the \verb|{eqnarray}| equation environment. Use
\verb|{align}| or \verb|{IEEEeqnarray}| instead. The \verb|{eqnarray}|
environment leaves unsightly spaces around relation symbols.

Please note that the \verb|{subequations}| environment in {\LaTeX}
will increment the main equation counter even when there are no
equation numbers displayed. If you forget that, you might write an
article in which the equation numbers skip from (17) to (20), causing
the copy editors to wonder if you've discovered a new method of
counting.

{\BibTeX} does not work by magic. It doesn't get the bibliographic
data from thin air but from .bib files. If you use {\BibTeX} to produce a
bibliography you must send the .bib files. 

{\LaTeX} can't read your mind. If you assign the same label to a
subsubsection and a table, you might find that Table I has been cross
referenced as Table IV-B3. 

{\LaTeX} does not have precognitive abilities. If you put a
\verb|\label| command before the command that updates the counter it's
supposed to be using, the label will pick up the last counter to be
cross referenced instead. In particular, a \verb|\label| command
should not go before the caption of a figure or a table.

Do not use \verb|\nonumber| inside the \verb|{array}| environment. It
will not stop equation numbers inside \verb|{array}| (there won't be
any anyway) and it might stop a wanted equation number in the
surrounding equation.

\subsection{Some Common Mistakes}\label{SCM}
\begin{itemize}
\item The word ``data'' is plural, not singular.
\item The subscript for the permeability of vacuum $\mu_{0}$, and other common scientific constants, is zero with subscript formatting, not a lowercase letter ``o''.
\item In American English, commas, semicolons, periods, question and exclamation marks are located within quotation marks only when a complete thought or name is cited, such as a title or full quotation. When quotation marks are used, instead of a bold or italic typeface, to highlight a word or phrase, punctuation should appear outside of the quotation marks. A parenthetical phrase or statement at the end of a sentence is punctuated outside of the closing parenthesis (like this). (A parenthetical sentence is punctuated within the parentheses.)
\item A graph within a graph is an ``inset'', not an ``insert''. The word alternatively is preferred to the word ``alternately'' (unless you really mean something that alternates).
\item Do not use the word ``essentially'' to mean ``approximately'' or ``effectively''.
\item In your paper title, if the words ``that uses'' can accurately replace the word ``using'', capitalize the ``u''; if not, keep using lower-cased.
\item Be aware of the different meanings of the homophones ``affect'' and ``effect'', ``complement'' and ``compliment'', ``discreet'' and ``discrete'', ``principal'' and ``principle''.
\item Do not confuse ``imply'' and ``infer''.
\item The prefix ``non'' is not a word; it should be joined to the word it modifies, usually without a hyphen.
\item There is no period after the ``et'' in the Latin abbreviation ``et al.''.
\item The abbreviation ``i.e.'' means ``that is'', and the abbreviation ``e.g.'' means ``for example''.
\end{itemize}
An excellent style manual for science writers is \cite{b7}.

\subsection{Authors and Affiliations}
\textbf{The class file is designed for, but not limited to, six authors.} A 
minimum of one author is required for all conference articles. Author names 
should be listed starting from left to right and then moving down to the 
next line. This is the author sequence that will be used in future citations 
and by indexing services. Names should not be listed in columns nor group by 
affiliation. Please keep your affiliations as succinct as possible (for 
example, do not differentiate among departments of the same organization).

\subsection{Identify the Headings}
Headings, or heads, are organizational devices that guide the reader through 
your paper. There are two types: component heads and text heads.

Component heads identify the different components of your paper and are not 
topically subordinate to each other. Examples include Acknowledgments and 
References and, for these, the correct style to use is ``Heading 5''. Use 
``figure caption'' for your Figure captions, and ``table head'' for your 
table title. Run-in heads, such as ``Abstract'', will require you to apply a 
style (in this case, italic) in addition to the style provided by the drop 
down menu to differentiate the head from the text.

Text heads organize the topics on a relational, hierarchical basis. For 
example, the paper title is the primary text head because all subsequent 
material relates and elaborates on this one topic. If there are two or more 
sub-topics, the next level head (uppercase Roman numerals) should be used 
and, conversely, if there are not at least two sub-topics, then no subheads 
should be introduced.

\subsection{Figures and Tables}
\paragraph{Positioning Figures and Tables} Place figures and tables at the top and 
bottom of columns. Avoid placing them in the middle of columns. Large 
figures and tables may span across both columns. Figure captions should be 
below the figures; table heads should appear above the tables. Insert 
figures and tables after they are cited in the text. Use the abbreviation 
``Fig.~\ref{fig}'', even at the beginning of a sentence.

\begin{table}[htbp]
\caption{Table Type Styles}
\begin{center}
\begin{tabular}{|c|c|c|c|}
\hline
\textbf{Table}&\multicolumn{3}{|c|}{\textbf{Table Column Head}} \\
\cline{2-4} 
\textbf{Head} & \textbf{\textit{Table column subhead}}& \textbf{\textit{Subhead}}& \textbf{\textit{Subhead}} \\
\hline
copy& More table copy$^{\mathrm{a}}$& &  \\
\hline
\multicolumn{4}{l}{$^{\mathrm{a}}$Sample of a Table footnote.}
\end{tabular}
\label{tab1}
\end{center}
\end{table}

\begin{figure}[htbp]
\centerline{\includegraphics{fig1.png}}
\caption{Example of a figure caption.}
\label{fig}
\end{figure}

Figure Labels: Use 8 point Times New Roman for Figure labels. Use words 
rather than symbols or abbreviations when writing Figure axis labels to 
avoid confusing the reader. As an example, write the quantity 
``Magnetization'', or ``Magnetization, M'', not just ``M''. If including 
units in the label, present them within parentheses. Do not label axes only 
with units. In the example, write ``Magnetization (A/m)'' or ``Magnetization 
\{A[m(1)]\}'', not just ``A/m''. Do not label axes with a ratio of 
quantities and units. For example, write ``Temperature (K)'', not 
``Temperature/K''.

\section*{Acknowledgment}

The preferred spelling of the word ``acknowledgment'' in America is without 
an ``e'' after the ``g''. Avoid the stilted expression ``one of us (R. B. 
G.) thanks $\ldots$''. Instead, try ``R. B. G. thanks$\ldots$''. Put sponsor 
acknowledgments in the unnumbered footnote on the first page.

\section*{References}

Please number citations consecutively within brackets \cite{b1}. The 
sentence punctuation follows the bracket \cite{b2}. Refer simply to the reference 
number, as in \cite{b3}---do not use ``Ref. \cite{b3}'' or ``reference \cite{b3}'' except at 
the beginning of a sentence: ``Reference \cite{b3} was the first $\ldots$''

Number footnotes separately in superscripts. Place the actual footnote at 
the bottom of the column in which it was cited. Do not put footnotes in the 
abstract or reference list. Use letters for table footnotes.

Unless there are six authors or more give all authors' names; do not use 
``et al.''. Papers that have not been published, even if they have been 
submitted for publication, should be cited as ``unpublished'' \cite{b4}. Papers 
that have been accepted for publication should be cited as ``in press'' \cite{b5}. 
Capitalize only the first word in a paper title, except for proper nouns and 
element symbols.

For papers published in translation journals, please give the English 
citation first, followed by the original foreign-language citation \cite{b6}.

\bibliographystyle{IEEEtran}
\bibliography{./Modules/Mendeley,./Modules/LocalBib}
\vspace{12pt}
\color{red}
IEEE conference templates contain guidance text for composing and formatting conference papers. Please ensure that all template text is removed from your conference paper prior to submission to the conference. Failure to remove the template text from your paper may result in your paper not being published.

\end{document}
