\documentclass[conference]{IEEEtran}
\IEEEoverridecommandlockouts
\usepackage{cite}
\usepackage{amsmath,amssymb,amsfonts}
\usepackage{algorithmic}
\usepackage{graphicx}
\usepackage{textcomp}
\usepackage{xcolor}
\usepackage{amsmath,amssymb,amsthm}
\usepackage{xparse}
\usepackage{latexsym}
\usepackage{amsfonts}
\usepackage{graphicx}
\usepackage{txfonts}
\usepackage{wasysym}
\usepackage{enumitem}
\usepackage{adjustbox}
\usepackage{ragged2e}
\usepackage{tabularx}
\usepackage{changepage}
\usepackage{setspace}
\usepackage{hhline}
\usepackage{multicol}
\usepackage{float}
\usepackage{multirow}
\usepackage{makecell}
\usepackage{fancyhdr}
\usepackage[toc,page]{appendix}
\usepackage[utf8]{inputenc}
\usepackage[T1]{fontenc}
\usepackage{hyperref}
\hypersetup{
    colorlinks=true,
    linkcolor=blue,
    filecolor=magenta,      
    urlcolor=cyan,
}
\usepackage{isomath}
\usepackage{fixmath}
\usepackage{tikz}
\usepackage{textcomp}
\usepackage{epstopdf} %converting to PDF
\usepackage{upgreek}
\usepackage{mathtools}
\usepackage{xfrac}
\usepackage{lipsum}
\usepackage[colorinlistoftodos]{todonotes}
\usepackage[percent]{overpic}
%general:
%Box and color definitions:
%--------------------------
\newenvironment{ColorBoxedminipage}
{\begin{minipage}} {\end{minipage}}
%{\begin{Sbox}\begin{minipage}}
%{\end{minipage}\end{Sbox}\fcolorbox{Blue}{White}{\TheSbox}}

%General definitions:
%-------------------
\newcommand{\etal}{{\em {et al.}}}
\newcommand{\B}[1]{\mathbf{#1}}
\newcommand{\df}{\triangleq}
\newcommand{\norm}[1]{\left\Vert#1\right\Vert}
\newcommand{\abs}[1]{\left\vert#1\right\vert}
\newcommand{\RE}{\operatorname{Re}}
\newcommand{\IM}{\operatorname{Im}}
\newcommand{\sgma}[3]{\sum\limits_{{#1}={#2}}^{#3}}
\newcommand{\Brace}[1]{\left\{{#1}\right\}} %Braces
\newcommand{\Brack}[1]{\left({#1}\right)} %Brackets
\newcommand{\sBrack}[1]{\left[{#1}\right]} %square Brackets

%\newcommand{\ip}[2]{{\langle{#1},{#2}\rangle}} %inner-product
\newcommand{\ipLW}[3]{{\langle{#1},{#2}\rangle}_{{#3}}} %weighted inner-product

\newcommand{\Tr}[1]{Tr\Brack{#1}}
\newcommand{\Mtr}[2] %short notation for 2x1 Matrix.
{\begin{bmatrix}
  #1 \\
  #2
\end{bmatrix}}
\newcommand{\cMtr}[2] %short notation for 2x1 Matrix with curves.
{\left(
\begin{array}{c}
    {#1} \\
    {#2} \\
\end{array}
\right)}
\newcommand{\Mtrs}[2] %short notation for 2x1 Matrix star (adjoint)
{\begin{bmatrix}
  #1 &
  #2
\end{bmatrix}}
\newcommand{\Mtrt}[3] %short notation for 3x1 Matrix.
{\begin{bmatrix}
  #1 \\
  #2 \\
  #3
\end{bmatrix}}

\newcommand{\Cases}[4]{
\left\{
\begin{tabular}{lcl}
    $#1$ & $=$ & $#2$\\
    $#3$     & $=$ & $#4$
\end{tabular}
\right. }

\newcommand{\und}{\underline} %How lazy can I get?
\newcommand{\ovr}{\overline}
\newcommand{\conj}[1]{{#1}^\ast} %Conjugation


\newcommand{\er}[1]{{(\ref{#1})}} %equation reference

\newtheorem{Lemma}{Lemma}{}
\newtheorem{Prop}{Proposition}{}
\newtheorem{theorem}{Theorem}{}


\newenvironment{alg}[5]
{
\begin{figure}[htbp]
\begin{center}
\fbox{
  \begin{ColorBoxedminipage}{13cm}
%    \leftline{\color{Black}\bf {#1}}
    {#4}
   \end{ColorBoxedminipage}
   }
\end{center}
  \bcaptionff{#1}{#2}{}{#3}
  \label{#5}
\end{figure}
}{}

%Just body, caption and label.
\newenvironment{algo}[3]
{
\begin{figure}[htbp]
\begin{center}
\fbox{
  \begin{ColorBoxedminipage}{7.5cm}
%    \leftline{\color{Black}\bf {#1}}
    {#1}
   \end{ColorBoxedminipage}
   }
\end{center}
  \caption{#2}
  \label{#3}
\end{figure}
}{}

\newenvironment{BOX}[1]
{
\begin{center}
\fbox{
  \begin{ColorBoxedminipage}{16cm}
%    \leftline{\color{Black}\bf {#1}}
    {#1}
   \end{ColorBoxedminipage}
   }
\end{center}
}{}

\newcommand\vecnot[1]{\boldsymbol{#1}}
\newcommand\optvecnot[1]{\vecnot{#1}_{opt}}

\usepackage{amsmath}
\DeclareMathOperator*{\argmax}{arg\,max}
\DeclareMathOperator*{\argmin}{arg\,min}
\usepackage{subfig}
\usepackage{stfloats}
%%%%%%%%%%%%%%    aliases    %%%%%%%%%%%%%%
\newcommand{\Brace}[1]{\left\{{#1}\right\}}
\newcommand{\rBrace}[1]{\left({#1}\right)}
\newcommand{\lBrace}[1]{\left|{#1}\right|}
\newcommand{\vBrace}[1]{\left[{#1}\right]}
\newcommand{\cBrace}[1]{\left\{{#1}\right\}}
\newcommand{\dTheta}{\Delta\theta}
\newcommand{\dPhi}{\Delta\phi}
\newcommand{\dOmega}{\Delta\omega}
\newcommand{\dR}{\Delta{R}}
\newcommand{\dTau}{CHANGE_TO_DPHI}
% \newcommand{\D}[2]{\mathcal{D}\rBrace{#1,#2}}
% \newcommand{\Dp}[2]{\mathcal{D}^{#2}\rBrace{#1}}
\newcommand{\D}[2]{\text{D}\rBrace{#1,#2}}
\newcommand{\Dp}[2]{\text{D}^{#2}\rBrace{#1}}

\newcommand{\vd}{\vecnot{d}}
\newcommand{\vx}{\vecnot{x}}
\newcommand{\vAlpha}{\vecnot{\alpha}}
\newcommand{\vBeta}{\vecnot{\beta}}
\newcommand{\vdT}{\vd^{T}}
\newcommand{\vxT}{\vecnot{x}^{T}}
\newcommand{\vAlphaT}{\vAlpha^{T}}
\newcommand{\vBetaT}{\vBeta^{T}}
\newcommand{\vdH}{\vd^{H}}
\newcommand{\vxH}{\vecnot{x}^{H}}
\newcommand{\vAlphaH}{\vAlpha^{H}}
\newcommand{\vBetaH}{\vBeta^{H}}
\newcommand{\vEta}{\vecnot{\eta}}
\newcommand{\vEtaT}{\vEta^{T}}
\newcommand{\vEtaH}{\vEta^{H}}
%\newcommand{\F}[1]{#1^{\mathcal{F}}}
\newcommand{\F}[1]{\MakeUppercase{#1}}
\newcommand{\ePhi}[1]{\exp{\rBrace{#1j\phi}}}
\newcommand{\thetaD}{\theta_{\text{d}}}

\NewDocumentCommand{\evalat}{sO{\big}mm}{%
  \IfBooleanTF{#1}
   {\mleft. #3 \mright|_{#4}}
   {#3#2|_{#4}}%
}

\newcommand{\Steer}[1]{\vd_{#1}} 
\newcommand{\aTd}{\vAlpha^{T}\Steer{}} 
\newcommand{\bTd}{\vBeta^{T}\Steer{}}
\newcommand{\Hr}{\mathcal{H}}
\newcommand{\myTodo}[2]{\ifdefined\showTodo{\todo[#1]{#2}}\else\fi}
\newcommand{\myTodoNew}[2]{\ifdefined\showTodoNew{\todo[#1]{#2}}\else\fi}
% \newcommand{\coefSetName}{\text{CB}}
\newcommand{\coefSetName}{\text{DS}}


%%%%%%%%%%%%%%%%%%%%%%%%%%%%%%%%%%%%%%%%%%%%%%%%%%%%%%
%%%%%%%%%%%%%%      Document flags     %%%%%%%%%%%%%%%
%%%%%%%%%%%%%%%%%%%%%%%%%%%%%%%%%%%%%%%%%%%%%%%%%%%%%%
% \def\showDev{}
% \def\showTodo{}
\def\showTodoNew{}
\def\DEFIncludeAttenuation{}
\def\DEFInclueApplication{}
%%%%%%%%%%%%  Document Code starts here %%%%%%%%%%%%%%
\def\BibTeX{{\rm B\kern-.05em{\sc i\kern-.025em b}\kern-.08em
    T\kern-.1667em\lower.7ex\hbox{E}\kern-.125emX}}
\begin{document}

\title{Localization With Feedback}

\author{\IEEEauthorblockN{Itay Yehezkel Karo}
\IEEEauthorblockA{
Technion –
Israel Institute of Technology
\\
Technion City - 
Haifa, Israel
\\
itayyeka@gmail.com}
\and
\IEEEauthorblockN{Tsvi G. Dvorkind}
\IEEEauthorblockA{\textit{RAFAEL} \\
\textit{Advanced Defense Systems LTD}, Israel\\
dvorkind@rafael.co.il}
\and
\IEEEauthorblockN{Israel Cohen}
\IEEEauthorblockA{
Technion –
Israel Institute of Technology
\\
Technion City - 
Haifa, Israel
\\
icohen@ee.technion.ac.il}
}
\maketitle

\begin{abstract}
A novel feedback based approach to beamforming is presented. \todo[inline]{to add info...} Performance analysis and comparison to traditional array processing show significant improvement. 
\end{abstract}

\begin{IEEEkeywords}
Spatial-IIR, spatial processing, array processing, beamforming, feedback based beamforming, localization, beam-pattern.
\end{IEEEkeywords}

\section{Introduction}
Array processing is a wide research area, dealing with the processing of impinging signals according to their spatial characteristics, such as direction-of-arrival (DOA) based signal enhancement, blind source separation and much more.
\par This work focuses on uniform linear arrays (ULA) due to their simplicity and the wide research that was conducted on them. 
\par ULAs have also some inherent spatial limitations, considering traditional processing, closely linking the array performance to the array's aperture \cite{VanTrees2002DetectionIV}. 
The bigger the aperture, the better the spatial performance of the array.
\par In pursuit of cost-effectiveness and system minimization, wide range of works have been done, trying to enhance the array's spatial performance without increasing the number of elements or its aperture.
\par One such work, examined minimum redundancy arrays \cite{Moffet1968Minimum-RedundancyArrays,Pillai1985AEstimation,UnnikrishnaPillai1987StatisticalMatrix}, while trying to reduce the spatial ambiguity through minimization of redundant inter-element spacing in order to increase the overall resolution.
\par Also, the ``virtual arrays" \cite{Pal2010NestedFreedom,Chevalier2005OnProcessing,Mendel1999ApplicationsProcessing} concept is based on the extraction of samples originated in sensors that do not really exist, relying on higher order statistics.
\par Inspired by the analogy between ULA spatial array processing of narrowband signals and finite impulse response (FIR) temporal filtering \cite{VanVeenBeamforming:Filtering}, some works \cite{Wen2013ExtendingStructure,Madanayake2008AFilters,Madanayake2008ABeamformer} have tried to find the spatial equivalent to infinite impulse response (IIR) filter.
\par Obviously, those works were motivated by the fact that for a given filter of order $N$, in many cases, the infinite impulse response (IIR) performs better than FIR, considering narrow transition regions and low sidelobes.
\par To this end, two approaches have been considered in \cite{Wen2013ExtendingStructure}.The first one was to estimate the time of arrival (TOA) difference between two consecutive sensors and to synthetically generate the recursive part of the IIR filter, entirely in the time-domain. The second approach suggested to consider overlapping subsets of one large ULA as finite approximation to an infinite array. Unfortunately, none of the above achieves the \textit{spatial-IIR} goal, failing to produce the desired spatial recursiveness. 
\par Other works \cite{Madanayake2008AFilters,Madanayake2008ABeamformer} use the concept of $2D$ spatio-temporal plane wave representation where the recursive part of the spatial filter is implemented in the temporal domain. 
\par In this contribution, we wish to present a sensor array processing approach which achieves the desired spatial domain exclusive IIR-like beampattern, in the context of target localization problem, without any temporal processing of the signal.
\par Comparing to traditional array processing, the novelty in this contribution is the incorporation of synthetic spatial feedback signal, generated by the array itself and re-transmitted to the medium in an omni-directional way. 
\par Assuming a reflector-like behaviour of the target, a spatial loop is generated, which is shown to achieve the desired \textit{spatial-IIR} beampattern characteristics.
\section{Spatial feedback}\label{sec_spatialFeedback}
We focus the discussion to a ULA of inter-element distance $d$, in the context of planar (i.e. 2D) localization. Consider an $N$ elements ULA, indexed by $n=\vBrace{0\hdots{}N-1}$, where it's $n\text{-th}$ element is positioned in $p_{n}$, measuring from an arbitrary reference array element (In this paper, chosen to be $p_{0}$). A target of interest, assumed to be static, is located in $p_{t}$, in a distance $R=\abs{p_{t}-p_{0}}$ at an angle of $\theta_{g}$, measured from the array plane.
\par In radar applications, a stimulus signal $x\rBrace{t}$ is emitted from the array, reflected by the target, re-impinges the array and processed to extract the spatial properties of the target. 
\par Assuming identical and omni-directional sensors, the measure in the $n\text{-th}$ array element is
\begin{equation}\label{eq_ULA_nth_measure}
x_{n,\theta_{g}}\rBrace{t}= g\rBrace{t-\tau_{pd}-\tau_{n,\theta_{g}}},
\end{equation}
where g represents attenuation between the emitted and the measured signals, $\tau_{pd}=2R/c$ ($c$ is the propagation velocity in the medium) is the round-trip propagation delay and $$\tau_{n,\theta_{g}}=d\cos{\rBrace{\theta_{g}}}/c$$ is the relative time-of-arrival (TOA) of the impinging signal at the $n\text{-th}$ sensor.
\par The $n-th$ element of the frequency domain steering vector $\vd_{\theta_{g}}$, which holds the spatial information of the impinged signal, is
\begin{equation}
    \vd_{\theta_{g}}\vBrace{n}=g\exp{\rBrace{-j\theta}},
\end{equation}
introducing $\theta\triangleq\omega\tau_{n,\theta_{g}}$ as the DOA related electrical phase of the impinging signal. 
\par Traditional delay-and-sum beamformers synthesize the array output by the individually weighting the signals from each sensor and summing them to create a spatially filtered signal, enhancing signals of specific spatial characteristics. Such scheme may be described by a weight set $\vBeta$ which is used to generate the beamformed signal $$\sum^{N-1}_{n=0}\vBeta\vBrace{n}\vd\vBrace{n}\F{x}\rBrace{\omega},$$ where $\omega$ is the radial frequency of the signal. As described in \cite{bbb}, this form is the spatial equivalent of the time-domain finite impulse response (FIR) filter.
\begin{figure}[t!]
    \begin{center}
        \begin{overpic}[width=0.95\linewidth, 
        % grid, 
        tics=10,trim={0 0 0 0}]{./Media/SpatialIIR-diagram/SpatialIIR_VER7.png}
            \put (1.5, 43.5){\footnotesize{$\beta_{0}$}}
            \put (12, 43.5){\footnotesize{$\beta_{1}$}}
            \put (30, 43.5){\footnotesize{$\beta_{N-1}$}}
            \put (48, 43.5){\footnotesize{$\alpha_{0}$}}
            \put (59, 43.5){\footnotesize{$\alpha_{1}$}}
            \put (80, 43.5){\footnotesize{$\alpha_{N-1}$}}
            \put (30.5, 66){\footnotesize{$d$}}
            \put (89, 96){\footnotesize{$p_{t}$}}
            \put (58, 58){\footnotesize{$p_{N-1}$}}
            \put (37, 58){\footnotesize{$p_{1}$}}
            \put (26.5, 58){\footnotesize{$p_{0}$}}
            \put (41.5, 64.5){\footnotesize{$\theta_{g}$}}
            \put (19, 27){$\Sigma$}
            \put (19, 11.25){\large{$+$}}
            \put (61.5, 27){$\Sigma$}
            \put (44.75, 11.75){$x\rBrace{t}$}
            \put (33.15, 11.75){$n\rBrace{t}$}
            \put (21,4){$y\rBrace{t}$}
            \put (63.5,14){$\text{Tx}\rBrace{t}$}
            \put (1, 51){$\text{FB}_{\vAlpha,\vBeta}$}
        \end{overpic}
    \end{center}
    \caption{The proposed feedback beamformer architecture consisting of ULA of inter-element distance $d$, where the target is assumed to reflect the impinged signal. We designate the feedback beamformer (FB) block (dashed line) for later use.}
    \label{fig_scheme}
\end{figure}
\par To generate a spatial feedback, inspired by the radar scheme, we propose to use an independent beamformed instance of the impinging signal as the feedback signal (see Fig.~\ref{fig_scheme}). Note that we assume that the target is reflective, such that the reflected signal generates a recursive spatial loop. Specifically, we add another weight set $\vAlpha$, such that the feedback signal will be $$\sum^{N-1}_{n=0}\vAlpha\vBrace{n}\vd\vBrace{n}\F{x}\rBrace{\omega}.$$
\par As described in \cite{myPaeper}, calculating the resultant output signal ($y$) yields 
\begin{equation}\label{eq_sysResp}
    H_{\vBeta,\vAlpha}
    =
    \F{y}\rBrace{\omega}/\F{x}\rBrace{\omega}
    =
    \frac{
    \vBetaT\vd\exp{\rBrace{-j\phi}}
    }{
    1-\vAlphaT\vd\exp{\rBrace{-j\phi}}
    },
\end{equation}
where $H_{\vBeta,\vAlpha}$ is the system response and $\phi=\omega_{\tau_{pd}}$ is the range related phase, thus creating a controllable spatial IIR (controllable denominator) response. 
\section{Generalizing the conventional beamformer}
In the absence of accurate knowledge of the target's location, we use estimated parameters when setting the coefficients, namely we use $\hat{\theta},\hat{\phi}$ and $\hat{g}$ to estimate the DOA-related phase, propagation related phase and the gain respectively.
We then express the estimated steering vector as \begin{equation}\label{eq_dHat}
    \hat{\vd}=\hat{g}\vBrace{1,\exp{\rBrace{-j\hat{\theta}}},\hdots,\exp{\rBrace{-j\rBrace{N-1}\hat{\theta}}}}^{T}.
\end{equation}
Information theory considerations in the context of localization \cite{myPaper}, namely evaluation of the spatial information in the feedback based architecture, lead us to conclude that $\vAlpha$ should be set such that $1-\vAlphaT\vd\exp{\rBrace{-j\phi}}$ is minimized.
\par Following those guidelines, we generalize the concept of the convectional beamformer (BF) \cite{VanTrees2002DetectionIV}. To this end, the spatially sampled signal (i.e. the array sensors outputs) is coherently summed, such that
\begin{equation}\label{eq_CB}
    \vBeta_{\text{CB}}=\vAlpha_{\text{CB}}
    =
    \hat{\vd}^{\ast}\exp{\rBrace{j\hat{\phi}}}/\norm{\hat{\vd}}^{2}
\end{equation}
Applying \eqref{eq_CB} to $H_{\vBeta,\vAlpha}$ of \eqref{eq_sysResp} yields
\begin{equation}\label{eq_sysRespCB}
    \resizebox{.9\linewidth}{!}{
        H_{\vBeta_{\text{CB}},\vAlpha_{\text{CB}}}
        =
        \frac{
        r\D{\dTheta/2}{N}\exp{\rBrace{-j\rBrace{\dPhi+\rBrace{N-1}\dTheta/2}}}
        }{
        1-r\D{\dTheta/2}{N}\exp{\rBrace{-j\rBrace{\dPhi+\rBrace{N-1}\dTheta/2}}}
        }
    },
\end{equation}
where $$ \D{x}{N} = \frac{1}{N}\frac{\sin{Nx}}{x} $$ is the normalized Dirichlet kernel and 
$$ \dTheta \triangleq \theta-\hat{\theta},\ \dPhi \triangleq \phi - \hat{\phi},\ r \triangleq g/\hat{g} $$ are defined to serve as the DOA, range and gain estimation errors respectively. Using estimation error terms, we define four fundamental scenarios:
\begin{itemize}
    \item{\makebox[.55\linewidth]{The perfectly aligned scenario \hfill} $\rBrace{\dTheta=0\ , \dPhi=0}$}
    \item{\makebox[.55\linewidth]{The steer error scenario \hfill} $\rBrace{\abs{\dTheta}>0\ , \dPhi=0}$}
    \item{\makebox[.55\linewidth]{The range error scenario \hfill} $\rBrace{\dTheta=0\ , \abs{\dPhi}>0}$}
    \item{\makebox[.55\linewidth]{The general scenario \hfill} $\rBrace{\abs{\dTheta}>0\ , \abs{\dPhi}>0}$}.
\end{itemize}
\section{Performance analysis}\label{sec_performance}
To evaluate the presented architecture's performance, we follow the step of traditional array analysis \cite{VanTrees2002DetectionIV} and compare the results to the feedback based architecture.
\par Aiming for an objective comparison, independent of the coefficient's magnitude, we define the normalized response $\Hr$ such that its maximal gain is 1 (i.e. $0_{\text{dB}}$), such that
\begin{equation}\label{eq_Hr}
    \Hr_{\dTheta,\dPhi,r}\rBrace{\omega}
    \triangleq
    \frac{
    H_{\vBeta_{\text{CB}},\vAlpha_{\text{CB}}}
    }{
    H_{\vBeta,\vAlpha}
    }
    =
    \frac{
    H_{\vBeta_{\text{CB}},\vAlpha_{\text{CB}}}
    }{
    r/\rBrace{1-r}
    },
\end{equation}
where $\vBeta,\vAlpha$ represent the perfect alignment scenario coefficients.
\par Aiming for comparable results to traditional array analysis, the following evaluation is conducted assuming the steer-error scenario, where the range related phase is perfectly known. 
Also, where possible, the $\omega$ dependency will be suppressed in the notation throughout the rest of the exposition.
\par Next, we investigate several array performance key criteria, commonly used in traditional array analysis.
Specifically, we address the beampattern's beamwidth, it's side-lobes level and the array directivity, following the exact definitions as in \cite{VanTrees2002DetectionIV}.
\subsection*{Half power beamwidth}
The Half-power beamwidth, being the azimuthal distance between two DOA's in which the normalized beampattern's gain is $-3_{dB}$, is a common measure for the spatial selectivity of the array.
\par The known \cite{VanTrees2002DetectionIV} result for traditional ULAs is
\begin{equation}\label{eq_classicHPBW}
    \frac{\dTheta_{HPBW}}{2}=\frac{1.4}{N},
\end{equation}
stating that as the array aperture increases (i.e. adding array elements), the spatial selectivity improves.
Numerical evaluation of $\Hr_{\dTheta,\dPhi=0,r}$ results in
\begin{equation}\label{eq_fbHPBW}
    \frac{\dTheta_{HPBW}}{2}=\frac{1.4}{B\rBrace{r}N},
\end{equation}
where 
\begin{equation}\label{eq_fbHPBW_Br}
    B\rBrace{r}=\frac{1}{\rBrace{1-r}\rBrace{-0.4r+1.4}}.
\end{equation}
$B\rBrace{r}/1.4$ may be considered to be the feedback related aperture improvement factor, where accurate gain estimation (i.e. $r\to1$) may lead to significant aperture improvement without adding array elements.
\subsection*{Sidelobes attenuation}
Another important and commonly used performance criterion is the attenuation between the beampattern's main-lobe and the first side-lobe, where higher attenuation means better spatial performance.
The feedback based system response's sidelobes are located at the exact DOA's as in classic beampattern \cite{myPaper}.
\par Evaluating $\Hr_{\dTheta,\dPhi=0,r}$ at the first side-lobe DOA and comparing to the known \cite{VanTrees2002DetectionIV} result of non-feedback based array enables the expression of the improvement factor as $1/\rBrace{1-r}$.
Again, accurate gain matching may significantly increase the side-lobe attenuation.
\subsection*{Array directivity}
The last performance criterion to be evaluated, is the array directivity $D$, measuring the ration between the beampattern's energy in the main-lobe DOA and the averaged energy across all directions.
Comparing to the known result of $D=N$ for classic arrays, numerical analysis of $\Hr_{\dTheta,\dPhi=0,r}$ \cite{myPaper} concludes that for feedback based arrays, \begin{equation*}
    D\rBrace{r}=\frac{r^{2}-\rBrace{N-1}r+N}{\rBrace{1-r}^2}.
\end{equation*}
One may observe that for large $N$, again, the improvement is $1/\rBrace{1-r}$, predicting high improvement for gain matched scenarios.
\subsection*{Summary}
In Tbl.~\ref{table_arrayPerformance}, the feedback integration related improvement of the performance criteria is summarized, emphasizing the significant affect of gain matching in terms of spatial performance. 
\begin{table}[h!]
    \caption{Comparing performances of classical ULA and the proposed feedback based architecture, with gain mismatch $r$.}
    \centering
    % \resizebox{1\linewidth}{!}{
        \begin{tabular}{||c c||} 
            \hline
            PARAMETER & IMPROVEMENT \\ [0.5ex] 
            \hline\hline
            HPBW & $1.4B\rBrace{r}$ times smaller\\ 
            \thead{FIRST\\SIDELOBE\\GAIN} & $1/\rBrace{1-r}$ \\
            DIRECTIVITY & $1/\rBrace{1-r}$ \\
            [1ex] 
            \hline
         \end{tabular}
    %  }
    \label{table_arrayPerformance}
\end{table}
\section{The dual frequency scheme}\label{sec_df}
\begin{figure}[t!]
    \begin{center}
        \begin{overpic}[width=0.7\linewidth, 
        % grid, 
        tics=10,
        % trim={<left> <lower> <right> <upper>}
        trim={0 0 0 0}
        ]{./Media/rangeError_r06.eps}
            \put (-1, 74) {\footnotesize{$10\log_{10}\abs{\Hr_{\dTheta,\dPhi,r}}^2$}}
            \put (46, -1) {\footnotesize{$\dTheta/\pi$}}
            \put (0, 37) {\footnotesize{dB}}
            \put (92, 40) {\footnotesize{$\Delta R_{rt}=0.1\lambda$}}
            \put (92, 50) {\footnotesize{$\Delta R_{rt}=0.3\lambda$}}
            \put (92, 30) {\footnotesize{$\Delta R_{rt}=0$}}
        \end{overpic}
    \end{center}
    \caption{Evaluation of the array response for several values of range error $\Delta R_{rt}$ (where $r=0.4$). Even minor range errors significantly distort the beampattern.}
  \label{fig_sensitivity}
\end{figure}
Unfortunately, although feedback integration achieves high performance improvement in the steer-error scenario, we find that range related phase mismatching (i.e. $\abs{\dPhi}>0$) greatly distorts the beampattern (see Fig.~\ref{fig_sensitivity}).
Simulations \cite{myPaper} have shown that even small error (tenth of a wavelength) significantly distort the beampattern, rendering the feedback integration to be impractical.
\begin{figure}[t!]
    \begin{center}
        \begin{overpic}[width=0.95\linewidth, 
        % grid, 
        tics=10,trim={0 0 0 0}]{./Media/SpatialIIR_APP.png}
            \put (12.5, 64){$\text{FB}_{\vAlpha_{1},\vBeta_{1}}$}
            \put (61, 64){$\text{FB}_{\vAlpha_{2},\vBeta_{2}}$}
            \put (4.5, 59){$y_{1}\rBrace{t}$}
            \put (54, 59){$y_{2}\rBrace{t}$}
            \put (12.9, 41.5){$\text{BPF}_{\omega_{1}}$}
            \put (61.65, 41.5){$\text{BPF}_{\omega_{2}}$}
            \put (6, 51.5){+}
            \put (55, 51.5){+}
            \put (16, 51.5){\footnotesize{$n_{1}\rBrace{t}$}}
            \put (64.75, 51.5){\footnotesize{$n_{2}\rBrace{t}$}}
            \put (24.5, 59){\footnotesize{$\text{Tx}_{1}\rBrace{t}$}}
            \put (73.5, 59){\footnotesize{$\text{Tx}_{2}\rBrace{t}$}}
            \put (36.25, 64){\scriptsize{$x_{1}\rBrace{t}$}}
            \put (43, 64){\scriptsize{$x_{2}\rBrace{t}$}}
            \put (18.25, 13){\footnotesize{Harmonic mean}}
            \put (32, 2){$y\rBrace{t}$}
            \put (54.75, 24){$\Sigma$}
            \put (21.75, 25){\footnotesize{$\mathcal{F}_{\omega_{1}}$}}
            \put (30.75, 25){\footnotesize{$\mathcal{F}_{\omega_{2}}$}}
        \end{overpic}
    \end{center}
    \caption{Dual-frequency beamformer, consisting of two independent FB blocks and narrowband bandpass filters. The blocks marked by $\mathcal{F}_{\omega_{i}}$ compute the Fourier coefficients in $\omega_{i}$ and their outputs feed the harmonic mean calculator which generates the DF beamformer's output.}
    \label{fig_DF}
\end{figure}
\begin{figure}[t!]
    \begin{center}
        \begin{overpic}[width=.7\linewidth, 
        % grid, 
        tics=10,trim=0 0 0 0]{./Media/fig_dualfreq_rangeErrorHighSnr.eps}
            \put (48, 43){\scriptsize{Ideal}}
            \put (48, 37.5){\scriptsize{SF}}
            \put (48, 32){\scriptsize{DF}}
            \put (2, 37.5){\footnotesize{dB}}
            \put (47,0){\footnotesize{$\dTheta/\pi$}}
            \put (92,58){\footnotesize{$\Delta{}R_{rt}=0.3\lambda$}}
        \end{overpic}
    \end{center}
    \caption{Simulating 3 element ULA with $r_1=0.6^{2},\; r_2=0.6$ (hence $\kappa=0.6$) for infinite SNR. The (fractional) range error is $\Delta{}R_{rt}=0.3\lambda$.
    The ideal response is obtained for $\Delta{}R_{rt}=0$ (blue dots), with the single frequency (SF) beamformer (red squares) and the  dual-frequency (DF) solution (green diamonds). 
    }
    \label{fig_DF_BP}
\end{figure}
\par Aiming to mitigate the aforementioned sensitivity, we propose the use of two separate frequencies. Using spectral separation of the signals and two independently configured feedback beamformers (see Fig.\ref{fig_DF}), we show \cite{myPaper} that with an additional frequency domain processing, one may achieve a significant sensitivity reduction. In Fig.~\ref{fig_DF_BP} the general scenario was simulated with both SF and DF architectures, where the DF proves to be robust in terms of range estimation errors.
\section{Conclusions}\label{sec_conc}
In this contribution we have presented a novel concept of spatial feedback-based beamforming. 
In the context of localization with ULAs, this approach may be considered to be an extension of FIR, to IIR filter design.
We have analyzed and shown that the feedback-beamformer is superior to the conventional delay and sum beamformer,
achieving improved beamwidth, sidelobe attenuation and directivity.
The suggested approach revealed itself to be sensitive to
target range errors. 
Hence, we have suggested to mitigate this sensitivity by using two independent and closely spaced frequencies.
This paper lays the ground for a diverse future research, starting from different approaches of setting the beamformer weights, waveform design, localization performance analysis, extension to moving targets and generalization to multiple-input-multiple-output (MIMO) systems.
\bibliographystyle{IEEEtran}
\bibliography{./Modules/Mendeley,./Modules/LocalBib}
\end{document}