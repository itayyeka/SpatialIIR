An intuitive understanding of the spatial feedback concept may be achieved when inspecting one of the two main IIR filter architectures, known as the ``\textit{Direct form II}" (see Fig.~\ref{fig_IIRBasicArch}). In this architecture, the recursive part of the filter is implemented by feeding the input $x$ with it's delayed and independently weighted (with $\vAlpha$ weights) instances before entering the FIR part (defined by $\vBeta$).
\begin{figure}[t!]
    \begin{center}
        \begin{overpic}[width=0.65\linewidth, 
        % grid, 
        tics=10,trim=0 0 0 0]{./Media/BASIC_IIR_FILTER_ARCH.png}
            \put (60, 50){\footnotesize{$\beta_{0}$}}
            \put (60, 30){\footnotesize{$\beta_{1}$}}
            \put (60, 10){\footnotesize{$\beta_{2}$}}
            \put (36, 30){\footnotesize{$\alpha_{1}$}}
            \put (36, 10){\footnotesize{$\alpha_{2}$}}
            \put (10, 50){\footnotesize{$x$}}
            \put (85, 50){\footnotesize{$y$}}
            \put (47.5, 34.5){\footnotesize{$z^{-1}$}}
            \put (47.5, 14.5){\footnotesize{$z^{-1}$}}
        \end{overpic}
    \end{center}
    \caption{\textit{Direct form II} $2^{nd}$ order IIR architecture.}
    \label{fig_IIRBasicArch}
\end{figure}
Inspired by this architecture, we seek for an analogous structure which will mimic the feedback loop in the spatial domain. 
In order to do so, we propose a feedback beamformer architecture (see Fig.~\ref{fig:Proposed_spatialIIR_ARCH}) where the weights $\vBeta$ generate the output signal $z$, the $\vAlpha$ weights are used to synthesize the feedback transmission ($\text{Tx}$) and $s$ is the input waveform. An additive noise model, denoted as n, is assumed at the array's output. 
Note that without feedback (i.e. setting $\vAlpha=\vecnot{0}$) the beamformed signal, acts as the DS beamformer. 
\begin{figure}[t!]
    \begin{center}
        \begin{overpic}[width=0.95\linewidth, 
        % grid, 
        tics=10,trim={0 0 0 0}]{./Media/SpatialIIR-diagram/SpatialIIR_VER7.png}
            \put (1.5, 43.5){\footnotesize{$\beta_{0}$}}
            \put (12, 43.5){\footnotesize{$\beta_{1}$}}
            \put (30, 43.5){\footnotesize{$\beta_{N-1}$}}
            \put (48, 43.5){\footnotesize{$\alpha_{0}$}}
            \put (59, 43.5){\footnotesize{$\alpha_{1}$}}
            \put (80, 43.5){\footnotesize{$\alpha_{N-1}$}}
            \put (30.5, 66){\footnotesize{$\delta$}}
            \put (89, 96){\footnotesize{$p_{t}$}}
            \put (58, 58){\footnotesize{$p_{N-1}$}}
            \put (37, 58){\footnotesize{$p_{1}$}}
            \put (26.5, 58){\footnotesize{$p_{0}$}}
            \put (41.5, 64.5){\footnotesize{$\theta_{g}$}}
            \put (19, 27){$\Sigma$}
            \put (19, 11.25){\large{$+$}}
            \put (61.5, 27){$\Sigma$}
            \put (44.75, 11.75){$s\rBrace{t}$}
            \put (33.15, 11.75){n$\rBrace{t}$}
            \put (21,4){$z\rBrace{t}$}
            \put (63.5,14){$\text{Tx}\rBrace{t}$}
            \put (1, 51){$\text{FB}_{\vAlpha,\vBeta}$}
        \end{overpic}
    \end{center}
    \caption{The proposed feedback beamformer architecture consisting of ULA of inter-element distance $\delta$, where the target is assumed to reflect the impinged signal. We designate the feedback beamformer (FB) block (dashed line) for later use.}
    \label{fig:Proposed_spatialIIR_ARCH}
\end{figure}
\subsection*{Obtained spatial response}
Time domain analysis of the proposed feedback based architecture, considering both propagation delay and attenuation, gives rise to
\begin{equation}
    \label{eqn:SingleSensorTemporalEquality}
    % \resizebox{.91\linewidth}{!}{
        \begin{split}
            x_{n}(t) = g\rBrace{s\rBrace{t-\tau_{pd}-\tau_{n}}
            +\sum_{m=0}^{N-1}{\alpha_{m}x_{m}\rBrace{t-\tau_{pd}-\tau_{n}}}},
        \end{split}
    % }
\end{equation}
where the first term of the right-hand side represents the contribution of the transmitted waveform $s(t)$ to the $n$'th array element and the second term represents the feedback contribution of the re-transmitted array signal to this same element.
Expressing \eqref{eqn:SingleSensorTemporalEquality}'s Fourier transform,
\begin{equation}
    \label{eqn_singleSensorFourier}
    % \resizebox{.91\linewidth}{!}{
        \begin{split}
            \F{x}_{n}\rBrace{\omega} =
            g\Bigg( & \F{s}\rBrace{\omega}
            \exp\rBrace{-j\omega\rBrace{\tau_{pd}+\tau_{n}}}
            \\&+\sum_{m=0}^{N-1}
            {
            \alpha_{m}\rBrace{\omega}\F{x}_{m}\rBrace{\omega}
            \exp\rBrace{-j\omega\rBrace{\tau_{pd}+\tau_{n}}}
            }\Bigg),
        \end{split}
    % }
\end{equation}
and its vector from,
$$
\F{\vx}\rBrace{\omega} = ge^{-j\omega\tau_{pd}} \rBrace{\F{s}\rBrace{\omega}+\vAlphaT \F{\vx}\rBrace{\omega}}\vd,
$$
we find that it can be simplified to
$$
\F{\vx}\rBrace{\omega} =\rBrace{I-g\vd\vAlphaT{}e^{-j\omega\tau_{pd}}}^{-1}g\vd\exp{\rBrace{-j\omega\tau_{pd}}}\F{s}\rBrace{\omega}.
$$
Then, denoting
\[
\phi\triangleq\omega\tau_{pd}
\]
as the round-trip signal propagation related electrical phase and using the Woodbury matrix identity \cite{woodbury1950inverting}, we find that
$$
\F{\vx}\rBrace{\omega}
=
\frac{    
g\vd\exp{\rBrace{-j\phi}}
}{
1 - g\aTd{}\exp{\rBrace{-j\phi}}
}\F{s}\rBrace{\omega}.
$$
Considering the noiseless case $\rBrace{\text{i.e. n}\rBrace{t}=0}$,
we express the general spatial response of FB as 
\begin{equation}
\label{eqn:GeneralFeedbackTransferFunction}
\Hba
\triangleq
\frac{\F{z}\rBrace{\omega}}{\F{s}\rBrace{\omega}} 
=
\frac{    
g\bTd{}\exp\rBrace{-j\phi}
}{
1 - g\aTd{}\exp\rBrace{-j\phi}
}.
\end{equation}
\par Note that this result confirms that our suggested array architecture achieves a controllable (via setting of $\vBeta$ and $\vAlpha$) and recursive (non-trivial denominator) spatial response.
As will be shown, high directivity and narrow beam-width are obtainable by proper selection of the weights. Comparing to traditional beamformers (i.e. with no feedback), the performance improvement will be expressed in terms of aperture increase, expressing the traditional beamformer aperture which achieves the same performance.
One may observe that opposed to traditional beamformers, the beampattern, $\Hba,$ is not only influenced by the impinging signal DOA, for it is also range selective due to its dependency on the phase parameter $\phi$.
As exemplified in Fig.~\ref{fig_rangeAzimuthSelectivity}, the combination of both angular and range selectivity enables the designer to enhance signals arriving from specific locations rather than only specific directions.
\begin{figure}[t!]
    \begin{center}
        \begin{overpic}[width=0.65\linewidth, 
        % grid, 
        tics=10,trim=0 0 0 0]{./Media/azimuthRangSelectivity.png}
            \put (20, 23){\rotatebox{0}{\footnotesize{Angular response}}}
            \put (30.5, 47){\rotatebox{0}{\footnotesize{Enhanced radial slice}}}
        \end{overpic}
    \end{center}
     \caption{A visualization of the spatial area selectivity concept. Combining both radial selectivity (i.e. enhancing signals from a specific distance) and DOA-based selectivity, allows the enhancement of signals arriving from specific areas (grey filled), while signals originated in other areas (even from the same DOA) are suppressed.}
    \label{fig_rangeAzimuthSelectivity}
\end{figure}
\section{Fisher Information Matrix}
\label{sec_FIM}
A possible evaluation for the contribution of the presented feedback mechanism is to measure the additional information in the system.
To this end, the FIM will now be calculated with respect to the DOA parameter $\thetaD$ and the range related parameter $\phi$. 
As the feedback-based transfer function (\ref{eqn:GeneralFeedbackTransferFunction}) is expressed in the frequency domain, we rely on \cite{zeira1990frequency}, to express the FIM in the frequency domain as well. 
\par A single element of the FIM, notated by the matrix $J$, can be expressed as
\begin{equation}
    \resizebox{.9\linewidth}{!}{
        \begin{split}
            J_{\vBrace{k,l}}\rBrace{\vEta} 
            =&
            \Re\cBrace{
            \frac{1}{2\pi}
            \int_{-\omega_{s}/2}^{\omega_{s}/2}
            {
            \frac{1}{\Phi\rBrace{\omega}}
            \mathfrak{F}^{*}\left\{
            \frac{\partial z(t)}{\partial\eta_{k}}
            \right\}
            \mathfrak{F}\left\{
            \frac{\partial z(t)}{\partial\eta_{l}}
            \right\}
            d\omega
            }}
            \\ &+
            \frac{T}{4\pi}
            \int_{-\omega_{s}/2}^{\omega_{s}/2}
            \frac{1}{\Phi^{2}\rBrace{\omega}}
            \frac{\partial\Phi\rBrace{\omega}}{\partial\eta_{k}}
            \frac{\partial\Phi\rBrace{\omega}}{\partial\eta_{l}}
            d\omega
        \end{split}
    }
\end{equation}
where $ \vEta = [\thetaD,\phi]^{T} $ is the parameters vector, $\Re$ stands for the real-part extraction operator, $k,l \in\cBrace{1,2}$, $\Phi\rBrace{\omega}$ is the noise spectrum, $\mathfrak{F}$ is the Fourier transform operator, $T$ is the measurement observation interval and $\omega_{s}$ is the signal bandwidth. 
For simplicity, $\text{n}\rBrace{t}$ is assumed to be white and Gaussian with some constant power spectral density $\Phi(\omega)=\sigma^2$ and does not depend on the estimated parameters $\eta$, hence the second term vanishes. 
Assuming continuously differentiable functions, where order alteration of the Fourier transform and the differentiation operations is allowed, the FIM's $\vBrace{k,l}$'th element is
\begin{equation}
    \label{eq_beamPatternFreqDomain_FIM}
    % \resizebox{1\linewidth}{!}{
        \begin{split}
            J_{\vBrace{k,l}}\rBrace{\vEta} = 
            \Re\cBrace{
            \frac{1}{2\pi\sigma^2}
            \int_{-\omega_{s}/2}^{\omega_{s}/2}
            {
            \rBrace{\frac{\partial{}\F{z}\rBrace{\omega}}{\partial\eta_{k}}}^{\ast}
            \frac{\partial{}\F{z}\rBrace{\omega}}{\partial\eta_{l}}
            d\omega
            }}
        \end{split}.
    % }
\end{equation}
For omni-directional array sensors, $g$ is independent of the estimated parameters, therefore
\begin{equation}\label{eq_vdDiff}
\frac{\partial\vd}{\partial\thetaD}=A\rBrace{\omega}\vd
\end{equation}
where $A\rBrace{\omega}$ is an $N\times{}N$ diagonal matrix and each of its diagonal elements may expressed as 
\[
A_{\vBrace{i,i}}\rBrace{\omega}=-j\omega\frac{\partial \tau_{i}}{\partial{\thetaD}}\ \  \forall{i\in\cBrace{0\hdots{}N-1}}.
\]
To further simplify the analysis, without loss of generality, we use (in this section only) $g=1$.
In App.~\ref{apdx_clacFim} we compute the FIM terms, concluding that
\begin{equation}
    \label{eqn_FIMelements}
    \resizebox{.91\linewidth}{!}{
        \begin{split}
            &J_{\theta\theta}
            =
            \frac{1}{2\pi\sigma^{2}}\int_{-\omega_{s}/2}^{\omega_{s}/2}{\frac{
            \lBrace{\vBetaT{}A\rBrace{\omega}\vd-\vBetaT{}B\rBrace{\omega}\vAlpha\ePhi{-}}^{2}
            }{
            \lBrace{\rBrace{1-\aTd\ePhi{-}}^{2}}^{2}
            }\lBrace{\F{s}\rBrace{\omega}}^{2}d\omega}
            \\
            &J_{\phi\phi}
            =
            \frac{1}{2\pi\sigma^{2}}\int_{-\omega_{s}/2}^{\omega_{s}/2}{\frac{
            \lBrace{\bTd}^{2}
            }{
            \lBrace{\rBrace{1-\aTd\ePhi{-}}^{2}}^{2}
            }\lBrace{\F{s}\rBrace{\omega}}^{2}d\omega}
        \end{split}
    }
\end{equation}
where $B\rBrace{\omega}\triangleq\vd\vdT{}A\rBrace{\omega}-A\rBrace{\omega}\vd\vdT$.
Also, using some mild assumptions and setting
\begin{equation}\label{eq_alphaBetaPropSteer}
    \vAlpha,\vBeta\propto\vd^{\ast},
\end{equation}
we show that the cross terms of the FIM are nullified, i.e. $J_{\theta\phi} = J_{\phi\theta}^{*}=0$.
\par 
Choosing the weights as in \eqref{eq_alphaBetaPropSteer} may be interpreted as a generalization of the conventional beamformer (CB) \cite{van2004optimum}, more commonly referenced as the delay-and-sum (DS) beamformer which coherently integrate the impinging signal along the array elements.
The same choice of weights also minimizes the $\lBrace{1-\aTd\ePhi{-}}$ term, significantly increasing the available information, as predicted by the FIM.
It is worth mentioning that \eqref{eq_vdDiff} is relevant even for arbitrary (non-omni-directional) sensors when smooth and slowly changing radiation patterns are assumed.
In practice, though, there will be unavoidable errors, and perfect knowledge of the steering vector $\vd$ is not always available.
In Sec.~\ref{sec_Performance}, we quantify the effect of mismatching $\vAlpha$ with the desired form and discuss its influence on the array performance. 
\section{Performance Analysis}
\label{sec_Performance}
In this section we analyze some of the fundamental properties of the suggested array; it's beamwidth, peak to side-lobe level and it's directivity which are all compared to traditional array processing. Focusing on ULA, we show that by integrating the spatial feedback, we obtain improved performance compared to classic beamforming.
\subsection*{Error terms}
Following previous observations, we analyze the case of $\coefSetName$ (i.e. coherently sum the array elements) using
\begin{equation}\label{eq:alpha_beta_hat}
\vBeta_{\coefSetName{}}=\vAlpha_{\coefSetName{}}=\frac{\hat{\vd}^{\ast}\exp\rBrace{j\hat{\phi}}}{\hat{g}\norm{\hat{\vd}}^2},
\end{equation}
where
\begin{equation}\label{eq:d_hat}
\hat{\vd}=\sBrack{1,\exp(-\hat\theta),\ldots,\exp(-(N-1)\hat\theta)}^T
\end{equation}
is denoted to serve as the steering vector estimation and $\hat{g},\hat{\phi},\hat{\theta}$ are the gain, range-related phase and the DOA-related phase estimations respectively.
Plugging \eqref{eq:alpha_beta_hat} within \eqref{eqn:GeneralFeedbackTransferFunction}, we obtain
\begin{equation}\label{eq:SF_CB}
    \resizebox{.894\linewidth}{!}{
        \begin{split}
            H_{\vBeta_{\coefSetName{}},\vAlpha_{\coefSetName{}}}\rBrace{\omega}=\frac{r\D{\dTheta/2}{N}\exp\rBrace{-j\rBrace{\dPhi+(N-1)\dTheta/2}}}{1-r\D{\dTheta/2}{N}\exp\rBrace{-j\rBrace{\dPhi+(N-1)\dTheta/2}}}
        \end{split}
    }
\end{equation}
where \[
\D{x}{N}\triangleq\frac{1}{N}\frac{\sin\rBrace{Nx}}{\sin\rBrace{x}}
\]
is the normalized Dirichlet kernel and
we define the DOA, range and gain error terms 
\[
\dTheta\triangleq {\theta-\hat{\theta}},\ \dPhi\triangleq {\phi-\hat{\phi}},\ 
r\triangleq g/\hat{g},
\]
respectively. 
We then define four fundamental scenarios:
\begin{itemize}
    \item{\makebox[.55\linewidth]{The perfectly aligned scenario \hfill} $\rBrace{\dTheta=0\ , \dPhi=0}$}
    \item{\makebox[.55\linewidth]{The steer error scenario \hfill} $\rBrace{\abs{\dTheta}>0\ , \dPhi=0}$}
    \item{\makebox[.55\linewidth]{The range error scenario \hfill} $\rBrace{\dTheta=0\ , \abs{\dPhi}>0}$}
    \item{\makebox[.55\linewidth]{The general scenario \hfill} $\rBrace{\abs{\dTheta}>0\ , \abs{\dPhi}>0}$}.
\end{itemize}
\ifdefined\showTodo
{
    \subsection*{Small estimation error analysis - \textbf{TO BE REMOVED - kept only for the todos}}
    \label{subsection_ArrayPerformance_TayolrAnalysis}
    \myTodo{inline}{\textbf{DONE:}\\ maybe to call this subsection 'small error analysis'}
    Plugging (\ref{eqn_CB_coefSet}) into (\ref{eqn:GeneralFeedbackTransferFunction}) and denoting $\D{N}{x} \triangleq \frac{\sin{\rBrace{Nx}}}{\sin{\rBrace{x}}}$ results in the general \coefSetName{} beampattern \myTodo{inline}{\textbf{DONE}\\you need to define the delta expressions (the errors)}
    \begin{equation*}
        h_{\coefSetName{}}\rBrace{\theta,\omega}
        =
        \frac{
        \D{N}{\dTheta/2}exp\rBrace{-j\rBrace{\dPhi+\frac{N-1}{2}\dTheta}}
        }{
        N - \D{N}{\dTheta/2}exp\rBrace{-j\rBrace{\dPhi+\frac{N-1}{2}\dTheta}}
        }.
    \end{equation*}
    For the evaluation of the various array parameters, we define the normalized beampattern \myTodo{inline}{\textbf{DONE:}\\not very readable. Consider writing a multiplication of two terms. Suggesting that you write the expression in eq(5) as $H(\theta,\phi)$ and here you'll have $H(\hat{\theta}, \hat{\phi})/ H(\theta,\phi)$}
    \begin{equation}
        \label{eqn_arrPerformance_beamwidth_3dB}
        \Hr{\theta}{\tau}{}\triangleq\fbBpRatio.
    \end{equation}
    \myTodo{inline}{\textbf{DONE:}\\this next section is very long. Suggesting that you express (10) in terms of the errors $\Delta\theta,\;\Delta\phi$, and then simply state that the second order Taylor expansion of those errors around zero gives (12). One more issue which I think might cause us some headache is that we cannot assure that $\Delta\phi$ (you call it $\Delta\tau$) is indeed close to zero, as this term fluctuates very fast. But we will think about it later on. Maybe something with wideband signal stimulus will solve this issue.}
    The pursuit for analytic dependencies between $\Hr{\theta}{\tau}{}$, $\dTheta$ and $\dPhi$, as in \cite{van2004optimum}, lead to expressing $\Hr{\theta}{\tau}{}$'s multivariate Taylor expansion, setting $\dTheta,\dPhi$ as the variables. Simulations have shown that $2^{nd}$ Taylor expansion achieves very accurate results, thus we use it to express the array parameters. As commonly known, the Taylor expansion of a multivariate analyzable function $f\rBrace{\vx}$ around $\vx_{0}$ where $\vx \in \mathbb{R}_{M\times1},$ is 
    \begin{equation}
        \label{eqn_h_Tylor_dTheta_dTau}
        \evalat{f\rBrace{\vx}}{\vx\to\vx_{0}}=\sum_{n=0}^{\infty}\frac{1}{n!}\rBrace{\sum_{i=1}^{M}(x_{i}-x_{0i})\frac {\partial}{\partial x_i} }^n f(x_k)|_{x_k=x_{k0}},
    \end{equation}
    where $\frac{\partial}{\partial x_i}$ is the derivative operator and $i\in\left[1\hdots{}M\right]$. Reducing (\ref{eqn_h_Tylor_dTheta_dTau}) to its $2^{nd}$ form (i.e $f(x,y)=\sum_{n=0}^{\infty} \frac 1 {n!}\rBrace{x\frac {\partial}{\partial x}+y\frac {\partial}{\partial y} }^n f(x,y)|_{(x,y)=(0,0)}$), combined with the binomial formula, $(x+y)^{n}=\sum _{k=0}^{n}{\binom {n}{k}}x^{n-k}y^{k},$ multiple iterations of L'Hôpital's rule and algebraic simplification finally yields
    \begin{equation}
        \begin{split}
            \evalat{\Hr{\theta}{\tau}{}_{\coefSetName{}}}{\dTheta\to0,\dPhi\to0} \approx\ & 1 
            \\+&\frac{\binom{2}{0}}{2!}\frac{-\rBrace{N - 1}\rBrace{N-4r+2Nr+1}}{6\rBrace{r-1}^{2}}\dTheta^{2}
            \\+&\frac{\binom{2}{1}}{2!}\frac{-r\omega\rBrace{N - 1}}{\rBrace{r-1}^{2}}\dTheta\dPhi
            \\+&\frac{\binom{2}{2}}{2!}\frac{-2r\omega^{2}}{\rBrace{r-1}^{2}}\dPhi^{2}
        \end{split}
    \end{equation}
}
\else
\fi
\subsection*{The normalized beampattern}
\label{subsection_spatialIIR_normBP}
As commonly done for ULA parameter analysis \cite{van2004optimum}, we focus on the normalized response (i.e. where the peak main lobe gain is set to $0_{dB}$). Hence forth, we define $\Hr$ to be the normalized pattern
\begin{equation}
    \label{eq_narmalized_pattern}
    %\resizebox{.89\linewidth}{!}{
    \begin{split}
        \Hr_{\dTheta,\dPhi,r}\rBrace{\omega}&\triangleq
        \frac{
        H_{\vBeta_{\coefSetName{}},\vAlpha_{\coefSetName{}}}\rBrace{\omega}
        }{
        H_{\vBeta_{\text{opt}},\vAlpha_{\text{opt}}}\rBrace{\omega}
        }
         =
         \frac{
        H_{\vBeta_{\coefSetName{}},\vAlpha_{\coefSetName{}}}\rBrace{\omega}
        }{
        r/\rBrace{1-r}
        }
        % \\
        % &=\frac{\rBrace{1-r}^{2}\Dp{\dTheta/2,N}{2}}{1+r^{2}\Dp{\dTheta/2,N}{2}-2r\D{\dTheta/2}{N}\cos{\rBrace{\dPhi+\frac{N-1}{2}\dTheta}}},
    \end{split}
    %}
\end{equation}
where $\vBeta_{\coefSetName{}},\vAlpha_{\coefSetName{}}$ are the weights as in \eqref{eq:alpha_beta_hat} and $\vBeta_{\text{opt}},\vAlpha_{\text{opt}}$ are optimal weights assuming exact knowledge of the steering vector and the phase $\phi$ (i.e. the perfectly aligned scenario), 
\[
\vBeta_{\text{opt}}=\vAlpha_{\text{opt}}={\vd}^{\ast}\exp\rBrace{j{\phi}}/g\norm{{\vd}}^2
\]
and the subscript of $\dTheta,\dPhi$ and $r$ emphasize its DOA, range and gain errors dependency.
% where that as $\vBeta_{\coefSetName{}},\vAlpha_{\coefSetName{}}$ are functions of the DOA, range and gain errors, we add to $\Hr$ the subscript of $\dTheta,\dPhi$ and $r$. 
Explicitly writing $\Hr$ gives rise to 
\begin{equation}\label{eq_generalH}
    \resizebox{0.89\linewidth}{!}{
        \begin{split}
             \Hr_{\dTheta,\dPhi,r} \rBrace{\omega}=
             \frac{\rBrace{1-r}\D{\dTheta/2}{N}}{\exp\rBrace{j\rBrace{\dPhi+(N-1)\dTheta/2}}-r\D{\dTheta/2}{N}}
        \end{split}
        }
\end{equation}
where the normalized pattern of traditional ULA may be achieved by setting $r=0$ (i.e. no feedback)
\[
\Hr_{\dTheta,\dPhi=0,r=0}\rBrace{\omega}=
             \D{\dTheta/2}{N}\exp\rBrace{-j\rBrace{(N-1)\dTheta/2}}.
\]
Next, we evaluate the FB's beamwidth, sidelobe level and directivity and compare them to standard ULA considering the steer error scenario (i.e. $\dPhi=0$) where 
\begin{equation}\label{eq_Hdphi0}
\Hr_{\dTheta,\dPhi=0,r}\rBrace{\omega}=
             \frac{\rBrace{1-r}\D{\dTheta/2}{N}}{\exp\rBrace{j\rBrace{(N-1)\dTheta/2}}-r\D{\dTheta/2}{N}}.
\end{equation}
% We now focus on the steer error scenario, i.e. $\dPhi=0$, and compare the beamwidth, sidelobe level and directivity between  
% \begin{equation}\label{eq_Hdphi0}
% \Hr_{\dTheta,\dPhi=0,r}\rBrace{\omega}=
%              \frac{\rBrace{1-r}\D{\dTheta/2}{N}}{\exp\rBrace{j\rBrace{(N-1)\dTheta/2}}-r\D{\dTheta/2}{N}}
% \end{equation}
% and the standard ULA beampattern, $\Hr_{\dTheta,\dPhi=0,r=0}$. 
To simplify the exposition, we shall suppress the $\omega$ dependency in the following sections (i.e. $\Hr_{\dTheta,\dPhi,r} \triangleq \Hr_{\dTheta,\dPhi,r}\rBrace{\omega}$) where possible.

\subsection*{Half power beamwidth}
The Half-Power-Beam-Width (HPBW) parameter quantifies the array's main lobe narrowness, marking the DOA where the beampattern's energy reduces to half of its maximal value.
For standard ULA, with aperture of $N\delta$, it is known \cite{van2004optimum} that for large $N$,
$$
 \frac{\lambda\dThetaHPBW}{2\pi{}\delta} = \frac{\lambda}{\pi{}N\delta}1.4,
$$
where $\lambda$ is the signal wavelength and $\dThetaHPBW/2$ is the electrical angle where the HPBW is obtained. This result is more commonly expressed as $\dThetaHPBW/2= 1.4/N$. 
\par In App.~\ref{apdx_HPBW} we extend this known result for any $r\geq 0$. It turns out that for large $N$, the HPBW is obtained by solving the equation
\begin{equation}\label{eq_HPBW}
    % \resizebox{1\linewidth}{!}{
        \begin{split}
            \rBrace{r^{2}-4r+2}\frac{\sin{\rBrace{x}}^{2}}{x^{2}}+r\frac{\sin{\rBrace{2x}}}{x}-1=0
        \end{split}
    % }
\end{equation}
where we define $x\triangleq{} N\dTheta_{HPBW}/2$. In Fig.~\ref{fig_feedbackULA_HPBW_Nx_vs_N_variousR} we plot the numerical solution of \eqref{eq_HPBW} for various values of $r$ and $N$, showing that $x$ reaches its limit around $N=20$. Also note that for $r=0$ we obtain the known result of standard ULA with the limiting factor of $1.4$.
Having the limiting factors for various values of the gain mismatch $r$, we investigate the feedback related improvement, expressing the HPBW by
\[
\dThetaHPBW/2\approx \frac{1.4}{f(r)N}
\]
where $f(r)$ represents the array aperture improvement factor, compared to the standard ULA. 
Fig.~\ref{fig_feedbackULA_beamwidth_limit_r_dependent} shows that $f(r)$ is closely fitted with a second order polynomial
\begin{equation}
    \label{eq_Bapprox}
    f\rBrace{r}\approx\frac{1.4}{\rBrace{1-r}\rBrace{-0.4r+1.4}}.
\end{equation}
Note that for accurate gain match (i.e. $r\to1$), the right-hand-side of \eqref{eq_Bapprox} tends towards infinity, implying that the equivalent array has infinite number  of elements ($f\rBrace{r}N$), hence obtaining perfect spatial selectivity.
\begin{figure}[t]
    \begin{center}
        \begin{overpic}[width=0.65\linewidth, 
        %grid, 
        tics=10,trim=0 0 0 0]{./Media/spatial_IIR_MATLAB/arrayParameters/HPBW_vs_N_various_r.eps}
            \put (4, 75){\footnotesize{$N\dThetaHPBW/2$}}
            \put (50, 62.5) {\footnotesize{$r=0$}}
            \put (50, 54) {\footnotesize{$r=0.1$}}
            \put (50, 39.5) {\footnotesize{$r=0.3$}}
            \put (50, 28.5) {\footnotesize{$r=0.5$}}
            \put (50, 19.75) {\footnotesize{$r=0.7$}}
            \put (50, 12.5) {\footnotesize{$r=0.9$}}
            \put (50, 2) {\footnotesize{$N$}}
        \end{overpic}
    \end{center}
     \caption{Plot of $x=N\dThetaHPBW/2$ vs. $N$, for various $r$ values, obtained by numerically solving \eqref{eq_HPBW}.}
    \label{fig_feedbackULA_HPBW_Nx_vs_N_variousR}
\end{figure}
\begin{figure}[t]
    \begin{center}
        \begin{overpic}[width=0.65\linewidth, 
        % grid, 
        tics=10,trim=0 0 0 0]{./Media/HPBW_limit_vs_r.eps}
            \put (39.5, 63.5) {\scriptsize{Numerical solution of \eqref{eq_HPBW}}}
            \put (39.5, 58.25) {\scriptsize{Polynomial fitting \eqref{eq_Bapprox}}}
            \put (39.5, 52.5) {\footnotesize{$\log_{10}f\rBrace{r}$}}
            \put (85, 75) {\footnotesize{$\log_{10}f\rBrace{r}$}}
            \put (4, 75){\footnotesize{$N\dThetaHPBW/2$}}
            \put (50, 2) {\footnotesize{$r$}}
        \end{overpic}
    \end{center}
    \caption{Evaluation of $N\dThetaHPBW/2$ for $N=100$ and its approximation $1.4/f\rBrace{r}$  (marked by red diamonds). $f\rBrace{r}$ is also presented, in logarithmic scale (dotted curve).} 
    \label{fig_feedbackULA_beamwidth_limit_r_dependent}
\end{figure}
\subsection*{Sidelobes attenuation}
\ifdefined\showDev
    \fbox{
    \begin{minipage}{0.9\linewidth}
    \textbf{development specifics}\\
    Let $f\rBrace{\dTheta} \triangleq \D{\dTheta/2}{N}\exp\rBrace{-j\rBrace{(N-1)\dTheta/2}},$ such that $\Hr_{\dTheta,\dPhi=0,r}\rBrace{\omega} = \frac{f\rBrace{\dTheta}}{1-rf\rBrace{\dTheta}}.$ Then, we compute the beampattern's derivative and state that
    \begin{equation*}
    % \resizebox{0.9\linewidth}{!}{
        \begin{split}
            &\frac{\partial}{\partial\dTheta}\Hr_{\dTheta,\dPhi=0,r}\rBrace{\omega} &\\
            &=\frac{\Dp{\dTheta/2,N}{'}\rBrace{1-r\D{\dTheta/2}{N}}-\D{\dTheta/2}{N}\rBrace{-r\Dp{\dTheta/2,N}{'}}}{\rBrace{1-r\D{\dTheta/2}{N}}^{2}}
            \\
            &=\frac{\Dp{\dTheta/2,N}{'}}{\rBrace{1-r\D{\dTheta/2}{N}}^{2}}.
        \end{split}
        % }
    \end{equation*}
    It follows that the sidelobes of the feedback based beampattern are located at the same angles as the in the ULA case.
    \end{minipage}
    }
\else
\fi
By taking a derivative of $\Hr_{\dTheta,\dPhi=0,r}\rBrace{\omega}$  with respect to $\dTheta$ it can be easily verified that extrema points of the beampattern are obtained at the same points as those of the standard ULA. Specifically, the sidelobes locations are
\begin{equation}
    \label{eqn_CB_sidelobesLocations}
    \dTheta_{\text{sidelobe}} = \frac{\rBrace{2m+1}\pi}{N}\ \forall m\in\cBrace{\pm 1,\pm 2,\hdots}.
\end{equation}
Our main interest is with the first sidelobe (i.e. $m=1$), therefore we evaluate \eqref{eq_Hdphi0} at $\dTheta = 3\pi/N$, which results in
\begin{equation}
    \abs{\Hr_{{3\pi}/{N},0,r}(\omega)}^2
    =
    \frac{
    2\rBrace{1-r}^{2}
    }{
    \rBrace{N^{2}-2Nr}\rBrace{1-\cos{\rBrace{\frac{3\pi}{N}}}}+2r^{2}
    }
    \label{eq_HSidelobes}
\end{equation}
and for large $N$ values 
\begin{equation*}
    \lim_{N\rightarrow\infty}\abs{\Hr_{3\pi/N,0,r}(\omega)}=\frac{
    2\rBrace{1-r}
    }{
    3\pi
    }.
\end{equation*}
\ifdefined\showDev
    \fbox{
    \begin{minipage}{0.9\linewidth}
    \textbf{development specifics}\\
    We know that \eqref{eq_Hdphi0} can be rewritten as
    \begin{equation*}
        \begin{split}
            \Hr_{\dTheta,\dPhi=0,r} \rBrace{\omega}=
            \frac{
            \rBrace{1-\cos{\rBrace{N\dTheta}}}\rBrace{1-r}^{2}
            }{
            \begin{split}
                N^{2}&\rBrace{1-\cos{\dTheta}}+r^{2}\rBrace{1-\cos{N\dTheta}}
                \\
                &+Nr\Bigg(1+\cos{\rBrace{\rBrace{N-1}\dTheta}}
                \\
                &-\cos{\rBrace{N\dTheta}-\cos{\rBrace{\dTheta}}}\Bigg)
            \end{split}
            }
        \end{split}
    \end{equation*}
    Using the symbolic toolbox in MATLAB, setting $\dTheta=3\pi/N$, results in \eqref{eq_HSidelobes}.
    \end{minipage}
    }
\else
\fi
\par For classical ULA, the gain of the first sidelobe is known to be $\frac{2}{3\pi}$ \cite{van2004optimum}, which implies that the suggested feedback based system has a suppression improvement of $1/\rBrace{1-r}$ at the first sidelobe.

\subsection*{Array directivity}
Another common measure for the array performance is its directivity \cite{van2004optimum} ($\mathcal{D}$), defined as
\begin{equation}\label{eq_D}
    \mathcal{D}(r) = \frac{\Hr_{\dTheta=0,\dPhi=0,r}}{\frac{1}{2\pi}\int_{0}^{2\pi}\Hr_{\dTheta,\dPhi=0,r}\ d\dTheta},
\end{equation}
measures the ratio between the maximal array gain at its mainlobe, to the averaged gain over all directions. 
For uniformly weighted ULAs with no feedback, it is known \cite{van2004optimum} that $\mathcal{D}(r=0) = N$.
Plugging \eqref{eq_Hdphi0} within \eqref{eq_D} and by numerical evaluation, we fit the following model for arbitrary $0\leq r\leq 1$
\begin{equation}\label{eq_D}
    \mathcal{D}(r) = \frac{r^{2}-\rBrace{N-1}r+N}{\rBrace{1-r}^{2}}.
\end{equation}
\par It is worth mentioning that for $r=0$ the passive ULA result is obtained. Also, for $r\to1$, we use L'Hôpital's rule to state that $\lim_{r\rightarrow 1}\mathcal{D}(r)=\infty$, implying infinite directivity for the perfectly gain matched feedback based array. Finally, expressing the improvement in directivity compared to the classical ULA, we have that for large $N$,
\begin{equation}\label{eq_Dimprovement}
\frac{\mathcal{D}(r)}{\mathcal{D}(0)}\overset{N\gg1}{=}\frac{1}{1-r}.
\end{equation}
\subsection*{Summary}
To conclude this section, we present Table.~\ref{table_arrayPerformance} which exemplifies the superior performance of our suggested architecture over the standard ULA case.
\begin{table}[h!]
    \caption{Comparing performances of classical ULA and the proposed feedback based architecture, with gain mismatch $r$.}
    \centering
    \resizebox{1\linewidth}{!}{
        \begin{tabular}{||c c c c||} 
            \hline
            & ULA & \thead{FEEDBACK\\BASED} & IMPROVEMENT \\ [0.5ex] 
            \hline\hline
            HPBW & $ 1.4/N$ & $1/B(r)N$ with $B$ of \eqref{eq_Bapprox} & $1.4B\rBrace{r}$ times smaller\\ 
            \thead{FIRST\\SIDELOBE\\GAIN} & $2/3\pi$ & $2\rBrace{1-r}/3\pi$ & $1-r$ \\
            DIRECTIVITY & $N$ & $\mathcal{D}$ of \eqref{eq_D} & $1/\rBrace{1-r}$ \\
            [1ex] 
            \hline
         \end{tabular}
     }
    \label{table_arrayPerformance}
\end{table}