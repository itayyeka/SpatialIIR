\IEEEPARstart{S}{ensor} arrays enable the extraction of spatial information, sampling spatially diverse snapshots of impinging signals. Array processing, being the generalized expression for the processing of spatially diverse signals, has been thoroughly studied throughout several decades, producing many important application such as spatial filtering, transmission energy focusing, source localization, detection etc. Uniform linear array (ULA), a linearly equally spaced sensor array, has always been a point of interest, being convenient to both production and analysis \cite{VanTrees2002DetectionIV}. Even after some decades of research, $N$ element array processing algorithms (such as the very influential MUSIC \cite{Ralph1986MultipleParameter}) enabled $N-1$ degrees-of-freedom (DOF), which, from detection point of view, states that one could detect up to $N-1$ spatially separated sources in a give arena. 
This limitation triggered an extensive pursuit for increasing the DOF of a given array, combining multiple approaches. 
One approach was using different array geometries, such as minimum redundancy arrays (MRA) \cite{Moffet1968Minimum-RedundancyArrays,Pillai1985AEstimation,UnnikrishnaPillai1987StatisticalMatrix}.  
Another approach was the use of cumulants \cite{Chevalier2005OnProcessing,Mendel1999ApplicationsProcessing,Chevalier1999OnProblemb} ($4^{th}$ and higher order statistics), producing a ``virtual array`` of $\mathcal{O}(N^{2})$ elements out of an ordinary $N$ elelemnt ULA, which obviously enables a matching number of DOF. Using the same approach, the $2q$-MUSIC algorithm \cite{Chevalier2006High-resolutionAlgorithm}, enabling an $N^{2q}$ ``virtual arrays``, was developed and implemented for other signal types \cite{Liu2008ExtendedSignals}, array geometries \cite{Pal2012MultipleProcessing} and diverse antenna configurations \cite{Chevalier2007HigherAlgorithms}.
Combining the two approaches, while setting the nested arrays concept, was \cite{Pal2010NestedFreedom}.
Although achieving high increase in the DOF, these methods demand costly array fabrication and exhaustive computing to access the high order statistics which is calculated according to the empirically received signals.
Another approach, the basis for this paper, is influenced by the digital signal processing theory. 
Bearing in mind the well known analogy between ULA and finite impulse response (FIR) filter \cite{VanVeenBeamforming:Filtering}, resulting form the equivalency between equally spaced temporal/spatial samples of nerrowband-signals, and the well known frequency response superiority of the infinite impulse response (IIR), many works try to find the equivalent analogous array structure to spatial-IIR filter.
Related works propose various solutions.
\cite{Wen2013ExtendingStructure} proposes two solutions, starting from synthetic spatial delay generation for the recursive part of the IIR, which achieves the same results as the MUSIC algorithm, demands a high accuracy delay chain which can be delicately controlled. The other solution is treating the ULA as multiple sub arrays, where the combination of the independent sub-arrays can be thought to be the recursive part of the spatial filter. Those two approaches does not achieve a recursive spatial response and are sensitive to estimation of the DOA-dependent delay between signals impinging the different sensors.
Other works (\cite{Madanayake2008AFilters,Madanayake2008ABeamformer}) use the concept of $2D$ spatio-temporal plane wave representation, combined with ultra-wide-band \cite{YangLiuqingandGiannakisUltra-widebandCome} (UWB) filtering and use systolic processors for achieving the recursive part of the filter.
The goal in this paper, same as the latter approach, is to find the analogous array structure that achieves a ``spatial-IIR``, while insisting on removing the temporal processing whatsoever. For this purpose, we integrate a transmitter into the array and suppose a cooperative target which reflects its received signal back to the receiver and we prove that an $N^{th}$ spatial IIR filter is achievable without using any temporal processing.
\subsection*{Notations}
Matrices are denoted by boldfaced capital letters (e.g., \vecnot{A}). Vectors are denoted by boldfaced lowercase letters (e.g., \vecnot{a}). Superscript $^{*}$ denotes conjugation, superscript $^{H}$ denotes transposed conjugation and superscript $^{\mathcal{F}}$ stands for Fourier transform. $\theta_{g}$ is a geometric angle, measured from the array virtual line (where $\theta_{g} = \frac{\pi}{2}$ is called the ``end-fire``),while $\theta_{e}$ is an electrical signal phase. Unless said otherwise, $\theta$ is treated as $\theta_{e}$.

\subsection*{Signal model}
Consider an $N$-element ULA with inter element spacing $d$, impinged by a far-field signal, $x\left(t\right)$ ($t$ is time in seconds), traveling in an an-echoic flat-response medium, which arrives from a certain direction of a arrival (DOA), $\theta_{g}$. The signal received at the $n^{th}$ sensor can be expressed as 
\begin{equation}
x_{n,\theta}(t) = x\left(t-2\tau_{pd}-\tau_{n,\theta}\right),
\label{eqn:noFeedbackULA_singleSensor_temporal}
\end{equation}
where $r$ is the range between the array and the signal source, $c$ represent the propagation velocity of the signal in the medium, $\tau_{pd} = \frac{r}{c}$ is the propagation delay from the source to the array and $\tau_{n,\theta} = n\frac{d\cos{\theta_{g}}}{c}$ is the impinging signal's time of arrival (TOA) difference between consecutive sensors. 
\subsection*{The analogy between ULA and FIR}
\label{subsec_ULAFIR_analogy}
FIR filters are basically delay-and-sum systems, weighting different delayed instances of a given received signal $x\left(k-n\right)$ with different weights $w_{n}$ and summing them to a single output $$y\left[k\right] = \sum_{n=0}^{N-1}w_{n}x\left[k-n\right] = w\left[k\right]*x\left[k\right],$$ where $*$ is the convolution operator. 
Under the Fourier transform, the frequency domain representation of the FIR's output is $$y_{FIR}^{\mathcal{F}}\left(\omega\right) = x^{\mathcal{F}}\left(\omega\right)w_{FIR}^{\mathcal{F}}\left(\omega\right),$$ where $$ w_{FIR}^{\mathcal{F}}\left(\omega\right) = \sum_{n=0}^{N-1}w_{n}exp\left(-j{}\frac{\omega}{\omega_{s}}n\right),$$ $\omega_{s}$ is the radial sampling frequency, $\omega$ is the radial frequency domain variable and $j \triangleq \sqrt{-1}$ .
Evidently, the attenuation/enhancement of spectral components of $x$ are determined by the spectral response of $\vecnot{w}$. 
Considering (\ref{eqn:noFeedbackULA_singleSensor_temporal}), and switching to the frequency domain $$ y_{ULA}^{\mathcal{F}}\left(\omega\right) = x^{\mathcal{F}}\left(\omega\right)w_{ULA}^{\mathcal{F}}\left(\omega\right),$$ where $$ 
w_{ULA}^{\mathcal{F}}\left(\omega\right) = \sum_{n=0}^{N-1}w_{n}exp\left(-j\omega\frac{d\cos{\theta_{g}}}{c}n\right) = \vecnot{w}^{T}\Steer{\theta_{g}} $$ and $\Steer{\theta} \triangleq \left[1,  exp\left(-j\omega\frac{d\cos\left(\theta_{g}\right)}{c}\right),\hdots,exp\left(-j\omega\left(N-1\right)\frac{d\cos\left(\theta_{g}\right)}{c}\right)\right]^{T} $ is the array ``steering vector`` (which is sometimes called the ``array factor``), can be interpreted as FIR filter sampling the impinging signal at $\omega_{s} = \frac{c}{d\cos{\theta_{g}}}$.
\subsection*{ULA beam-width}
\label{section_arrayPerformance_classicULA}
In this paper, to define the "beamwidth" we use $3_{dB}$-beam, more commonly known as the Half-Power-Beam-Width (HPBW). An important known \cite{VanTrees2002DetectionIV} result, is 
$$
u_{HPBW} = 2\left(\cos\left(\theta_{g}\right)-\cos\left(\theta_{g,s}\right)\right) = \frac{\lambda}{Nd}\frac{1.4}{\pi},
$$
where $\lambda$ is the signal wavelength which links the array beamforming performance (beamwidth) to its physical aperture ($Nd$). 
Appendix \ref{appendix_theULABeamwidth} thoroughly elaborates the methodology used to evaluate the half power beamwidth (HPBW) in a passive ULA implementing the conventional beamformer ($\triangleq \frac{1}{N}\Steer{\theta_{g,s}}^{*}$) \cite{VanTrees2002DetectionIV}, where $\theta_{g,s}$ is the spatially enhanced DOA.