\IEEEPARstart{T}{he} 
general field of array processing has been thoroughly studied in many contexts throughout several decades, producing many important application such as spatial filtering, source localization, signal detection, source separation, manifold learning, feature extraction, and many more.
Spatial sensor arrays enable the extraction of spatial information such as localizing a transmitting source \cite{skolnik2008radar} \myTodo{inline}{\textbf{DONE:}\\cite some radar book}, blindly separating mixtures of impinging signals \cite{Comon1994IndependentConcept} \myTodo{inline}{\textbf{DONE:}\\here you can cite some papers which talk about BSS and ICA signal source separation. Make sure to cite something which is well known, that is, has many citations}, improving signal to noise ratio (SNR) \myTodo{inline}{\textbf{DONE:}\\cite some works/books about MVDR,MPDR processing}\cite{Frost1972AProcessing,verdu1998multiuser} and many more. 
\par Uniform linear array (ULA), a uniformly spaced structure of sensors, has always been a point of interest, due to its simplicity of analysis \cite{VanTrees2002DetectionIV}. 
The array size and the number of its elements $N$ has significant influence on the obtained array performance, such as SNR improvement, spatial separation capabilities and its number of degrees of freedom (DOF). For example, the influential MUSIC algorithm \cite{Ralph1986MultipleParameter} enables to detect up to $N-1$ directions of arrival (DOA) due to $N-1$ spatially separated sources, by projecting the possible steering vectors (SV) of the array manifold onto the noise subspace.
The limitation on the number of processed or detected sources due to the limited amount of DOF of a given array, inspired different directions of research.\myTodo{inline}{\textbf{ACCEPTED:}\\you are focusing on the DOF problem, but this is not the main focus of our paper...} 
\par One approach, involving different array geometries, examined minimum redundancy arrays \cite{Moffet1968Minimum-RedundancyArrays,Pillai1985AEstimation,UnnikrishnaPillai1987StatisticalMatrix} \myTodo{inline}{are those the works were we look at arrays with 'holes' and fill in the void due to second order statistics across the existing elements?\\\textbf{ANSWER}\\ No, those works deal with non-uniform linear arrays, reducing the element spacing redundancy, such that for $N$ element array, there will be the minimal number of pair with a specific difference. The main purpose is to increase resolution by reducing ambiguity. The extraction of the "holes" is the "virtual array" concept which is mentioned next} trying to reduce the spatial ambiguity through minimization of redundant inter-element spacing in order to increase the overall resolution. 
\par Another approach, commonly referenced as ``virtual arrays" \cite{Pal2010NestedFreedom,Chevalier2005OnProcessing,Mendel1999ApplicationsProcessing}\myTodo{inline}{are you citing known works? \\\textbf{(Yes, for example \cite{Chevalier2005OnProcessing} is cited ~200 times)}\\ I think Veydanathan \\\textbf{ADDED his work.}\\(hope that spelling correctly) did some works on this. Also Arye Yeredor from TAU} deals with the extraction of samples originated in sensors that do not really exist (i.e. relying on higher order statistics and manipulating multiple statistical cross-terms in order to estimate statistical characteristics of signals impinging in missing sensors). 
Using a similar approach, the $2q$-MUSIC algorithm \cite{Chevalier2006High-resolutionAlgorithm} \myTodo{inline}{not familiar... \\\textbf{Has around ~140 citations}\\}, enables the use of $N^{2q}$ ``virtual elements'', by calculating the $q$'th order statistics. \myTodo{inline}{\textbf{REPHRASED: and reduced}\\did not understand this last sentence. You did not explain what are nested arrays.}
\par A well known \cite{VanVeenBeamforming:Filtering} equivalence, which inspired the current paper, is the analogy between ULA spatial array processing of narrow-band signals and finite impulse response (FIR) temporal filtering. 
In the context of temporal signal processing, it is well known that for a given filter order $N$, in many cases, the infinite impulse response (IIR) filter leads to improved performance, as compared to FIR filter design of the same order. In particular, narrow transition regions, and low sidelobes can be achieved by implementing feedback based filtering. 
\par This naturally arises the question, ``what are the equivalent spatial domain processing methods which will be analogous to temporal IIR filtering? " 
\par A related work \cite{Wen2013ExtendingStructure} has also addressed this question. 
There, in the context of ULA, two approaches were considered. \myTodo{inline}{\textbf{DONE:}\\The next sentence is not clear. Please rephrase}
The first one was to estimate the time of arrival (TOA) difference between two consecutive sensors and to synthetically generate the recursive part of the IIR filter, entirely in the time-domain. 
The second approach suggested to consider overlapping subsets of one large ULA as finite approximation to an infinite array. Surely, the former approach heavily relies on the accuracy of the delay estimation and the latter approach does not achieve a recursive spatial response. In both cases, there is no true spatial feedback between the array and the source of interest.
\par Other works \cite{Madanayake2008AFilters,Madanayake2008ABeamformer} use the concept of $2D$ spatio-temporal plane wave representation (i.e. a straight line angled according to the DOA in the spatio-temporal plane), to design ultra-wide-band (UWB) filters \cite{L.Bruton1983HighlyPlanes} which both sample spatial snapshots of the signal and recursively process it in temporal domain. \myTodo{inline} {\textbf{REPHRASED}\\please explain (maybe later just to me) how UWB filtering is used}\myTodo{inline}{\textbf{Removed - not important - a way of implementation}\\what is that?}\myTodo{inline}{\textbf{DONE:}\\what are the drawbacks if these last papers?}
Here as well, the recursive part of the filter is obtained entirely in the temporal domain.  
\par The goal in this contribution, is to present a sensor array processing approach which achieves spatial-domain feedback-based processing, similarly to IIR filtering in the time domain. Opposed the previous works, we do not wish to estimate the inter-element delay and apply the recursive part in the time domain, but rather implement the feedback entirely in the spatial domain.
We focus on a localization problem, where our goal is to estimate the direction and the range of some target.
Opposed to classical array processing, we incorporate spatial feedback, show its equivalence to IIR filtering in case of a ULA, and analyze some key aspects of the proposed scheme. 

The spatial feedback between the array and the target is created by constantly re-transmitting a signal and its echoes between the array and the target. 
The initial stimulus can be generated at the target itself or at the location of the array. In the text to follow, we assume this is the latter. 

Also, similar to radar applications, the target can be passive and merely reflect the impinging signal, or to be cooperative, i.e. by receiving, enhancing the signal and re-transmitting it back to the array. In this work, we assume the former.

\par The outline of this paper is as follows. We first formulate the classic spatial beamforming setup in Sec.~\ref{sec:setup}. Then, in Sec.~\ref{sec_introduceFeedback}, we propose our novel feedback based architecture, and calculate its spatial response. In Sec.~ \ref{sec_FIM} we evaluate the Fisher Information Matrix (FIM) and discuss optimal choices of the array weights in order to  maximize the information in detecting the target's range and DOA. In Sec.~\ref{sec_Performance} we evaluate some key features of the proposed beamforming with feedback. Specifically, we compute the array beamwidth, its peak to sidelobe ratio and the array directivity, showing significant improvement compared to traditional beamforming without spatial feedback. 
In Sec.~\ref{sec_app} we simulate the proposed processing scheme, and emphasis its sensitivity to range errors. We then suggest a strategy which eliminates this sensitivity. Finally, concluding remarks are stated in Sec.~\ref{sec_conclusions}.
\myTodo{inline}{Here you should state the content of the paper. In Sec. \ref{sec:setup} we define our setup. In Sec. ... we analyze the suggested feedback based system. etc...}
\section{Classical Beamforming }\label{sec:setup}
\myTodo{inline}{\textbf{DONE:}\\ why to speak only about ULA? The work is more general and ULA is a special case. Will rephrase this whole section}Consider an $N$-element array with $n$'th sensor at position $p_n,\; n=0,\ldots,N-1$, where we define $p_{0}=0$ to be the reference point for future analysis. Let $p_t$ be the position of a target of interest. We focus on a far field localization problem, aiming for DOA and target range estimation. Without loss of generality, and for simplicity of the exposition, we reduce the discussion to a 2D problem, denoting the DOA by a single angle $\theta_g$. The range $R$ between the array and the target of interest can be computed with respect to some reference sensor, say $p_{0}$, i.e. $R\triangleq\norm{p_{t}-p_{0}}$. 
Inspired by radar based applications, we assume that the signal $x(t)$ is transmitted from the array, reflects back from the target and re-impinges the array, with total time delay of $\tau_{pd}=2R/c$ seconds, where $c$ represents the propagation velocity of the signal in the medium. Opposed to radar applications, we assume that the target of interest is static. 

Let $x_{n,\theta_g}(t)$ be the measured signal at the $n$'th sensor of the array
\begin{equation}
x_{n,\theta_g}(t) = g_{n,\theta_{g}}x\Brack{t-\tau_{pd}-\tau_{n,\theta_{g}}},
\label{eqn:noFeedbackULA_singleSensor_temporal}
\end{equation}
where $g_{n,\theta_{g}}$ represents the gain due to medium attenuation, the influence of target's radar cross section (RCS) and the sensor's gain at DOA $\theta_g$. Here, $\tau_{n,\theta_{g}}$ represents the arrival time difference of the signal between the $n$'th sensor (at $p_n$) and the reference sensors (at $p_{0}$). In the Fourier domain, we can express the array signals in a vector form 
\[
\Steer{\theta_g}x^\mathcal{F}_0(\omega),
\]
where $x^\mathcal{F}_0(\omega)$ is the Fourier transform of the signal on the first element $x_{0,\theta_g}\Brack{t}$ and the $n$'th element of the steering vector is
\begin{equation}
    \label{eq:d}
    \vd_{\theta_g}[n] = g_{n,\theta_{g}}\exp{\rBrace{-j\omega\tau_{n,\theta_g}}},\;n=0,\ldots,N-1 
\end{equation}
is the steering vector. An N element array beamformer which consists of weights $\vBeta$ will then shape the array reception pattern to produce the beamformed signal $\sum_n \beta_n x_{n,\theta}$. The pattern is then given by 
$$ \vBetaT\vd_{\theta_g}x^\mathcal{F}_0(\omega). $$ 
\par For ULA with inter-element spacing $d$, the inter element delay is
$$
\tau_{n,\theta_g}=n\frac{d\cos\Brack{\theta_{g}}}{c}.
$$
Re-writing of the array response in terms of the electric phase
\begin{equation}\label{eq:thetaULA}
\theta=\omega{d\cos\Brack{\theta_{g}}}/{c},
\end{equation}
gives rise to
\[
x^\mathcal{F}_0(\omega)\sum_{n=0}^{N-1}\beta_n
\exp\Brack{-jn\theta}.
\]
Thus, in the ULA case, setting the weights vector $\vBeta$ for obtaining a desired spatial response is mathematically equivalent to FIR filter design~\cite{VanVeenBeamforming:Filtering}.
\par In the standard scheme of processing radar signals, a waveform $x(t)$ is transmitted to, and reflected from the target of interest. Then, the reflected signal is processed by the radar reception array in order to estimate the DOA, range (and Doppler) of the target. On the other-hand, what we propose here is to continuously re-transmit the signal and its echoes back to the platform, such that a spatial feedback loop is created between the array and the target. In the context of ULA, we show this scheme to be equivalent to IIR filter design. The suggested structure is elaborated in Sec.~\ref{sec_introduceFeedback} and analyzed in subsequent sections.
