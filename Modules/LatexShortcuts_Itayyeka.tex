%general:
%Box and color definitions:
%--------------------------
\newenvironment{ColorBoxedminipage}
{\begin{minipage}} {\end{minipage}}
%{\begin{Sbox}\begin{minipage}}
%{\end{minipage}\end{Sbox}\fcolorbox{Blue}{White}{\TheSbox}}

\newcommand{\sheading}[1]{
  \large\bf
  \textcolor{NavyBlue}{\begin{Bcenter}#1\end{Bcenter}}}

\setlength{\fboxsep}{0.35cm} \setlength{\fboxrule}{4pt}

\definecolor{LightBlue}{rgb}{0.8,0.85,1}
\definecolor{LightGreen}{rgb}{0,1,.85}
\definecolor{BrickRed}{named}{BrickRed}
\definecolor{Black}{named}{Black}
\definecolor{PineGreen}{named}{PineGreen}
\definecolor{Green}{named}{Green}
\definecolor{Red}{named}{Red}
\definecolor{NavyBlue}{named}{NavyBlue}
\definecolor{WildStrawberry}{named}{WildStrawberry}
\definecolor{White}{named}{White}
\definecolor{CadetBlue}{named}{CadetBlue}
\definecolor{Blue}{named}{Blue}

%General definitions:
%-------------------
%\newcommand{\PlotPath}{e:/thesa/latex/thesa_plots/}
\newcommand{\PlotPath}{y:/thesa_plots/}
\newcommand{\inp}[2]{\left<{{#1},{#2}}\right>} %inner product
\newcommand{\etal}{{\em {et al.}}}
\newcommand{\B}[1]{\mathbf{#1}}
\newcommand{\df}{\triangleq}
\newcommand{\W}{\omega}
\newcommand{\norm}[1]{\left\Vert#1\right\Vert}
\newcommand{\abs}[1]{\left\vert#1\right\vert}
\newcommand{\RE}{\operatorname{Re}}
\newcommand{\IM}{\operatorname{Im}}
\newcommand{\sgma}[3]{\sum\limits_{{#1}={#2}}^{#3}}
\newcommand{\Brk}[1]{\left(#1\right)} %Put argument in brackets
\newcommand{\E}[1]{E\left\{#1\right\}} %Expectation
\newcommand{\Var}[1]{Var\left\{#1\right\}} %Variance
\newcommand{\Cov}[1]{Cov\left\{#1\right\}} %Covariance
\newcommand{\Brace}[1]{\left\{{#1}\right\}} %Braces
\newcommand{\Brack}[1]{\left({#1}\right)} %Brackets
\newcommand{\sBrack}[1]{\left[{#1}\right]} %square Brackets
\newcommand{\Mtr}[2] %short notation for 2x1 Matrix.
{\begin{bmatrix}
  #1 \\
  #2
\end{bmatrix}}
\newcommand{\cMtr}[2] %short notation for 2x1 Matrix with curves.
{\left(
\begin{array}{c}
    {#1} \\
    {#2} \\
\end{array}
\right)}
\newcommand{\Mtrs}[2] %short notation for 2x1 Matrix star (adjoint)
{\begin{bmatrix}
  #1 &
  #2
\end{bmatrix}}
\newcommand{\Mtrt}[3] %short notation for 3x1 Matrix.
{\begin{bmatrix}
  #1 \\
  #2 \\
  #3
\end{bmatrix}}

\newcommand{\Cases}[4]{
\left\{
\begin{tabular}{lcl}
    $#1$ & $=$ & $#2$\\
    $#3$     & $=$ & $#4$
\end{tabular}
\right. }

\newcommand{\und}{\underline} %How lazy can I get?
\newcommand{\ovr}{\overline}
\newcommand{\conj}[1]{{#1}^\ast} %Conjugation

\newcommand{\ls}[2]{\Brk{\conj{#1} {#1}}^{-1}\conj{#1}{#2}} %LS
\newcommand{\wls}[3]{\Brk{\conj{#1}{#3}{#1}}^{-1}\conj{#1}{#3}{#2}} %weighted LS

\newcommand{\iter}[1]{^{({#1})}}
\newcommand{\ut}{\und{\theta}} %underline theta
\newcommand{\us}{\und{s}} %underline s.
\newcommand{\usk}[1]{\und{s}^{(#1)}} %underline s at k-th iteration
\newcommand{\ust}[1]{\und{s}({#1})} %underline s at time t
\newcommand{\ti}[1]{\und{\theta}^{(#1)}} %theta^(i)
\newcommand{\tli}[1]{\und{\theta}{(#1)}} %theta low i: theta_i (for time indexing)
\newcommand{\htli}[1]{\hat{\und{\theta}}{(#1)}} %theta low i: theta_i (for time indexing)



%identity matrices:
\newcommand{\It}{I_2}
\newcommand{\Io}{I_1}
\newcommand{\uz}{\und{0}} %underline Zero
\newcommand{\uo}{\und{1}} %underline One


%Power spectrum shortcuts:
\newcommand{\nw}[1]{({#1},\W)}
\newcommand{\Ps}{\Phi_{ss}(\W)}
\newcommand{\Psk}[1]{\Phi_{ss}({#1},\W)}
\newcommand{\Pn}{\Phi_{nn}(\W)}
\newcommand{\uPn}{\und{\Phi}_{nn}(\W)} %underline Pnn
\newcommand{\Poo}{\Phi_{z_1z_1}(\W)} %P one one
\newcommand{\hPoo}{\hat{\Phi}_{z_1z_1}(\W)} %hat P one one
\newcommand{\uPoo}{\und{\Phi}_{z_1z_1}(\W)} %underline P one one
\newcommand{\uhPoo}{\hat{\und{\Phi}}_{z_1z_1}(\W)} %underline hat P one one
\newcommand{\Pook}[1]{\hat{\Phi}_{z_1z_1}\nw{#1}} %the argument is the frame index.
\newcommand{\Pot}{\Phi_{z_1z_2}(\W)} %P one two
\newcommand{\Pto}{\Phi_{z_2z_1}(\W)} %P two one
\newcommand{\Ptt}{\Phi_{z_2z_2}(\W)} %P two two
\newcommand{\Pom}{\Phi_{z_1z_m}(\W)} %P one m
\newcommand{\Pij}{\Phi_{z_iz_j}(\W)} %P i j
\newcommand{\Pii}{\Phi_{z_iz_i}(\W)} %P i i
\newcommand{\Pjj}{\Phi_{z_jz_j}(\W)} %P j j
\newcommand{\hPij}{\hat{\Phi}_{z_iz_j}(\W)} %hat P i j
\newcommand{\hPom}{\hat{\Phi}_{z_1z_m}(\W)} %P one m
\newcommand{\uPom}{\und{\Phi}_{z_1z_m}(\W)} %underline P one m
\newcommand{\uhPom}{\und{\hat{\Phi}}_{z_1z_m}(\W)} %underline hat P one m
\newcommand{\Pmo}{\Phi_{z_mz_1}(\W)} %P m one
\newcommand{\hPmo}{\hat{\Phi}_{z_mz_1}(\W)} % hat P m one
\newcommand{\uPmo}{\und{\Phi}_{z_mz_1}(\W)} %underline P m one
\newcommand{\uhPmo}{\und{\hat{\Phi}}_{z_mz_1}(\W)} %underline hat P m one
\newcommand{\Pmok}[1]{\hat{\Phi}_{z_mz_1}\nw{#1}} %the argument is the frame index.
\newcommand{\Pomk}[1]{\hat{\Phi}_{z_1z_m}\nw{#1}} %the argument is the frame index.
\newcommand{\Pmmk}[1]{\hat{\Phi}_{z_mz_m}\nw{#1}} %the argument is the frame index.
\newcommand{\Pmm}{\Phi_{z_mz_m}(\W)} %P m m
\newcommand{\hPmm}{\hat{\Phi}_{z_mz_m}(\W)} %hat P m m
\newcommand{\uPmm}{\und{\Phi}_{z_mz_m}(\W)} %underline P m m
\newcommand{\uhPmm}{\und{\hat{\Phi}}_{z_mz_m}(\W)} %underline hat P m m
\newcommand{\Pz} %Pz matrix
{\begin{bmatrix}
  \Poo & \Pom\\
  \Pmo & \Pmm
\end{bmatrix}}
%decorrelation matrix stuff
\newcommand{\uow}{u_1(\W)}
\newcommand{\utw}{u_2(\W)}
\newcommand{\utwn}{\conj{u}_2(\W)}


%\newcommand{\Pu}{{\Phi}_{\check{n}}(\W)}
\newcommand{\biasOne}{b_1} %will be used for the PSD and the correlation.
\newcommand{\Pbo}{{\Phi}_{\biasOne}(\W)} %P bias one
\newcommand{\Pbok}[1]{{\Phi}_{\biasOne}^{(#1)}(\W)} %P bias one, at the k-th iteration
\newcommand{\hPbo}{{\hat{\Phi}}_{\biasOne}(\W)} %hat P bias one
\newcommand{\hPbot}[1]{{\hat{\Phi}}_{\biasOne}({#1},\W)} %hat P bias one at the t-th time instance
\newcommand{\biasm}{b_m} %The seond bias. will be used for the PSD
\newcommand{\Pbm}{{\Phi}_{\biasm}(\W)} %P bias m
\newcommand{\hPbm}{\hat{\Phi}_{\biasm}(\W)} %P bias m


%Signals Fourier transforms:
\newcommand{\N}{N(\W)} %N
\newcommand{\Sig}{S(\W)} %S
\newcommand{\Zo}{Z_1(\omega)} %Z one
\newcommand{\Zt}{Z_2(\omega)} %Z two

%Transfer functions:
\newcommand{\Ao}{{A_1(\omega)}}
\newcommand{\At}{{A_m(\omega)}}
\newcommand{\Am}{{A_m(\omega)}}
\newcommand{\Bo}{{B_1(\omega)}}
\newcommand{\Bt}{{B_m(\omega)}}
\newcommand{\Bm}{{B_m(\omega)}}
\newcommand{\Aon}{\conj{A_1}(\omega)}
\newcommand{\Amn}{\conj{A_m}(\omega)}
\newcommand{\Bon}{\conj{B_1}(\omega)}
\newcommand{\Bmn}{\conj{B_m}(\omega)}
\newcommand{\Hm}{{\mathcal{H}_m(\omega)}}
\newcommand{\Hmk}[1]{{\mathcal{H}_m^{(#1)}(\omega)}}
\newcommand{\hHm}{{\hat{\mathcal{H}}_m(\W)}} %hat Hm
\newcommand{\hHmt}[1]{{\hat{\mathcal{H}}_m({#1},\W)}} %hat Hm at the t-th time instance
\newcommand{\Gm}{{\mathcal{G}_m(\W)}}
\newcommand{\Gmn}{\conj{\mathcal{G}}_m(\W)} %conj Gm
\newcommand{\Gmnk}[1]{\conj{\mathcal{G}_m^{(#1)}}(\W)} %conj Gm at the k-th iteration
\newcommand{\hGmnt}[1]{\conj{\hat{\mathcal{G}}}_m({#1},\W)} %hat conj Gm at the t-th time instance.
\newcommand{\Ar}{\frac{\Am}{\Ao}} % Am/A1 ratio
\newcommand{\Br}{\frac{\Bm}{\Bo}} % Bm/B1 ratio

%Impulse response shortcuts:
\newcommand{\hm}{h_m}
\newcommand{\hhm}{\hat{h}_m} %hat hm
\newcommand{\uhm}{\und{h}_m} %underlined hm
\newcommand{\uhhm}{\und{\hat{h}}_m} %underlined hm
\newcommand{\ao}{a_1} %a one
\newcommand{\at}{a_2} %a two
\newcommand{\am}{a_m} %a m
\newcommand{\bo}{b_1} %b one
%\newcommand{\bt}{b_2} %b two
%\newcommand{\bm}{b_m} %b m


%Special notations for transfer function ratios (related to decorrelation):
\newcommand{\ko}{k_1(\omega)} %k one
\newcommand{\ktw}{k_2(\omega)} %k two
\newcommand{\kt}{k_3(\omega)} %k three
\newcommand{\kon}{\overline{\ko}} %k one negate
\newcommand{\ktwn}{\overline{\ktw}} %k two negate
\newcommand{\ktn}{\overline{\kt}} %k three negate

%Correlation shortcuts:
\newcommand{\Roo}{R_{z_1z_1}} %R one one
\newcommand{\Rmo}{R_{z_mz_1}} %R m one
\newcommand{\Rhoo}{\hat{R}_{z_1z_1}} %R hat one one
\newcommand{\Rhook}[1]{\Rhoo^{(#1)}} %R hat one one at k'th (n'th) frame
\newcommand{\uRhoo}{{\underline \Rhoo}} %underlined R hat one one
\newcommand{\uRhook}[1]{\underline {\Rhoo^{(#1)}}} %underlined R hat one one at k'th (n'th) frame
%\newcommand{\Rto}{R_{z_2z_1}} %R two one
\newcommand{\Rhmo}{\hat{R}_{z_mz_1}} %R hat m one
\newcommand{\Rhmok}[1]{\Rhmo^{(#1)}} %R hat m one at k'th frame
\newcommand{\uRhmo}{{\und{\hat{R}}_{z_mz_1}}} %underlined R hat m one
\newcommand{\uRhmok}[1]{\und{\hat{R}}_{z_mz_1}^{(#1)}} %underlined R hat m one at k'th frame
\newcommand{\Rhu}{\hat{R}_{\check{n}}} %the noise bias vector
\newcommand{\Too}{\hat{\B{T}}_{z_1z_1}} %T one one (Toeplitz matrix for z1 z1)
\newcommand{\Took}[1]{\Too^{(#1)}} %T one one (Toeplitz matrix for z1 z1) at k'th frame
\newcommand{\Rbo}{R_{\biasOne}} %R bias one
\newcommand{\uRbo}{\und{R}_{\biasOne}} %underlined R noise one
\newcommand{\uhRbo}{\hat{\und{R}}_{\biasOne}} %underlined hat R noise one

\newenvironment{alg}[5]
{
\begin{figure}[htbp]
\begin{center}
\fbox{
  \begin{ColorBoxedminipage}{13cm}
%    \leftline{\color{Black}\bf {#1}}
    {#4}
   \end{ColorBoxedminipage}
   }
\end{center}
  \bcaptionff{#1}{#2}{}{#3}
  \label{#5}
\end{figure}
}{}
