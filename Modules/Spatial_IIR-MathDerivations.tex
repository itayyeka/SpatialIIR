\ifdefined\DEFIncludeAttenuation
    Time domain analysis of the proposed feedback based architecture, considering both propagation delay and attenuation, results in 
    \begin{equation}
        \label{eqn:SingleSensorTemporalEquality}
        x_{n,\theta}(t) = g^{2}\left(x\left(t-2\tau_{pd}-n\tau_{\theta}\right)+\sum_{m=0}^{N-1}{\beta_{m}x_{m,\theta}(t-2\tau_{pd}-n\tau_{\theta})}\right)
    \end{equation}
    where $g$ is the single-direction-range-dependent channel attenuation and $\tau_{\theta} \triangleq \frac{d\cos\left(\theta\right)}{c}$.
    Converting (\ref{eqn:SingleSensorTemporalEquality}) to frequency domain, one gets
    \begin{equation}
        \begin{split}
            x_{n,\theta}^{\mathcal{F}}(\omega) =
            g^{2}\Bigg( & x^{\mathcal{F}}(\omega)
            exp\left(-j\omega(2\tau_{pd}+n\tau_{\theta})\right)
            \\&+\sum_{m=0}^{N-1}
            {
            \beta_{m}x_{m}^{\mathcal{F}}(\omega)
            exp\left(-j\omega\left(2\tau_{pd}+n\tau_{\theta}\right)\right)
            }\Bigg) 
        \end{split}
    \end{equation} 
    which, when generalized to the entire array transforms to its vector form,
    \begin{equation}
        \resizebox{\linewidth}{!}{
            $
            \vecnot{x}^{\mathcal{F}}_{\theta}(\omega) = 
            g^{2}\left(
            x^{\mathcal{F}}(\omega)\vecnot{d}_{\theta}exp\left(-2j\omega\tau_{pd}\right)
            +
            \vecnot{d}_{\theta}\vecnot{\beta}^{T}exp\left(-j2\omega\tau_{pd}\right)\vecnot{x}_{\theta}^{\mathcal{F}}\left(\omega\right)\right).
            $
        }
    \end{equation}
    It follows that
    $$
    \ifdefined\DEFIncludeAttenuation
    y^{\mathcal{F}}_{\theta}(\omega) 
    = 
    \vecnot{\alpha}^{T}
    \left(
    I
    -g^{2}\vecnot{d}_{\theta}
    \vecnot{\beta}^{T}
    e^{-j\omega(\tau_{pd}+\tau_{tx})}
    \right)
    ^{-1}
    g^{2}
    \vecnot{d}_{\theta}
    x^{\mathcal{F}}(\omega)
    e^{-j\omega\tau_{pd}}
\else
    y^{\mathcal{F}}_{\theta}(\omega) 
    = 
    \vecnot{\alpha}^{T}
    \left(
    I
    -\vecnot{d}_{\theta}
    \vecnot{\beta}^{T}
    e^{-j\omega(\tau_{pd}+\tau_{tx})}
    \right)
    ^{-1}
    \vecnot{d}_{\theta}
    x^{\mathcal{F}}(\omega)
    e^{-j\omega\tau_{pd}}
\fi
    $$
    and with the use of the Woodbury matrix identity \cite{woodbury1950inverting} and denoting $\tau\triangleq2\omega\tau_{pd}$ , finally yields
    \begin{equation}
    \label{eqn:GeneralFeedbackTransferFunction}
    \ifdefined\DEFIncludeAttenuation
    h\rBrace{\theta,\omega} 
    \triangleq
    \frac{y_{\theta}^{\mathcal{F}}(\omega)}{x^{\mathcal{F}}(\omega)} 
    =
    \frac{    
    g\aTd{}exp\rBrace{-j\phi}
    }{
    1 - g\bTd{}exp\rBrace{-j\phi}
    }
\else
    y_{\theta}^{\mathcal{F}}(\omega) 
    =
    \frac
    {
    \vecnot{\alpha}^{T}
    \vecnot{d}_{\theta}
    exp\left(-j\tau\right)
    }
    {
    1
    -
    \vecnot{\beta}^{T}\vecnot{d}_{\theta}
    exp\left(-j\tau\right)
    }
    x^{\mathcal{F}}(\omega)
\fi,
    \end{equation}
    which, under the same reasoning of \cite{VanVeenBeamforming:Filtering}, is the desired spatial domain configurable (\vecnot{\alpha},\vecnot{\beta}) IIR response, allowing high gain and sharp (i.e. of spatially narrow) beampattern.
    Plugging in the spherical EM wave decay (i.e. $g = \frac{1}{4\pi{}R}$ where $R\triangleq{}c\tau_{pd}$ is the range between the array and the reflector), results in
    \begin{equation}
    \label{eqn:GeneralFeedbackTransferFunction_sphericalDecay}
    y_{\theta}^{\mathcal{F}}(\omega) 
    =
    \frac{    
    \vecnot{\alpha}^{T}\vecnot{d}_{\theta}exp\left(-j\tau\right)
    }{
    4c^{2}\tau^{2} - \vecnot{\beta}^{T}\vecnot{d}_{\theta}exp\left(-j\tau\right)
    }
    x^{\mathcal{F}}(\omega)
    \end{equation}
    One very important observation is the effect of $\tau$, which unlike in the passive ULA case, significantly effects the array response gain (and not only its phase).
    From spatial filtering point of view, we interpret it as a very powerful range-DOA integrated beampatterning, adding another DOF (range) to our spatially selective system.
    This observation will be thoroughly investigated throughout this paper.
\else
    %%%%% - No attenuation
    Time domain analysis of the proposed feedback based architecture results in 
    \begin{equation}
        \label{eqn:SingleSensorTemporalEquality}
        x_{n,\theta}(t) = x\left(t-2\tau_{pd}-n\tau_{\theta}\right)+\sum_{m=0}^{N-1}{\beta_{m}x_{m,\theta}(t-2\tau_{pd}-n\tau_{\theta})}
    \end{equation}
    where $\tau_{\theta} \triangleq \frac{d\cos\left(\theta\right)}{c}$.
    Converting (\ref{eqn:SingleSensorTemporalEquality}) to frequency domain, one gets
    \begin{equation}
    \resizebox{\linewidth}{!}{%
            $
            x_{n,\theta}^{\mathcal{F}}(\omega) = 
            x^{\mathcal{F}}(\omega)
            exp\left(-j\omega(2\tau_{pd}+n\tau_{\theta})\right)
            +\sum_{m=0}^{N-1}
            {
            \beta_{m}x_{m}^{\mathcal{F}}(\omega)
            exp\left(-j\omega\left(2\tau_{pd}+n\tau_{\theta}\right)\right)
            } 
            $}
    \end{equation} 
    which, when generalized to the entire array transforms to its vector form,
    \begin{equation}
    \vecnot{x}^{\mathcal{F}}_{\theta}(\omega) = 
    x^{\mathcal{F}}(\omega)\vecnot{d}_{\theta}exp\left(-2j\omega\tau_{pd}\right)
    +
    \vecnot{d}_{\theta}\vecnot{\beta}^{T}exp\left(-j2\omega\tau_{pd}\right)\vecnot{x}_{\theta}^{\mathcal{F}}\left(\omega\right).
    \end{equation}
    It follows that
    $$
    \ifdefined\DEFIncludeAttenuation
    y^{\mathcal{F}}_{\theta}(\omega) 
    = 
    \vecnot{\alpha}^{T}
    \left(
    I
    -g^{2}\vecnot{d}_{\theta}
    \vecnot{\beta}^{T}
    e^{-j\omega(\tau_{pd}+\tau_{tx})}
    \right)
    ^{-1}
    g^{2}
    \vecnot{d}_{\theta}
    x^{\mathcal{F}}(\omega)
    e^{-j\omega\tau_{pd}}
\else
    y^{\mathcal{F}}_{\theta}(\omega) 
    = 
    \vecnot{\alpha}^{T}
    \left(
    I
    -\vecnot{d}_{\theta}
    \vecnot{\beta}^{T}
    e^{-j\omega(\tau_{pd}+\tau_{tx})}
    \right)
    ^{-1}
    \vecnot{d}_{\theta}
    x^{\mathcal{F}}(\omega)
    e^{-j\omega\tau_{pd}}
\fi
    $$
    , which with the use of the Woodbury matrix identity \cite{woodbury1950inverting} and denoting $\tau\triangleq2\omega\tau_{pd}$ , finally yields
    \begin{equation}
    \label{eqn:GeneralFeedbackTransferFunction}
    \ifdefined\DEFIncludeAttenuation
    h\rBrace{\theta,\omega} 
    \triangleq
    \frac{y_{\theta}^{\mathcal{F}}(\omega)}{x^{\mathcal{F}}(\omega)} 
    =
    \frac{    
    g\aTd{}exp\rBrace{-j\phi}
    }{
    1 - g\bTd{}exp\rBrace{-j\phi}
    }
\else
    y_{\theta}^{\mathcal{F}}(\omega) 
    =
    \frac
    {
    \vecnot{\alpha}^{T}
    \vecnot{d}_{\theta}
    exp\left(-j\tau\right)
    }
    {
    1
    -
    \vecnot{\beta}^{T}\vecnot{d}_{\theta}
    exp\left(-j\tau\right)
    }
    x^{\mathcal{F}}(\omega)
\fi,
    \end{equation}
    which, under the same reasoning of \ref{subsec_ULAFIR_analogy}, is the desired spatial domain configurable (\vecnot{\alpha},\vecnot{\beta}) IIR response, allowing high gain and sharp (i.e. of spatially narrow) beampatterns.
    One very important observation is the effect of $\tau$, which unlike in the passive ULA case, significantly effects the array response gain (and not only its phase).
    From spatial filtering point of view, we interpret it as a very powerful range-DOA integrated beampatterning, adding another DOF (range) to our spatially selective system.
    This observation will be thoroughly investigated throughout this paper.
\fi