\ifdefined\DEFIncludeAttenuation
    Time domain analysis of the proposed feedback based architecture, considering both propagation delay and attenuation, results in 
    \begin{equation}
        \label{eqn:SingleSensorTemporalEquality}
        % \resizebox{.91\linewidth}{!}{
            \begin{split}
                x_{n,\theta_g}(t) = g_{n,\theta_g}\Bigg{(}&x\rBrace{t-\tau_{pd}-\tau_{n,\theta_g}}\\
                &+\sum_{m=0}^{N-1}{\alpha_{m}x_{m,\theta_g}\rBrace{t-\tau_{pd}-\tau_{n,\theta_g}}}\Bigg{)},
            \end{split}
        % }
    \end{equation}
    where the first term represents the contribution of the transmitted waveform $x(t)$ as received by the $n$'th array element and the second term represents the feedback contribution of the re-transmitted array signals to this same element.
    
    To simplify the exposition, from now on, we omit the $\theta_{g}$ dependency where possible. For example, we denote $g_n$ and $\vd$ instead of $g_{n,\theta_g}$ and $\vd_{\theta_{g}}$ respectively. 
    Taking the Fourier transform of (\ref{eqn:SingleSensorTemporalEquality})
    \begin{equation}
        \label{eqn_singleSensorFourier}
        % \resizebox{.91\linewidth}{!}{
            \begin{split}
                x_{n}^{\mathcal{F}}(\omega) =
                g_n\Bigg( & x^{\mathcal{F}}(\omega)
                \exp\rBrace{-j\omega\rBrace{\tau_{pd}+\tau_{n}}}
                \\&+\sum_{m=0}^{N-1}
                {
                \alpha_{m}x_{m}^{\mathcal{F}}(\omega)
                \exp\rBrace{-j\omega\rBrace{\tau_{pd}+\tau_{n}}}
                }\Bigg). 
            \end{split}
        % }
    \end{equation}
    Let $\vx{\rBrace{t}}\triangleq\vBrace{x_{0}\rBrace{t}\hdots{}x_{N-1}\rBrace{t}}^{T}$ be the vector of received array signals, and  $\vx^{\mathcal{F}}\triangleq\vBrace{x_{0}^{\mathcal{F}},\hdots,x_{N-1}^{\mathcal{F}}}^{T}$ its Fourier transform. Rewriting (\ref{eqn_singleSensorFourier}) in  vector form
    $$
    \vx^{\mathcal{F}} = e^{-j\omega\tau_{pd}} \rBrace{x^{\mathcal{F}}+\vAlphaT \vx^{\mathcal{F}}}\vd
    $$
    which simplifies to
    $$
    \vx^{\mathcal{F}} =\rBrace{I-\vd\vAlphaT{}e^{-j\omega\tau_{pd}}}^{-1}e^{-j\omega\tau_{pd}}x^{\mathcal{F}} \vd .
    $$
    Using Woodbury matrix identity \cite{woodbury1950inverting} and denoting
    \[
    \phi\triangleq\omega\tau_{pd}
    \]
    as the electrical phase due to the round-trip signal propagation results in
    $$
    \vx^{\mathcal{F}}
    =
    \frac{    
    \exp{\rBrace{-j\phi}}x^{\mathcal{F}}
    }{
    1 - \aTd{}\exp{\rBrace{-j\phi}}
    }\vd.
    $$
    Considering the noiseless case (i.e. $n\rBrace{t}=0, y\rBrace{t}=\vBetaT{}\vx\rBrace{t}$),
    we express the general spatial response of the suggested feedback-based array, 
    \begin{equation}
    \label{eqn:GeneralFeedbackTransferFunction}
    H_{\vBeta,\vAlpha}\rBrace{\omega} 
    \triangleq
    \frac{y^{\mathcal{F}}\rBrace{\omega}}{x^{\mathcal{F}}(\omega)} 
    =
    \frac{    
    \bTd{}\exp\rBrace{-j\phi}
    }{
    1 - \aTd{}\exp\rBrace{-j\phi}
    }.
    \end{equation}
    \myTodo{inline}{\textbf{DONE : did some re-writing}\\think that we need to rewrite this section, but meanwhile considering using a notation which resembles angle. not delay. For example $\phi=\phi(\omega \tau_{pd})=2\omega\tau_{pd}$}
    \myTodo{inline}{\textbf{DONE : I believe that citing \cite{VanVeenBeamforming:Filtering} and the end of the intro provide enough intuition. I can add more if you think not.}\\think that only if you assume that the elements are the same (but not necessarily isotropic) you get a similar expression. At this point you can talk about the special case of ULA, and describe the $\aTd,\bTd$ inner-products as filtering. }
    The spatial domain response $H\rBrace{\omega}$ depends on the DOA $\theta_g$ via the steering vector $\Steer$ of \eqref{eq:d}. It is also configurable by proper selection of the dual beamformer weights $\vBeta$ and $\vAlpha$. While the choice of $\vBeta$ defines the classical beamformer, which is equivalent to FIR filtering for ULA, the choice of $\vAlpha$ defines the recursive response, analogous to IIR filter design. As will be shown, by proper selection of the weights we can obtain high directivity and narrow beam-width. Compared to standard beamfoermers, which do not incorporate a spatial feedback loop, we will show that this scheme is equivalent to a classical beamformer of a much larger aperture.
    \myTodo{inline}{\textbf{DONE: removed.}\\For the next text, I don't think that this is wise to plug in specific propagation models. We need to say (maybe after the Cramer-Rao section) that we are setting weights to mitigate the loss $g$ and the delay manifested by the phase $\phi$. As these values are never known exactly, we will need to define the error terms $g_e$ and $\phi_e$, and express everything using this notations. }
    % Plugging in the spherical EM wave decay (i.e. $g = \frac{1}{4\pi{}R}$ where $R\triangleq{}c\tau_{pd}$ is the range between the array and the reflector), results in
    % \begin{equation}
    % \label{eqn:GeneralFeedbackTransferFunction_sphericalDecay}
    % y_{\theta}^{\mathcal{F}}(\omega) 
    % =
    % \frac{    
    % \vBetaT{}\vdexp\rBrace{-j\tau}
    % }{
    % 4c^{2}\tau^{2} - \vAlphaT{}\vdexp\rBrace{-j\tau}
    % }
    % x^{\mathcal{F}}(\omega)
    % \end{equation}
    % One very important observation is the effect of $\tau$, which unlike in the passive ULA case, significantly effects the array response gain (and not only its phase).
    % From spatial filtering point of view, we interpret it as a very powerful range-DOA integrated beampatterning, adding another DOF (range) to our spatially selective system.
    % This observation will be thoroughly investigated throughout this paper.
\else
    %%%%% - No attenuation
    Time domain analysis of the proposed feedback based architecture results in 
    \begin{equation}
        \label{eqn:SingleSensorTemporalEquality}
        x_{n,\theta}(t) = x\rBrace{t-2\tau_{pd}-\tau_{n,\theta}}+\sum_{m=0}^{N-1}{\alpha_{m}x_{m,\theta}(t-2\tau_{pd}-\tau_{n,\theta}}
    \end{equation}
    where $\tau_{\theta} \triangleq \frac{d\cos\rBrace{\theta}}{c}$.
    Converting (\ref{eqn:SingleSensorTemporalEquality}) to frequency domain, one gets
    \begin{equation}
    \resizebox{\linewidth}{!}{%
            $
            x_{n,\theta}^{\mathcal{F}}(\omega) = 
            x^{\mathcal{F}}(\omega)
            exp\rBrace{-j\omega(2\tau_{pd}+\tau_{n,\theta}}
            +\sum_{m=0}^{N-1}
            {
            \alpha_{m}x_{m}^{\mathcal{F}}(\omega)
            exp\rBrace{-j\omega\rBrace{2\tau_{pd}+\tau_{n,\theta}}}
            } 
            $}
    \end{equation} 
    which, when generalized to the entire array transforms to its vector form,
    \begin{equation}
    \vx^{\mathcal{F}}_{\theta}(\omega) = 
    x^{\mathcal{F}}(\omega)\vdexp\rBrace{-2j\omega\tau_{pd}}
    +
    \vd\vAlphaT{}exp\rBrace{-j2\omega\tau_{pd}}\vx_{\theta}^{\mathcal{F}}\rBrace{\omega}.
    \end{equation}
    It follows that
    $$
    \ifdefined\DEFIncludeAttenuation
    y^{\mathcal{F}}_{\theta}(\omega) 
    = 
    \vecnot{\alpha}^{T}
    \left(
    I
    -g^{2}\vecnot{d}_{\theta}
    \vecnot{\beta}^{T}
    e^{-j\omega(\tau_{pd}+\tau_{tx})}
    \right)
    ^{-1}
    g^{2}
    \vecnot{d}_{\theta}
    x^{\mathcal{F}}(\omega)
    e^{-j\omega\tau_{pd}}
\else
    y^{\mathcal{F}}_{\theta}(\omega) 
    = 
    \vecnot{\alpha}^{T}
    \left(
    I
    -\vecnot{d}_{\theta}
    \vecnot{\beta}^{T}
    e^{-j\omega(\tau_{pd}+\tau_{tx})}
    \right)
    ^{-1}
    \vecnot{d}_{\theta}
    x^{\mathcal{F}}(\omega)
    e^{-j\omega\tau_{pd}}
\fi
    $$
    , which with the use of the Woodbury matrix identity \cite{woodbury1950inverting} and denoting $\tau\triangleq2\omega\tau_{pd}$ , finally yields
    \begin{equation}
        \label{eqn:GeneralFeedbackTransferFunction}
        h\rBrace{\theta,\omega} 
        \triangleq
        \frac{y_{\theta}^{\mathcal{F}}(\omega)}{x^{\mathcal{F}}(\omega)} 
        =
        \frac{    
        g\aTd{}exp\rBrace{-j\phi}
        }{
        1 - g\bTd{}exp\rBrace{-j\phi}
        },
    \end{equation}
    which, under the same reasoning of \ref{subsec_ULAFIR_analogy}, is the desired spatial domain configurable (\vBeta,\vAlpha) IIR response, allowing high gain and sharp (i.e. of spatially narrow) beampatterns.
    One very important observation is the effect of $\tau$, which unlike in the passive ULA case, significantly effects the array response gain (and not only its phase).
    From spatial filtering point of view, we interpret it as a very powerful range-DOA integrated beampatterning, adding another DOF (range) to our spatially selective system.
    This observation will be thoroughly investigated throughout this paper.
\fi
\par One may observe that opposed to traditional beamformers, the beampattern, $H_{\vBeta,\vAlpha}\rBrace{\omega},$ is not only influenced by the impinging signal DOA. It is also range selective due to the dependency on the phase parameter $\phi$, thus possibly enhancing radial rings around the array. As exemplified in Fig.~\ref{fig_rangeAzimuthSelectivity}, the combination of both angular and range selective pattern enables the designer to enhance signals arriving from specific locations rather than only specific directions.
\begin{figure}[t!]
    \begin{center}
        \begin{overpic}[width=0.65\linewidth, 
        % grid, 
        tics=10,trim=0 0 0 0]{./Media/azimuthRangSelectivity.png}
            \put (20, 23){\rotatebox{0}{\footnotesize{Angular response}}}
            \put (30.5, 47){\rotatebox{0}{\footnotesize{Enhanced radial slice}}}
        \end{overpic}
    \end{center}
     \caption{Combining both radial and angular selectivity enables the designer to spatially process specific areas.}
    \label{fig_rangeAzimuthSelectivity}
\end{figure}