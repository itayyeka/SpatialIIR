Time domain analysis of the proposed feedback based architecture, considering both propagation delay and attenuation, gives rise to
\begin{equation}
    \label{eqn:SingleSensorTemporalEquality}
    % \resizebox{.91\linewidth}{!}{
        \begin{split}
            x_{n}(t) = g\rBrace{s\rBrace{t-\tau_{pd}-\tau_{n}}
            +\sum_{m=0}^{N-1}{\alpha^{*}_{m}x_{m}\rBrace{t-\tau_{pd}-\tau_{n}}}},
        \end{split}
    % }
\end{equation}
where the first term on the right-hand-side (RHS) represents the contribution of the transmitted waveform $s(t)$ to the $n$'th array element and the second term represents the feedback contribution of the re-transmitted array signal to this same element.
Expressing the Fourier transform of \eqref{eqn:SingleSensorTemporalEquality},
\begin{equation}
    \label{eqn_singleSensorFourier}
    % \resizebox{.91\linewidth}{!}{
        \begin{split}
            X_{n} =
            g\Bigg( & \F{s}
            \exp\rBrace{-j\omega\rBrace{\tau_{pd}+\tau_{n}}}
            \\&+\sum_{m=0}^{N-1}
            {
            \alpha^{*}_{m}\omegaB\F{x}_{m}
            \exp\rBrace{-j\omega\rBrace{\tau_{pd}+\tau_{n}}}
            }\Bigg),
        \end{split}
    % }
\end{equation}
and its vector from,
$$
\F{\vx} = g\rBrace{\F{s}+\vAlphaH \F{\vx}}\vd\exp{\rBrace{-j\omega\tau_{pd}}},
$$
we find that it can be simplified to
$$
\F{\vx} =g\rBrace{I-g\vd\vAlphaH{}e^{-j\omega\tau_{pd}}}^{-1}\vd\F{s}\exp{\rBrace{-j\omega\tau_{pd}}}.
$$
Then, denoting
\[
\phi\triangleq\omega\tau_{pd}
\]
as the round-trip signal propagation related electrical phase and using the Woodbury matrix identity \cite{woodbury1950inverting}, we find that
$$
\F{\vx}
=
\frac{    
g\vd\exp{\rBrace{-j\phi}}
}{
1 - g\aHd{}\exp{\rBrace{-j\phi}}
}\F{s}.
$$
Let $z=\vBetaH{}\vecnot{x}+\text{n}$ be the beamformer's output (see Fig.~\ref{fig:Proposed_spatialIIR_ARCH}), with Fourier transform $Z$. Considering the noiseless case $\rBrace{\text{i.e., n}=0}$, the frequency response of the FB is 
\begin{equation}
\label{eqn:GeneralFeedbackTransferFunction}
\Hba
\triangleq
\frac{\F{z}}{\F{s}} 
=
\frac{    
g\bHd{}\exp\rBrace{-j\phi}
}{
1 - g\aHd{}\exp\rBrace{-j\phi}
}.
\end{equation}
\par Note that this architecture achieves a controllable (via setting of $\vBeta$ and $\vAlpha$) and recursive (non-trivial denominator) spatial response.
As will be shown, high directivity and narrow beam-width are obtainable by proper selection of the weights. Comparing to traditional beamformers (i.e., without feedback), the performance improvement will be expressed in terms of increased aperture, narrower beamwidth and improved sidelobes attenuation.
One may observe that opposed to traditional beamformers, the array response, $\Hba,$ is not only influenced by the impinging signal DOA, for it is also range selective due to its $\phi$ dependency.
As exemplified in Fig.~\ref{fig_rangeAzimuthSelectivity}, the combination of both angular and range selectivity enables the designer to enhance signals arriving from specific locations (greyed area) rather than only specific directions.
\begin{figure}[t!]
    \begin{center}
        \begin{overpic}[width=0.65\linewidth, 
        % grid, 
        tics=10,trim=0 0 0 0]{./Media/azimuthRangSelectivity.png}
            \put (20, 23){\rotatebox{0}{\footnotesize{Angular response}}}
            \put (30.5, 47){\rotatebox{0}{\footnotesize{Enhanced radial slice}}}
        \end{overpic}
    \end{center}
    \caption{
    % A visualization of the spatial area selectivity concept.
    Combining both radial selectivity and DOA-based selectivity allows to localize the transmitter.
    }
    \label{fig_rangeAzimuthSelectivity}
\end{figure}