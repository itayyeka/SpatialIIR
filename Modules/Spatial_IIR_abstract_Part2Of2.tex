A natural question to consider is ``what is the array structure in spatial domain which is analogous to IIR filtering in the time domain?''.
The motivation for finding an IIR equivalent in the spatial domain is obvious:
\begin{itemize}
\item
{
Increased degrees of freedom over conventional array processing to control the array response.
}
\item
{
Substantially lower number of taps (smaller number of array elements and reduced array aperture) is required for obtaining a desired response in comparison to an FIR based design. This, naturally, results in cost-efficient array.
}
\end{itemize}
In his PhD thesis \cite{wen2013array}, Wen presented the same question. In his work two approaches were suggested for achieving ``spatial-IIR'' filtering.
First was to use shifted-sub-arrays, in order to approximate the auto-regressive part of the filter.
In practice, this approximation is equivalent to a truncated response of the IIR filter thus obtaining only an FIR implementation.
The second approach in \cite{wen2013array}, was to estimate the DOA of the impinging signal and approximate the auto-regressive part of the filter by temporal processing.
The latter approach is speculated to be very sensitive to DOA estimation errors; and it is not clear how one may extend this approach to multiple speakers scenario.
Additional works \cite{Madanayake2008ABeamformer,Madanayake2009SystolicWDFs,Madanayake2008AFilters,Bruton2003Three-dimensionalBanks,Ward1986ABeamforming,Joshi2012SynthesisApplications} consider a different approach of 2-D spatio-temporal filtering. In particular, the wavefront is viewed as a two dimensional signal, and the processing is done by performing IIR filtering in the time domain, but only FIR filtering (using a finite number of sensors) is performed in the spatial domain.
As can be seen in \cite{Bruton2003Three-dimensionalBanks}, the obtained 2-D filter is not ideal, causing imperfections in the overall spatial response.
\\
In this work we propose spatial-IIR filtering. As time-domain filter design relies on feeding back the filter's input with a combination of its taps, our approach relies on spatial feedback of received array signals back to the transmitter.
In the context of acoustic signal processing, a cooperative speaker will hold an electronic or acoustic transponder. 
Thus, the received signals at a microphone array will consist of the direct speech signal of the speaker, and a feedback signal which was recorded by the array at a previous time epoch (see Fig. \ref{fig:SignalModel}). 
Naturally, this suggested approach requires a cooperative speaker/source but, as will be explained, spatial-only-IIR filtering will be achieved by this scheme.
Comparing to \cite{wen2013array}, our method solves the main issues of both suggested approaches, achieving spatial IIR processing of multiple speaker scenarios.