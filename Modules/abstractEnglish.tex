State-of-the-art array processing methods, ranging from high-order statistics to adaptive configuration, require costly computing efforts in pursuit for spatial performance improvement.
A feedback based approach is introduced in the context of localization, featuring low complexity and high spatial performance in the excess of integrating a transmitter to the array.  
In the proposed scheme, a signal is continuously re-transmitted between the array and the target of interest.
Considering ideal scenarios, the feedback beamformer virtually achieves an infinite aperture, increasing the available spatial information about the target and significantly improves the array's spatial performance.
Using a traditional beamforming performance analysis, the beamwidth, peak to side-lobe ratio, array directivity and white noise sensitivity are evaluated for the feedback based array.
A significant improvement in all aspects is shown, while thoroughly discussing the conditions for enhanced performance.
As a practical and robust implementation of the feedback-based localization concept, an application of low estimation errors sensitivity, 
is presented and analyzed.