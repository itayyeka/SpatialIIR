A commonly used parameter for quantifying the narrowness of the array main lobe is the Half-Power-Beam-Width (HPBW), marking the DOA where the beampattern's energy reduces to half of its maximal value (i.e. $\Hr_{\dTheta_{\text{HPBW}}}=\frac{1}{2}$). 
\par For the passive ULA case, it is known \cite{VanTrees2002DetectionIV} that 
$$
u_{\text{HPBW}} \triangleq \frac{\lambda\dTheta_{\text{HPBW}}}{2\pi{}d} = \frac{\lambda}{\pi{}Nd}1.4,
$$
where $\lambda$ is the signal wavelength, linking the HPBW to the array physical aperture $Nd$.
\myTodo{inline}{\textbf{DONE:}\\maybe you can also express this last equation in terms of the electric phase $\theta$, or even better, the error $\Delta\theta$ as well. Then the connection to (12) will be clearer I hope.}
\myTodo{inline}{got up till here with the check. Will continue when I can. Best, Tsvika.}
\par Bearing in mind the tendency of $N\dTheta_{\text{HPBW}}$ towards a finite limit in passive ULA case, we simulate $\Hr_{\dTheta}$ for various $\rBrace{N,r}$ values in Fig.~\ref{fig_feedbackULA_HPBW_Nx_vs_N_variousR}. Investigating  the simulation results lead to the following observations.
\begin{itemize}
    \item $N\dTheta_{\text{HPBW}}$ tendency towards a finite limit is kept, though the limit value changes for different $r$ values.
    \item Plotting the $\underset{N\to\infty}{lim}N\dTheta_{\text{HPBW}}$ for various $r$ values, and fitting it with $2^{nd}$ order polynomial curve, results in $\underset{N\to\infty}{lim}N\dTheta_{\text{HPBW}} \approx \rBrace{1-r}\rBrace{-0.4r+1.4}$ which degenerates to the passive ULA result when cancelling the feedback (i.e. setting $r=0$).
\end{itemize}
\begin{figure}[t]
    \begin{center}
        \begin{overpic}[width=0.65\linewidth, 
        %grid, 
        tics=10,trim=0 0 0 0]{./Media/spatial_IIR_MATLAB/arrayParameters/HPBW_vs_N_various_r.eps}
            \put (2, 72){\footnotesize{$N\dTheta_{\text{HPBW}}$}}
            \put (50, 62.5) {\footnotesize{$r=0$}}
            \put (50, 54) {\footnotesize{$r=0.1$}}
            \put (50, 39.5) {\footnotesize{$r=0.3$}}
            \put (50, 28.5) {\footnotesize{$r=0.5$}}
            \put (50, 19.75) {\footnotesize{$r=0.7$}}
            \put (50, 12.5) {\footnotesize{$r=0.9$}}
            \put (50, 2) {\footnotesize{$N$}}
        \end{overpic}
    \end{center}
     \caption{Plot of $N\dTheta_{\text{HPBW}}$ vs. N for various $r$ values. In each simulation, $r$ is constant and $\dTheta_{\text{HPBW}}$ is calculated for each $N$ separately and the stored value is $N\dTheta_{\text{HPBW}}$.}
    \label{fig_feedbackULA_HPBW_Nx_vs_N_variousR}
\end{figure}

\begin{figure}[t]
    \begin{center}
        \begin{overpic}[width=0.65\linewidth, 
        %grid, 
        tics=10,trim=0 0 0 0]{./Media/spatial_IIR_MATLAB/arrayParameters/HPBW_limit_vs_r.eps}
            \put (40, 61) {\footnotesize{Simulation results}}
            \put (40, 56) {\footnotesize{$\rBrace{1-r}\rBrace{-0.4r+1.4}$}}
            \put (-1, 72){\footnotesize{$\underset{N\to\infty}{lim}N\dTheta_{\text{HPBW}}$}}
            \put (50, 2) {\footnotesize{$r$}}
        \end{overpic}
    \end{center}
    \caption{$\underset{N\to\infty}{lim}N\dTheta_{\text{HPBW}}$ vs. $r$ and comparing to $\rBrace{1-r}\rBrace{-0.4r+1.4}$}
    \label{fig_feedbackULA_beamwidth_limit_r_dependent}
\end{figure}
Following theses two important empirical observations, we formulate the feedback based HPBW expression
\begin{equation}
        \dTheta_{HPBW} =& \frac{\rBrace{1-r}\rBrace{-0.4r+1.4}}{N},
\end{equation}
which is naturally translated to the spatial feedback related HPBW improvement factor
\begin{equation}
    \mu_{\text{HPBW},\coefSetName}=\frac{1.4}{\rBrace{1-r}\rBrace{-0.4r+1.4}},
\end{equation}
stating that a perfectly aligned array (i.e. $r\to1$) will achieve a zero-width spatial beam and an infinite improvement over the classic passive ULA case.