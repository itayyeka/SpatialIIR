The Half-Power-Beam-Width (HPBW) parameter quantifies the array's main lobe narrowness, marking the DOA where the beampattern's energy reduces to half of its maximal value.
For standard ULA, with aperture of $N\delta$, it is known \cite{van2004optimum} that for large $N$,
$$
 \frac{\lambda\dThetaHPBW}{2\pi{}\delta} = \frac{\lambda}{\pi{}N\delta}1.4,
$$
where $\lambda$ is the signal wavelength and $\dThetaHPBW/2$ is the electrical angle where the HPBW is obtained. This result is more commonly expressed as $\dThetaHPBW/2= 1.4/N$. 
\par In App.~\ref{apdx_HPBW} we extend this known result for any $r\geq 0$. It turns out that for large $N$, the HPBW is obtained by solving the equation
\begin{equation}\label{eq_HPBW}
    % \resizebox{1\linewidth}{!}{
        \begin{split}
            \rBrace{r^{2}-4r+2}\frac{\sin{\rBrace{x}}^{2}}{x^{2}}+r\frac{\sin{\rBrace{2x}}}{x}-1=0
        \end{split}
    % }
\end{equation}
where we define $x\triangleq{} N\dTheta/2$. In Fig.~\ref{fig_feedbackULA_HPBW_Nx_vs_N_variousR} we plot the numerical solution of \eqref{eq_HPBW} for various values of $r$ and $N$, showing that $x$ reaches its limit already for $N>20$. Also note that for $r=0$ we obtain the known result of standard ULA with the limiting factor of $1.4$.
Having the limiting factors for various values of the gain mismatch $r$, we investigate the feedback related improvement, expressing the HPBW by
\[
\dThetaHPBW/2\approx \frac{1.4}{f(r)N}
\]
where $f(r)$ represents the array aperture improvement factor, compared to the standard ULA. 
Fig.~\ref{fig_feedbackULA_beamwidth_limit_r_dependent} shows that $f(r)$ is closely fitted with a second order polynomial
\begin{equation}
    \label{eq_Bapprox}
    \hat{f}\rBrace{r}\approx\frac{1.4}{\rBrace{1-r}\rBrace{-0.4r+1.4}}.
\end{equation}
Note that for accurate gain match (i.e. $r\to1$), $\hat{f}\rBrace{r}$ tends towards infinity, implying that the equivalent array has infinite number  of elements ($\hat{f}\rBrace{r}N$), hence obtaining perfect spatial selectivity.
\begin{figure}[t]
    \begin{center}
        \begin{overpic}[width=0.65\linewidth, 
        %grid, 
        tics=10,trim=0 0 0 0]{./Media/spatial_IIR_MATLAB/arrayParameters/HPBW_vs_N_various_r.eps}
            \put (4, 75){\footnotesize{$N\dThetaHPBW/2$}}
            \put (50, 62.5) {\footnotesize{$r=0$}}
            \put (50, 54) {\footnotesize{$r=0.1$}}
            \put (50, 39.5) {\footnotesize{$r=0.3$}}
            \put (50, 28.5) {\footnotesize{$r=0.5$}}
            \put (50, 19.75) {\footnotesize{$r=0.7$}}
            \put (50, 12.5) {\footnotesize{$r=0.9$}}
            \put (50, 2) {\footnotesize{$N$}}
        \end{overpic}
    \end{center}
     \caption{Plot of $x=N\dThetaHPBW/2$ vs. $N$, for various $r$ values, obtained by numerically solving \eqref{eq_HPBW}.}
    \label{fig_feedbackULA_HPBW_Nx_vs_N_variousR}
\end{figure}
\begin{figure}[t]
    \begin{center}
        \begin{overpic}[width=0.65\linewidth, 
        % grid, 
        tics=10,trim=0 0 0 0]{./Media/HPBW_limit_vs_r.eps}
            \put (39.5, 64) {\scriptsize{Numerical solution of \eqref{eq_HPBW}}}
            \put (39.5, 58) {\scriptsize{$N\dTheta/2$ fitting}}
            \put (39.5, 52.5) {\footnotesize{$\log_{10}\hat{f}\rBrace{r}$}}
            \put (85, 75) {\footnotesize{$\log_{10}\hat{f}\rBrace{r}$}}
            \put (4, 75){\footnotesize{$N\dThetaHPBW/2$}}
            \put (50, 2) {\footnotesize{$r$}}
        \end{overpic}
    \end{center}
    \caption{Evaluation of $N\dThetaHPBW/2$ for $N=100$ and its approximation $1.4/\hat{f}\rBrace{r}$  (marked by red diamonds). $\hat{f}\rBrace{r}$ is also presented, in logarithmic scale (dotted curve).} 
    \label{fig_feedbackULA_beamwidth_limit_r_dependent}
\end{figure}