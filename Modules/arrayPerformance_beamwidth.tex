A commonly used parameter for quantifying the array's main lobe narrowness is the Half-Power-Beam-Width (HPBW), marking the DOA where the beampattern's energy reduces to half of its maximal value (i.e. $\abs{\Hr}^2=\frac{1}{2}$). 
\par For standard ULA, it is known \cite{VanTrees2002DetectionIV} that for large $N$,
$$
 \frac{\lambda\dTheta_{\text{HPBW}}}{2\pi{}\delta} = \frac{\lambda}{\pi{}N\delta}1.4,
$$
where $\lambda$ is the signal wavelength, and $\dTheta_{\text{HPBW}}/2$ is the electrical angle where the HPBW is obtained. This result is more commonly expressed as $N\dTheta_{\text{HPBW}}/2= 1.4$, where we note that that $N\delta$ is the array aperture size.
\par In App.~\ref{apdx_HPBW} we extend this known result for any $r\geq 0$. It turns out that for large $N$, the HPBW is obtained by solving the equation
\begin{equation}\label{eq_HPBW}
    % \resizebox{1\linewidth}{!}{
        \begin{split}
            \rBrace{r^{2}-4r+2}\frac{\sin{\rBrace{x}}^{2}}{x^{2}}+r\frac{\sin{\rBrace{2x}}}{x}-1=0
        \end{split}
    % }
\end{equation}
where we define $x\triangleq{} N\dTheta/2$. In Fig.~\ref{fig_feedbackULA_HPBW_Nx_vs_N_variousR} we plot the numerical solution of \eqref{eq_HPBW} for various values of $r$ and $N$, showing that $x$ reaches its limit already for $N>20$. Also note that for $r=0$ we obtain the known result of classical ULA with the limiting factor of $1.4$.
Having the limiting factors for various values of the gain mismatch $r$, we investigate the feedback related improvement.
Specifically, we can approximate the HPBW by
\[
\dTheta_{\text{HPBW}}/2\approx \frac{1}{B(r)N}
\]
% where $B(r)=1/\rBrace{1-r}\rBrace{-0.4r+1.4}$ is the second order polynomial approximation of the limiting values.
where $B(r)$ represents the array aperture improvement factor (compared to the classical ULA). In Fig.~\ref{fig_feedbackULA_beamwidth_limit_r_dependent} we show that $B(r)$ may be approximated with a second order polynomial, such that
\begin{equation}
    \label{eq_Bapprox}
    B(r)\approx\frac{1}{\rBrace{1-r}\rBrace{-0.4r+1.4}}.
\end{equation} 
Note that large values of $B(r)$ imply that the suggested feedback based system, has smaller HPBW. 
Another interpretation is that classical ULA needs to be $B$-times larger in order to achieve that same performance.
As evident from \eqref{eq_Bapprox} and Fig.~\ref{fig_feedbackULA_beamwidth_limit_r_dependent}, feedback based arrays with close-to-unity gain mismatch factors ($r$) are equivalent to classical ULAs of infinite apertures.
\begin{figure}[t]
    \begin{center}
        \begin{overpic}[width=0.65\linewidth, 
        %grid, 
        tics=10,trim=0 0 0 0]{./Media/spatial_IIR_MATLAB/arrayParameters/HPBW_vs_N_various_r.eps}
            \put (4, 75){\footnotesize{$N\dTheta_{\text{HPBW}}/2$}}
            \put (50, 62.5) {\footnotesize{$r=0$}}
            \put (50, 54) {\footnotesize{$r=0.1$}}
            \put (50, 39.5) {\footnotesize{$r=0.3$}}
            \put (50, 28.5) {\footnotesize{$r=0.5$}}
            \put (50, 19.75) {\footnotesize{$r=0.7$}}
            \put (50, 12.5) {\footnotesize{$r=0.9$}}
            \put (50, 2) {\footnotesize{$N$}}
        \end{overpic}
    \end{center}
     \caption{Plot of $N\dTheta_{\text{HPBW}}/2$ vs. $N$, for various $r$ values. In each simulation, $r$ is constant and $\dTheta_{\text{HPBW}}$ is calculated for each $N$ separately. The product $N\dTheta_{\text{HPBW}}/2$ reaches its limit value approximately for $N>20$.}
    \label{fig_feedbackULA_HPBW_Nx_vs_N_variousR}
\end{figure}
\begin{figure}[t]
    \begin{center}
        \begin{overpic}[width=0.65\linewidth, 
        % grid, 
        tics=10,trim=0 0 0 0]{./Media/HPBW_limit_vs_r.eps}
            
            \put (51, 61) {\footnotesize{Simulation results}}
            \put (51, 55) {\footnotesize{Approximation}}
            \put (51, 49.5) {\footnotesize{$B\rBrace{r}/1.4$}}
            \put (87, 75) {\footnotesize{$B\rBrace{r}/1.4$}}
            \put (4, 75){\footnotesize{$N\dTheta_{\text{HPBW}}/2$}}
            \put (50, 2) {\footnotesize{$r$}}
        \end{overpic}
    \end{center}
    \caption{Evaluation of $N\dTheta_{\text{HPBW}}/2$ for $N=100$ and its approximation $1/B(r)$  (marked by red diamonds) with $B(r)$ of \eqref{eq_Bapprox}. Also presented the array aperture improvement factor $B\rBrace{r}/1.4$  (dotted curve).} 
    \label{fig_feedbackULA_beamwidth_limit_r_dependent}
\end{figure}