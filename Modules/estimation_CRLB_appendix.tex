A good evaluation method for the contribution of our proposed feedback mechanism is to measure the additional information in the system. Basically, it means that we calculate the improvement in the system's parameters estimation theoretical performance comparing to non-feedback-based systems. The most common way of expressing this measure is the Cramer-Rao Lower Bound (CRLB) which sets a lower bound to the variance of unbiased estimators. Due to the fact that our feedback-based transfer function (\ref{eqn:GeneralFeedbackTransferFunction}) is expressed in the frequency domain, it seems natural the we should also, as done in \cite{ARIELAZEIRAANDARYENEHORAIFrequencyProcesses}, express the CRLB in the same domain. Following \cite{ARIELAZEIRAANDARYENEHORAIFrequencyProcesses}'s steps, with a simple modification of our system model (i.e. adding zero-mean Gaussian noise to the output of our system), we express the Fisher Information Matrix (FIM) of the system.
\par The unknown parameters in our case are $ \eta = [\tau,\theta]^{T} $, therefore \cite{ARIELAZEIRAANDARYENEHORAIFrequencyProcesses} dictates that
\resizebox{.97\linewidth}{!}{
  \begin{minipage}{\linewidth}
      \begin{align}
        \nonumber
        \left[J(\eta)\right]_{kl} = 
        \Re\Bigg\{&
        \frac{1}{2\pi}
        \int_{-\frac{\omega_{s}}{2}}^{\frac{\omega_{s}}{2}}
        {
        \frac{1}{\phi\left(\omega,\eta\right)}
        \mathfrak{F}^{*}\left\{
        \frac{\partial y(t,\eta)}{\partial\eta_{k}}
        \right\}
        \mathfrak{F}\left\{
        \frac{\partial y(t,\eta)}{\partial\eta_{l}}
        \right\}
        d\omega
        }
        \\ &+
        \frac{T}{4\pi}
        \int_{-\frac{\omega_{s}}{2}}^{\frac{\omega_{s}}{2}}
        \frac{
        \frac{\partial\phi\left(\omega,\eta\right)}{\eta_{k}}
        \frac{\partial\phi\left(\omega,\eta\right)}{\eta_{l}}
        }
        {\phi^{2}\left(\omega,\eta\right)}
        d\omega
        \Bigg\}
    \end{align}
  \end{minipage}
}
where $\Re$ is the real-part-extraction operator, $k,l = 1..2$, $\phi\left(\omega,\eta\right)$ is the noise spectrum, $\mathfrak{F}$ is the Fourier transform operator and $T$ is the coherent integration time. In our model, the noise is white and does not depend on any of the parameters so the later expression (i.e. 
$
\frac{T}{4\pi}
\int_{-\frac{\omega_{s}}{2}}^{\frac{\omega_{s}}{2}}
\frac{
\frac{\partial\phi\left(\omega,\eta\right)}{\eta_{k}}
\frac{\partial\phi\left(\omega,\eta\right)}{\eta_{l}}
}
{\phi^{2}\left(\omega,\eta\right)} 
$) is $0$ and one can write $\phi\left(\omega,\eta\right) = \phi_{NOISE}$. Denoting $A,B$ and $g$ such that, 
$
\frac{\partial \vecnot{d}_{\theta}}{\partial\theta} = -j\omega\frac{d\sin\left(\theta\right)}{c}diag\left\{{\left[0,1,\hdots,N-1\right]}\right\} \triangleq A\vecnot{d}_{\theta}
$
, $B \triangleq \vecnot{d}_{\theta}\vecnot{d}^{T}_{\theta}A - A\vecnot{d}_{\theta}\vecnot{d}^{T}_{\theta}$ and $g \triangleq \vecnot{\beta}^{T}\vecnot{d}_{\theta}e^{-j\omega\tau}$, enables writing the FIM elements as 
\\
\ifdefined\showDev
    \fbox{
    \begin{minipage}{.95\linewidth}
    \textbf{development specifics}
    \begin{equation*}
    \begin{split}
        \frac{\partial y_{\theta}^{\mathcal{F}}(\omega) }{\partial\theta}
        &=
        \frac{
        \vecnot{\alpha}^{T}Ad_{\theta}e^{-j\omega\tau}
        -\vecnot{\alpha}^{T}Ad_{\theta}\vecnot{\beta}^{T}d_{\theta}e^{-2j\omega\tau}
        +\vecnot{\alpha}^{T}d_{\theta}\vecnot{\beta}^{T}Ad_{\theta}e^{-2j\omega\tau}
        }{
        \left(1-g\right)^{2}
        }
        \\
        &=
        \frac{
        \vecnot{\alpha}^{T}
        \left(
        Ad_{\theta}+B\vecnot{\beta}e^{-j\omega\tau}
        \right)e^{-j\omega\tau}
        }{
        \left(1-g\right)^{2}
        }
        \\
        \frac{\partial y_{\theta}^{\mathcal{F}}(\omega) }{\partial\tau}
        &=
        \frac{
        -j\omega\vecnot{\alpha}^{T}\vecnot{d}_{\theta}e^{-j\omega\tau}\left(1-\vecnot{\beta}^{T}\vecnot{d}_{\theta}e^{-j\omega\tau}\right)
        -j\omega\vecnot{\alpha}^{T}\vecnot{d}_{\theta}\vecnot{\beta}^{T}\vecnot{d}_{\theta}e^{-2j\omega\tau}
        }{
        \left(1-g\right)^{2}
        }
        \\
        &=
        \frac{
        -j\omega\vecnot{\alpha}^{T}\vecnot{d}_{\theta}e^{-j\omega\tau}
        }{
        \left(1-g\right)^{2}
        }
    \end{split}
    \end{equation*}
    \end{minipage}
    }
\else
\fi
\resizebox{.97\linewidth}{!}{
  \begin{minipage}{\linewidth}
    \begin{equation}
        \label{eqn_FIMelements}
        \begin{split}
            J_{\theta\theta}
            = &
            \frac{1}{2\pi\phi}
            \int_{-\frac{\omega_{s}}{2}}^{\frac{\omega_{s}}{2}}
            \left|
            \frac
            {
            \vecnot{\alpha}^{T}
            \left(
            Ad_{\theta} + B\vecnot{\beta}e^{-j\omega\tau}
            \right)
            }
            {\left(1-g\right)^{2}}
            \right|^{2}
            \left|
            s\left(\omega\right)
            \right|^{2}
            d\omega
            \\
            J_{\tau\tau}
            = &
            \frac{1}{2\pi\phi}
            \int_{-\frac{\omega_{s}}{2}}^{\frac{\omega_{s}}{2}}
            \frac
            {\omega^{2}
            \left|
            \vecnot{\alpha^{T}}\vecnot{d}_{\theta}
            \right|^{2}}
            {\left|\left(1-g\right)^{2}\right|^{2}}
            \left|
            s\left(\omega\right)
            \right|^{2}
            d\omega
            \\
            J_{\theta\tau} = J_{\tau\theta}^{*}
            = &
            \Re
            \left\{
            \frac{1}{2\pi\phi}
            \int_{-\frac{\omega_{s}}{2}}^{\frac{\omega_{s}}{2}}
            \frac
            {
            j\omega\vecnot{\alpha}^{T}
            \left(
            A\vecnot{d}_{\theta}+B\vecnot{\beta}e^{-j\omega\tau}
            \right)
            \vecnot{\alpha}^{H}\vecnot{d}^{*}_{\theta}
            }
            {\left|\left(1-g\right)^{2}\right|^{2}}
            \left|
            s\left(\omega\right)
            \right|^{2}
            d\omega
            \right\}
        \end{split}
    \end{equation}
  \end{minipage}
}
\ifdefined\showDev
    \fbox{
    \begin{minipage}{.95\linewidth}
    \textbf{development specifics}
    \begin{align*}
        \det\left\{J\right\}
        &=
        J_{\theta\theta}J_{\tau\tau}-J_{\theta\tau}J_{\tau\theta}
        \\
        &=
        \frac{1}{4\pi^{2}\Phi^{2}}
        \int_{-\frac{\omega_{s}}{2}}^{\frac{\omega_{s}}{2}}
        \frac{
        \left|s\left(\omega\right)\right|^{2}
        }{
        \left|\left(1-g\right)^{2}\right|^{4}
        }
        \omega^{2}
        \Biggl(
        \left|\vecnot{\alpha^T\vecnot{d}_{\theta}}\right|^{2}
        \left|\vecnot{\alpha}^T\left(A\vecnot{d}_{\theta}+B\becnot{\beta}e^{-j\omega\tau}\right)\right|^{2}
        \\
        &
        -\Re^{2}
        \left\{
        j\omega\vecnot{\alpha}^{T}
        \left(
        A\vecnot{d}_{\theta}+B\vecnot{\beta}e^{-j\omega\tau}
        \right)
        \vecnot{\alpha}^{H}\vecnot{d}^{*}_{\theta}
        \right\}
        \Biggl)
        d\omega
        \\
    \end{align*}
    \end{minipage}
    }
\else
\fi
Analyzing the FIM elements (\ref{eqn_FIMelements}), few important observation arise
\todo{TODO}
\begin{itemize}
    \item emphasize the $\tau$ sensitivity through $1-g$ 
    \item signal bandwidth non-negative contribution
    \item The fact that there is no information on the actual $\tau$ but on $\omega\tau \% 2\pi$
\end{itemize}
Here I intend to plot the $CRLB_{\theta}$ vs $\tau_{Err}$ and the signal bandiwdth.