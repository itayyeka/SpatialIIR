A possible evaluation for the contribution of the presented feedback mechanism is to measure the additional information in the system.
To this end, the FIM, denoted by $\vecnot{J}$, will now be calculated with respect to the DOA $\rBrace{\thetaD}$ and range $\rBrace{\phi}$ parameters. 
As the feedback-based transfer function (\ref{eqn:GeneralFeedbackTransferFunction}) is expressed in frequency domain, we rely on \cite{zeira1990frequency} to express the frequency domain FIM as well. 
\par The $\vBrace{k,l}$'th FIM element, may be expressed as
\begin{equation}\label{eq_FIM_kl_full}
    \resizebox{.9\linewidth}{!}{
        \begin{split}
            J_{k,l}\rBrace{\vEta} 
            =&
            \Re\cBrace{
            \frac{1}{2\pi}
            \int_{-\omega_{s}/2}^{\omega_{s}/2}
            {
            \frac{1}{\Phi\rBrace{\omega}}
            \mathfrak{F}^{*}\left\{
            \frac{\partial z(t)}{\partial\eta_{k}}
            \right\}
            \mathfrak{F}\left\{
            \frac{\partial z(t)}{\partial\eta_{l}}
            \right\}
            d\omega
            }}
            \\ &+
            \frac{T}{4\pi}
            \int_{-\omega_{s}/2}^{\omega_{s}/2}
            \frac{1}{\Phi^{2}\rBrace{\omega}}
            \rBrace{\frac{\partial\Phi\rBrace{\omega}}{\partial\eta_{k}}}^{\ast}
            \frac{\partial\Phi\rBrace{\omega}}{\partial\eta_{l}}
            d\omega
        \end{split}
    }
\end{equation}
where $ \vEta = [\thetaD,\phi]^{T} $ is the parameters vector, $\Re$ stands for the real-part extraction operator, $k,l \in\cBrace{1,2}$, $\Phi\rBrace{\omega}$ is the noise spectrum, $\mathfrak{F}$ is the Fourier transform operator, $T$ is the measurement observation interval and $\omega_{s}$ is the signal bandwidth. 
For simplicity, $\text{n}\rBrace{t}$ is assumed to be a white Gaussian with some constant power spectral density $\Phi(\omega)=\sigma^2$ and independent of the estimated parameters $\vEta$. Hence, the second term in the right-hand-side of \eqref{eq_FIM_kl_full} vanishes. 
Assuming continuously differentiable functions, where order alteration of the Fourier transform and the differentiation operations is allowed, \eqref{eq_FIM_kl_full} simplifies to
\begin{equation}
    \label{eq_beamPatternFreqDomain_FIM}
    % \resizebox{1\linewidth}{!}{
        \begin{split}
            J_{kl}\rBrace{\vEta} = 
            \Re\cBrace{
            \frac{1}{2\pi\sigma^2}
            \int_{-\omega_{s}/2}^{\omega_{s}/2}
            {
            \rBrace{\frac{\partial{}\F{z}\rBrace{\omega}}{\partial\eta_{k}}}^{\ast}
            \frac{\partial{}\F{z}\rBrace{\omega}}{\partial\eta_{l}}
            d\omega
            }}
        \end{split}.
    % }
\end{equation}
Expressing the steering vector derivative with respect to $\thetaD$, results in
\begin{equation}\label{eq_vdDiff}
\frac{\partial\vd}{\partial\thetaD}=\vecnot{A}\omegaB\vd
\end{equation}
where $\vecnot{A}\omegaB$ is an $N\times{}N$ diagonal matrix with
\[
A_{ii}\omegaB=-j\omega\frac{\partial \tau_{i}}{\partial{\thetaD}}\ \  \forall{i\in\cBrace{0\hdots{}N-1}}.
\]
It is worth mentioning that \eqref{eq_vdDiff} is relevant even for arbitrary arrays (not necessarily ULA) when smooth and slowly changing radiation patterns are assumed.
In App.~\ref{apdx_clacFim} we compute the FIM terms, concluding that
\begin{equation}
    \label{eqn_FIMelements}
    \resizebox{.91\linewidth}{!}{
        \begin{split}
            &J_{11}=J_{\thetaD\thetaD}
            =
            \frac{1}{2\pi\sigma^{2}}\int_{-\omega_{s}/2}^{\omega_{s}/2}{\frac{
            \lBrace{g\vBetaH{}\vecnot{A}\omegaB\vd-g^{2}\vBetaH{}B\omegaB\vAlphaC\ePhi{-}}^{2}
            }{
            \lBrace{1-g\aTH\ePhi{-}}^{4}
            }\lBrace{\F{s}\rBrace{\omega}}^{2}d\omega}
            \\
            &J_{22}=J_{\phi\phi}
            =
            \frac{1}{2\pi\sigma^{2}}\int_{-\omega_{s}/2}^{\omega_{s}/2}{\frac{
            \lBrace{g\bHd}^{2}
            }{
            \lBrace{1-g\aHd\ePhi{-}}^{4}
            }\lBrace{\F{s}\rBrace{\omega}}^{2}d\omega}.
        \end{split}
    }
\end{equation}
where $\vecnot{B}\omegaB\triangleq\vd\vdT{}\vecnot{A}\omegaB-\vecnot{A}\omegaB\vd\vdT.$ 
% Generalizing the conventional beamformer (CB) \cite{van2004optimum}, such that the feedback weights are
% \begin{equation}\label{eq:alpha_beta_opt}
% \vAlpha_{\text{CB,opt}}=\frac{\vdC\exp\rBrace{j\phi}}{\hat{g}\norm{\vd}^2}
% \end{equation}
% where $\hat{g}$ is the channel gain estimate, both maximizes the FIM diagonal elements (via denominator minimization) and nulls it's cross-terms (see App.~\ref{apdx_clacFim}), thus significantly increases the available information.
% Assuming perfect knowledge of the target's location, while
Aiming to maximize the FIM diagonal elements via denominator (i.e., $\lBrace{1-g\aHd\ePhi{-}}$) minimization, the optimal feedback weights are 
\begin{equation}\label{eq:alpha_beta_opt}
\vAlphaC_{\text{CB,opt}}=\frac{\vdC\exp\rBrace{j\phi}}{\hat{g}\norm{\vd}^2},
\end{equation}
where $\hat{g}$ is the channel gain estimate.
This choice of weights may be interpreted as a generalized version of the conventional beamformer (CB) \cite{van2004optimum}. 
Furthermore, setting $\vBeta=\vBeta_{\text{CB,opt}}=\vAlpha_{\text{CB,opt}}$, is shown (see App.~\ref{apdx_clacFim}) to nullify the FIM cross terms, such that $J_{12}=J_{\thetaD,\phi}=J_{21}=J_{\phi\thetaD}=0.$
\par Note that setting the feedback weights as in \eqref{eq:alpha_beta_opt} requires perfect knowledge of the target's range, since $\phi$ is range dependant. 
In practice, though, there will be unavoidable errors, and perfect knowledge of target's location is usually unknown.
In Sec.~\ref{sec_Performance}, we quantify the effect of such errors and discuss its influence on the array performance. 