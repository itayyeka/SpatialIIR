A possible evaluation for the contribution of the presented feedback mechanism is to measure the additional information in the system.
To this end, the FIM will now be calculated with respect to the DOA parameter $\thetaD$ and the range related parameter $\phi$. 
As the feedback-based transfer function (\ref{eqn:GeneralFeedbackTransferFunction}) is expressed in the frequency domain, we rely on \cite{zeira1990frequency}, to express the FIM in the frequency domain as well. 
\par A single element of the FIM, notated by the matrix $J$, can be expressed as
\begin{equation}
    \resizebox{.9\linewidth}{!}{
        \begin{split}
            J_{\vBrace{k,l}}\rBrace{\vEta} 
            =&
            \Re\cBrace{
            \frac{1}{2\pi}
            \int_{-\omega_{s}/2}^{\omega_{s}/2}
            {
            \frac{1}{\Phi\rBrace{\omega}}
            \mathfrak{F}^{*}\left\{
            \frac{\partial z(t)}{\partial\eta_{k}}
            \right\}
            \mathfrak{F}\left\{
            \frac{\partial z(t)}{\partial\eta_{l}}
            \right\}
            d\omega
            }}
            \\ &+
            \frac{T}{4\pi}
            \int_{-\omega_{s}/2}^{\omega_{s}/2}
            \frac{1}{\Phi^{2}\rBrace{\omega}}
            \frac{\partial\Phi\rBrace{\omega}}{\partial\eta_{k}}
            \frac{\partial\Phi\rBrace{\omega}}{\partial\eta_{l}}
            d\omega
        \end{split}
    }
\end{equation}
where $ \vEta = [\thetaD,\phi]^{T} $ is the parameters vector, $\Re$ stands for the real-part extraction operator, $k,l \in\cBrace{1,2}$, $\Phi\rBrace{\omega}$ is the noise spectrum, $\mathfrak{F}$ is the Fourier transform operator, $T$ is the measurement observation interval and $\omega_{s}$ is the signal bandwidth. 
For simplicity, $\text{n}\rBrace{t}$ is assumed to be white and Gaussian with some constant power spectral density $\Phi(\omega)=\sigma^2$ and does not depend on the estimated parameters $\eta$, hence the second term vanishes. 
Assuming continuously differentiable functions, where order alteration of the Fourier transform and the differentiation operations is allowed, the FIM's $\vBrace{k,l}$'th element is
\begin{equation}
    \label{eq_beamPatternFreqDomain_FIM}
    % \resizebox{1\linewidth}{!}{
        \begin{split}
            J_{\vBrace{k,l}}\rBrace{\vEta} = 
            \Re\cBrace{
            \frac{1}{2\pi\sigma^2}
            \int_{-\omega_{s}/2}^{\omega_{s}/2}
            {
            \rBrace{\frac{\partial{}\F{z}\rBrace{\omega}}{\partial\eta_{k}}}^{\ast}
            \frac{\partial{}\F{z}\rBrace{\omega}}{\partial\eta_{l}}
            d\omega
            }}
        \end{split}.
    % }
\end{equation}
For omni-directional array sensors, $g$ is independent of the estimated parameters, therefore
\begin{equation}\label{eq_vdDiff}
\frac{\partial\vd}{\partial\thetaD}=A\rBrace{\omega}\vd
\end{equation}
where $A\rBrace{\omega}$ is an $N\times{}N$ diagonal matrix and each of its diagonal elements may expressed as 
\[
A_{\vBrace{i,i}}\rBrace{\omega}=-j\omega\frac{\partial \tau_{i}}{\partial{\thetaD}}\ \  \forall{i\in\cBrace{0\hdots{}N-1}}.
\]
To further simplify the analysis, without loss of generality, we use (in this section only) $g=1$.
In App.~\ref{apdx_clacFim} we compute the FIM terms, concluding that
\begin{equation}
    \label{eqn_FIMelements}
    \resizebox{.91\linewidth}{!}{
        \begin{split}
            &J_{\theta\theta}
            =
            \frac{1}{2\pi\sigma^{2}}\int_{-\omega_{s}/2}^{\omega_{s}/2}{\frac{
            \lBrace{\vBetaT{}A\rBrace{\omega}\vd-\vBetaT{}B\rBrace{\omega}\vAlpha\ePhi{-}}^{2}
            }{
            \lBrace{\rBrace{1-\aTd\ePhi{-}}^{2}}^{2}
            }\lBrace{\F{s}\rBrace{\omega}}^{2}d\omega}
            \\
            &J_{\phi\phi}
            =
            \frac{1}{2\pi\sigma^{2}}\int_{-\omega_{s}/2}^{\omega_{s}/2}{\frac{
            \lBrace{\bTd}^{2}
            }{
            \lBrace{\rBrace{1-\aTd\ePhi{-}}^{2}}^{2}
            }\lBrace{\F{s}\rBrace{\omega}}^{2}d\omega}
        \end{split}
    }
\end{equation}
where $B\rBrace{\omega}\triangleq\vd\vdT{}A\rBrace{\omega}-A\rBrace{\omega}\vd\vdT$.
Also, using some mild assumptions and setting
\begin{equation}\label{eq_alphaBetaPropSteer}
    \vAlpha,\vBeta\propto\vd^{\ast},
\end{equation}
we show that the cross terms of the FIM are nullified, i.e. $J_{\theta\phi} = J_{\phi\theta}^{*}=0$.
\par 
Choosing the weights as in \eqref{eq_alphaBetaPropSteer} may be interpreted as a generalization of the conventional beamformer (CB) \cite{van2004optimum}, more commonly referenced as the delay-and-sum (DS) beamformer which coherently integrate the impinging signal along the array elements.
The same choice of weights also minimizes the $\lBrace{1-\aTd\ePhi{-}}$ term, significantly increasing the available information, as predicted by the FIM.
It is worth mentioning that \eqref{eq_vdDiff} is relevant even for arbitrary (non-omni-directional) sensors when smooth and slowly changing radiation patterns are assumed.
In practice, though, there will be unavoidable errors, and perfect knowledge of the steering vector $\vd$ is not always available.
In Sec.~\ref{sec_Performance}, we quantify the effect of mismatching $\vAlpha$ with the desired form and discuss its influence on the array performance. 