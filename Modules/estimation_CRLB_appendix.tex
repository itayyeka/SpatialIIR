A possible evaluation for the contribution of the presented feedback mechanism is to measure the additional information in the system.
To this end, the FIM, denoted by $J$, will now be calculated with respect to the DOA parameter $\thetaD$ and the range related parameter $\phi$. 
As the feedback-based transfer function (\ref{eqn:GeneralFeedbackTransferFunction}) is expressed in frequency domain, we rely on \cite{zeira1990frequency} to express the frequency domain FIM as well. 
\par A single FIM element, may be expressed as
\begin{equation}\label{eq_FIM_kl_full}
    \resizebox{.9\linewidth}{!}{
        \begin{split}
            J_{\vBrace{k,l}}\rBrace{\vEta} 
            =&
            \Re\cBrace{
            \frac{1}{2\pi}
            \int_{-\omega_{s}/2}^{\omega_{s}/2}
            {
            \frac{1}{\Phi\rBrace{\omega}}
            \mathfrak{F}^{*}\left\{
            \frac{\partial z(t)}{\partial\eta_{k}}
            \right\}
            \mathfrak{F}\left\{
            \frac{\partial z(t)}{\partial\eta_{l}}
            \right\}
            d\omega
            }}
            \\ &+
            \frac{T}{4\pi}
            \int_{-\omega_{s}/2}^{\omega_{s}/2}
            \frac{1}{\Phi^{2}\rBrace{\omega}}
            \frac{\partial\Phi\rBrace{\omega}}{\partial\eta_{k}}
            \frac{\partial\Phi\rBrace{\omega}}{\partial\eta_{l}}
            d\omega
        \end{split}
    }
\end{equation}
where $ \vEta = [\thetaD,\phi]^{T} $ is the parameters vector, $\Re$ stands for the real-part extraction operator, $k,l \in\cBrace{1,2}$, $\Phi\rBrace{\omega}$ is the noise spectrum, $\mathfrak{F}$ is the Fourier transform operator, $T$ is the measurement observation interval and $\omega_{s}$ is the signal bandwidth. 
For simplicity, $\text{n}\rBrace{t}$ is assumed to be a white Gaussian with some constant power spectral density $\Phi(\omega)=\sigma^2$ and independent of the estimated parameters $\eta$, hence the second term vanishes. 
Assuming continuously differentiable functions, where order alteration of the Fourier transform and the differentiation operations is allowed, \eqref{eq_FIM_kl_full} simplifies to
\begin{equation}
    \label{eq_beamPatternFreqDomain_FIM}
    % \resizebox{1\linewidth}{!}{
        \begin{split}
            J_{\vBrace{k,l}}\rBrace{\vEta} = 
            \Re\cBrace{
            \frac{1}{2\pi\sigma^2}
            \int_{-\omega_{s}/2}^{\omega_{s}/2}
            {
            \rBrace{\frac{\partial{}\F{z}\rBrace{\omega}}{\partial\eta_{k}}}^{\ast}
            \frac{\partial{}\F{z}\rBrace{\omega}}{\partial\eta_{l}}
            d\omega
            }}
        \end{split}.
    % }
\end{equation}
As mentioned before, $g$ is independent of the estimated parameters, therefore
\begin{equation}\label{eq_vdDiff}
\frac{\partial\vd}{\partial\thetaD}=A\omegaB\vd
\end{equation}
where $A\omegaB$ is an $N\times{}N$ diagonal matrix and each of its diagonal elements may expressed as 
\[
A_{\vBrace{i,i}}\omegaB=-j\omega\frac{\partial \tau_{i}}{\partial{\thetaD}}\ \  \forall{i\in\cBrace{0\hdots{}N-1}}.
\]
To further simplify the analysis, without loss of generality, we use (in this section only) $g=1$.
In App.~\ref{apdx_clacFim} we compute the FIM terms, concluding that
\begin{equation}
    \label{eqn_FIMelements}
    \resizebox{.91\linewidth}{!}{
        \begin{split}
            &J_{\theta\theta}
            =
            \frac{1}{2\pi\sigma^{2}}\int_{-\omega_{s}/2}^{\omega_{s}/2}{\frac{
            \lBrace{\vBetaT{}A\omegaB\vd-\vBetaT{}B\omegaB\vAlpha\ePhi{-}}^{2}
            }{
            \lBrace{\rBrace{1-\aTd\ePhi{-}}^{2}}^{2}
            }\lBrace{\F{s}\rBrace{\omega}}^{2}d\omega}
            \\
            &J_{\phi\phi}
            =
            \frac{1}{2\pi\sigma^{2}}\int_{-\omega_{s}/2}^{\omega_{s}/2}{\frac{
            \lBrace{\bTd}^{2}
            }{
            \lBrace{\rBrace{1-\aTd\ePhi{-}}^{2}}^{2}
            }\lBrace{\F{s}\rBrace{\omega}}^{2}d\omega}
        \end{split}
    }
\end{equation}
where $B\omegaB\triangleq\vd\vdT{}A\omegaB-A\omegaB\vd\vdT$.
Moreover, using some mild assumptions and setting
\begin{equation}\label{eq_alphaBetaPropSteer}
    \vAlpha,\vBeta\propto\vd^{\ast},
\end{equation}
we show that the cross terms of the FIM are nullified, i.e. $J_{\theta\phi} = J_{\phi\theta}^{*}=0$.
\par 
Choosing the weights as in \eqref{eq_alphaBetaPropSteer} may be interpreted as a generalization of the DS beamformer, formerly referenced as the conventional beamformer (CB) \cite{van2004optimum}, which coherently integrate the impinging signal along the array elements.
The same choice of weights also minimizes the $\lBrace{1-\aTd\ePhi{-}}$ term, significantly increasing the available information, as predicted by the FIM.
It is worth mentioning that \eqref{eq_vdDiff} is relevant even for arbitrary (non-omni-directional) sensors when smooth and slowly changing radiation patterns are assumed.
In practice, though, there will be unavoidable errors, and perfect knowledge of the steering vector $\vd$ is not always available.
In Sec.~\ref{sec_Performance}, we quantify the effect of such estimation errors and discuss its influence on the array performance. 