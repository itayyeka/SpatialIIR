\section{Conclusions}
\label{sec_conclusions}
Integrating feedback into standard beamformers proved to achieve the spatial domain equivalent of the temporal IIR filtering.
It seems that a simple generalization of the conventional-beamformer maximizes (locally) the system's spatial information, thus enabling high localization accuracy.
The feedback-based architecture performance evaluation predicts an unlimited improvement in all criteria, when considering perfect knowledge of the target's range and the channel attenuation.
It turns out that a single frequency waveform based feedback-beamformer is impractical, being too sensitive to even mild target range estimation errors.
Fortunately, using a dual-frequency waveform and applying simple frequency domain manipulations to the output and feedback signals, were found to serve as a low frequency (hence low sensitivity) equivalent of the single frequency scheme.
Also, the dual frequency scheme proved to be of low noise sensitivity, featuring high performance even in relatively low signal-to-noise-ratio scenarios.
\par Future study of the feedback beamforming concept may be applied to other array processing applications other than localization.
Furthermore, it is worthwhile to inspect other interesting choices of coefficients rather than the conventional-beamformer generalization, other waveforms and associated processing schemes, to extend the results to dynamic/multiple targets and to consider sensors with general radiation patterns.
Also, one may consider generalizing the suggested architecture to multiple-input-multiple-output systems, enabling a steered/focused (rather than omni-directional) feedback transmission.