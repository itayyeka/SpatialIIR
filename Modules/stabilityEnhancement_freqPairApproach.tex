As demonstrated in the previous section, the beampattern \eqref{eq_generalH} is sensitive to range errors.
We now propose an architecture which obtains the desired beampattern $\Hr_{\Delta\theta,\Delta\phi\to0,r}$ even for relatively large range errors $\Delta R_{\text{rt}}$. Also, we show that the suggested architecture achieves high performance at moderately low signal-to-noise ratio (SNR) scenarios.

\subsection*{Intuition}
Bearing in mind that the system's phase alignment sensitivity resides in \eqref{eq_generalH} via the  term $\exp\Brack{j\dPhi}=\exp\Brack{j\omega\Delta\tau_{pd}}$ and that the round-trip delay ($\tau_{pd}$) cannot be controlled, one may suggest to use lower frequencies. 
Unfortunately, aiming for practical range estimation errors, the transmission of such low frequencies is physically unfeasible. 
Instead, we suggest simultaneous transmission of several frequencies in order to resolve the range error sensitivity. In the following, we suggest a dual frequency (DF) waveform, utilizing two harmonics, $\omega_1$ and $\omega_2$.

\subsection*{Suggested processing scheme}
\begin{figure}[t!]
    \begin{center}
        \begin{overpic}[width=0.9\linewidth, 
        % grid, 
        tics=10,trim={0 0 0 0}]{./Media/SpatialIIR_APP.png}
            \put (12.5, 64){$\text{FB}_{\vAlphaI{1},\vBetaI{1}}$}
            \put (61, 64){$\text{FB}_{\vAlphaI{2},\vBetaI{2}}$}
            \put (4.5, 59){$z_{1}\rBrace{t}$}
            \put (54, 59){$z_{2}\rBrace{t}$}
            \put (12.9, 41.5){$\text{BPF}_{\omega_{1}}$}
            \put (61.65, 41.5){$\text{BPF}_{\omega_{2}}$}
            \put (6, 51.5){+}
            \put (55, 51.5){+}
            \put (16, 51.5){\footnotesize{$\text{n}_{1}\rBrace{t}$}}
            \put (64.75, 51.5){\footnotesize{$\text{n}_{2}\rBrace{t}$}}
            \put (24.5, 59){\footnotesize{$\Tx_{1}\rBrace{t}$}}
            \put (73.5, 59){\footnotesize{$\Tx_{2}\rBrace{t}$}}
            \put (36.25, 64){\scriptsize{$s_{1}\rBrace{t}$}}
            \put (43, 64){\scriptsize{$s_{2}\rBrace{t}$}}
            \put (18.25, 13){\footnotesize{Harmonic mean}}
            \put (32, 2){$Z_{\text{DF}}$}
            \put (54.75, 24){$\Sigma$}
            \put (21.75, 25){\footnotesize{$\mathcal{F}_{\omega_{1}}$}}
            \put (30.75, 25){\footnotesize{$\mathcal{F}_{\omega_{2}}$}}
        \end{overpic}
    \end{center}
    \caption{Dual-frequency beamformer, consisting of two independent FB blocks and narrowband bandpass filters. The blocks marked by $\mathcal{F}_{\omega_{i}}$ compute the Fourier coefficients in $\omega_{i}$ and their outputs feed the harmonic mean calculator, which generates the DF beamformer's output.}
    \label{fig_app}
\end{figure}
In Fig.~\ref{fig_app}, we demonstrate the use of two independently configured FB instances (see also Fig.~\ref{fig:Proposed_spatialIIR_ARCH}), where each instance is designed to treat a specific frequency band.
A bandpass filter $\text{BPF}_{\omega_{i}},\;i=1,2$ filters a narrowband slice around $\omega_{i}=2\pi f_i$. These filters are used to generate both the transmitted feedback signal ($\text{Tx}_{i}$) and the outputs $z_{i}$. 
The inputs $s_{i}(t) = \exp\Brack{j\omega_{i}t}$  are the two narrowband stimuli signals and $\text{n}_{i}(t)$ describe the additive noise. 

Note that in the suggested architecture we do not add array elements, but merely double the beamformer processing effort. 
\par For each FB block, its output is given by
\[
z_{i}(t)=H_{\vBetaI{i},\vAlphaI{i}}\rBrace{\omega_{i}}\exp{\rBrace{j\omega_{i}t}}\ \ {i\in\cBrace{1,2}},
\]
where $\vBetaI{i},\vAlphaI{i}$ are the coefficients of the $i$'th beamformer.
Motivated by the desire to mitigate the $\Delta\phi$ dependency of the system, which appears in the denominator of \eqref{eq_generalH}, we compute the reciprocal of each frequency response, average, and compute the reciprocal again. This leads to the harmonic mean of both beamformers' outputs (see Fig.~\ref{fig_app}), which formally, up to a constant term, takes the form
\begin{equation}
    \label{eqn_HDF_def}
    Z_{\text{DF}} = \abs{H^{-1}_{\vBetaI{1}\vAlphaI{1}}\rBrace{\omega_{1}}+H^{-1}_{\vBetaI{2},\vAlphaI{2}}\rBrace{\omega_{2}}}^{-1}.
\end{equation}
% Note that, as in IIR filters, the system stabilizes after a certain period, therefore the Fourier weights should be calculated after stabilization.
% As a rule of thumb, simulations show that the stabilization period is around $100R/c$. 
\ifdefined\useOmega
Straight forward derivations, Allowing $g$ to be frequency dependent, leads to
\else
For convenience, we use subscripts instead of formal $\omega$ dependency such that $\phiI{i}\triangleq\phi\rBrace{\omega_{i}}, \gI{i}\triangleq{}g\rBrace{\omega_{i}}, \vdI{i}\triangleq\vecnot{d}\rBrace{\omega_{i}}$.
Also, $r_i\triangleq{}\gI{i}/\gIHat{i}$ is denoted to be the gain mismatch at $\omega_i$.


\begin{theorem}
\label{thrm_DF}
Consider the architecture suggested in Fig.~\ref{fig_app}, and let \textit{$\vAlphaI{i},\vBetaI{i}$} be the coefficients of the $i$'th FB. Then setting 
\begin{equation}\label{eqn_simpleBeta}
    \vAlphaI{1}~=~\vBetaI{1},\ \vAlphaI{2}=-\vBetaI{2}=\vBrace{1/\gIHat{2},0,\hdots,0},
\end{equation}
results in 
\begin{equation}
    Z_{\text{DF}} = \lBrace{\frac{\gI{1}\vBetaHI{1}\vdI{1}}{1-
    \rBrace{\gI{1}\vBetaHI{1}\vdI{1}/r_{2}}\exp\rBrace{-j\rBrace{\phiI{1}-\phiI{2}}}
    }},
    \label{eqn_H_DF_general}
\end{equation}
\end{theorem}

\begin{proof}
See App.~\ref{apdx_thrm_DF}. 
\end{proof}
\par Assuming close frequencies, $Z_{\text{DF}}$ in  \eqref{eqn_H_DF_general} closely resembles the single frequency (SF) beampattern in  \eqref{eqn:GeneralFeedbackTransferFunction}, where the range related phase $\phi$ is now replaced by  $\phiI{1}-\phiI{2}=(\omega_1-\omega_2)\tau_{pd}$. 
Hence, one may significantly mitigate the range mismatch distortion of the beampattern by selecting close stimuli frequencies.
% Also, $\tau_n$ is arbitrary, hence the discussion is not restricted to a specific array geometry.
We also note that throughout the development of \eqref{eqn_H_DF_general}, we did not assume any specific array geometry, hence this result is valid for arbitrary arrays and not just ULA.
\subsection*{Numerical example}
Consider a radio frequency carrier of $10$~GHz and typical range error of $\Delta{}R_{\text{rt}}=10$~m, which is $333\frac{1}{3}\lambda$ (assuming  speed of light, $c=3\cdot 10^{8}$~m/s). The single frequency beampattern distortion, being periodic in $\lambda$, will closely resemble the $0.3\lambda$ error plot presented in Fig.~\ref{fig_rangError}. Assume that we aim to achieve a maximal phase error of $\Delta \phi=0.01\pi$. Hence, when using DF architecture, the dictated frequency separation must satisfy
\begin{equation}\label{eq_DF_phErr}
\abs{(\omega_1-\omega_2)\frac{\Delta{}R_{\text{rt}}}{c}}<0.01\pi.
\end{equation}
or equivalently, for maximal range error of $10$~m, a frequency separation of
\[
\abs{f_1-f_2}<0.005 c/\Delta{}R_{\text{rt}}=150 \,\text{kHz}
\]
is required. 

\subsection*{Dual frequency simulation}
We now simulate \eqref{eqn_H_DF_general} for the DF structured FB,
generalizing the $\coefSetName$ approach and setting
\begin{equation*}
    \vBetaCI{1}=\frac{-\vdHatIC{1}\exp\rBrace{j\rBrace{\phiIHat{1}-\phiIHat{2}}}}{\hat{\gI{1}}\norm{\vdHatI{1}}^2}.
\end{equation*}
% where $\vdHatI{1}$ is the estimated steering vector, as in \eqref{eq:d_hat}.
With this choice, and similarly to \eqref{eq:SF_CB}, \eqref{eqn_H_DF_general} becomes
\begin{equation}
    \label{eqn_H_DF_CB}
    \resizebox{.89\linewidth}{!}{
        \begin{split}
            H_{\text{DF,\coefSetName}}\rBrace{\omega} =
            \lBrace{\frac{r_{1}\D{\dTheta/2}{N}}{1-
            \kappa\D{\dTheta/2}{N}\exp\rBrace{-j\rBrace{\phiI{2}-\phiI{1}+(N-1)\dTheta/2}}}
            },
        \end{split}
    }
\end{equation}
where $\kappa\triangleq{}r_{1}/r_{2}$ is the gain mismatch ratio.
For close frequencies, one may assume that $r_{1}\approx{}r_{2}$, hence $\kappa$ tends towards unity, thus significantly mitigating the gain mismatch effect even when both feedback beamformers are mismatched.
\par Simulating the DF architecture, configured to mitigate range estimation errors as in \eqref{eq_DF_phErr} and plotting its normalized (to $0$~dB peak gain) beampattern together with the perfectly range aligned scenario, we show in Fig.~\ref{fig_dualfreq_rangeErrorHighSnr} that the DF architecture achieves near-optimal performance, despite the inherent range error of $\Delta{}R_{\text{rt}}=10$~m.
In Fig.~\ref{fig_dualfreq_perfectAlignLowSnr}, we repeat the simulation while adding white Gaussian noise to the output of each feedback beamformer. Evidently, even in the noisy case, the DF beamformer achieves close-to-ideal beampattern, while the SF beamformer suffers severe distortions.
\begin{figure}[t!]
    \begin{center}
        \begin{overpic}[width=.65\linewidth, 
        % grid, 
        tics=10,trim=0 0 0 0]{./Media/fig_dualfreq_rangeErrorHighSnr.eps}
            \put (48, 43){\scriptsize{Ideal}}
            \put (48, 37.5){\scriptsize{SF}\tiny{$_{\Delta{}R_{\text{rt}}=10_{m}}$}}
            \put (48, 32){\scriptsize{DF}\tiny{$_{\Delta{}R_{\text{rt}}=10_{m}}$}}
            \put (2, 37.5){\footnotesize{dB}}
            \put (47,0){\footnotesize{$\thetaD/\pi$}}
        \end{overpic}
    \end{center}
    \caption{Simulating the 3 element ULA with $r_1=0.6^{2},\; r_2=0.6$ (hence $\kappa=0.6$), assuming an infinite SNR. For each target direction $\thetaD$, the DF beamformer output $Z_\text{DF}$ is evaluated where the beamformer is set to enhance signals impinging from $\thetaD=\pi/2$. 
    The (modulus $\lambda$) range error is $\Delta{}R_{\text{rt}}=0.3\lambda$.
    The single frequency (SF) beamformer (red squares) and the dual-frequency (DF) solution (green diamonds) are compared to the ideal response  $\Delta{}R_{\text{rt}}=0$ (blue dots) as a reference. 
    }
    \label{fig_dualfreq_rangeErrorHighSnr}
\end{figure}
% \begin{figure}[t!]
%     \begin{center}
%         \begin{overpic}[width=0.9\linewidth, 
%         % grid, 
%         tics=10,
%         % trim={<left> <lower> <right> <upper>}
%         trim={1.75cm 0 1.75cm 0}
%         ]{./Media/fig_dualfreq_rangeErrorLowSnr.eps}
%             \put (42.25, 17.5){\scriptsize{Ideal}}
%             \put (42.25, 14.5){\scriptsize{SF}}
%             \put (42.25, 11.5){\scriptsize{DF}}
%             \put (-1, 26.5){\footnotesize{dB}}
%             \put (22, 0){\footnotesize{$\thetaD/\pi$}}
%             \put (74.5, 0){\footnotesize{$\thetaD/\pi$}}
%             \put (19,17){\scriptsize{$\text{SNR}=6_{dB}$}}
%             \put (71.5,17){\scriptsize{$\text{SNR}=0_{dB}$}}
%         \end{overpic}
%     \end{center}
%     \caption{Directional response of the 3 element ULA, as in Fig.~\ref{fig_dualfreq_rangeErrorHighSnr}, simulated for the noisy scenario. The additive noises $\text{n}_1(t)$ and $\text{n}_2(t)$ (see Fig.~\ref{fig_app}), are set to obtain SNR of $6_{dB}$ (left plot) and $0_{dB}$ (right plot).}
%     \label{fig_dualfreq_perfectAlignLowSnr}
% \end{figure}
\begin{figure}[t!]
    \begin{center}
        \begin{overpic}[width=.9\linewidth, 
        % grid, 
        tics=10,
        % trim={<left> <lower> <right> <upper>}
        trim={1.75cm 0 1.75cm 0}
        ]{./Media/fig_dualfreq_rangeErrorLowSnr_NEW.eps}
            \put (9, 30){\scriptsize{Ideal}}
            \put (9, 25.5){\scriptsize{SF}}
            \put (9, 21){\scriptsize{DF}}
            \put (-4, 60){\footnotesize{dB}}
            \put (53, 60){\footnotesize{dB}}
            \put (25, 20){\footnotesize{dB}}
            \put (20, 42){\footnotesize{$\thetaD/\pi$}}
            \put (77, 42){\footnotesize{$\thetaD/\pi$}}
            \put (49, 2){\footnotesize{$\thetaD/\pi$}}
            \put (16,52){\scriptsize{$\text{SNR}=6_{dB}$}}
            \put (74,52){\scriptsize{$\text{SNR}=0_{dB}$}}
            \put (44.5,12){\scriptsize{$\text{SNR}=-6_{dB}$}}
            \put (4,42.5){\footnotesize{(a)}}
            \put (60.5,42.5){\footnotesize{(b)}}
            \put (32,2){\footnotesize{(c)}}
        \end{overpic}
    \end{center}
    \caption{Directional response of the 3 element ULA, as in Fig.~\ref{fig_dualfreq_rangeErrorHighSnr}, simulated for the noisy scenario. The additive noises $\text{n}_1(t)$ and $\text{n}_2(t)$ (see Fig.~\ref{fig_app}), are set to obtain SNRs of $6$~dB (a), $0$~dB (b) and $-6$~dB (c).}
    \label{fig_dualfreq_perfectAlignLowSnr}
\end{figure}