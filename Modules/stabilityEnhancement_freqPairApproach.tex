The transfer function (\ref{eqn:GeneralFeedbackTransferFunction}) analysis and the supporting simulations, unveiled a significant sensitivity to the phase correction of the feedback-based coefficients. For nominal scenarios of both acoustic and EM signals, the necessary target range related phase alignment, had to be too accurate for practical applications. We propose an architecture which achieves the goal transfer function while significantly relaxing the need for phase alignment accuracy.
\subsection*{Motivation}
Bearing in mind that the system's phase alignment sensitivity resides in the $e^{-j\omega\tau}$ term, and that the target's range (i.e. $\tau$) cannot be controlled, obviously leads to using lower frequencies. Unfortunately, in most cases, using such frequencies is in not physically feasible. One can recall the trigonometric analogy between sum of frequencies and their multiplication, which guides us to predict that using two relatively close frequencies $\left(f_{1},f_{2}=f_{1}+\Delta_{f} \ , \Delta_{f} \ll f_{1}\right)$ may yield the desired low-frequency-dependent transfer function.
\subsection*{Suggested processing scheme}
Assume $x(t) = e^{j\omega_{1}t}+e^{j\omega_{2}t}, \omega_{i} = 2\pi{f_{i}}$ acts as the stimulating signal, where two independent feedback based systems produce $
y_{\theta,i}(t)=h\rBrace{\theta,\omega_{i}}exp\rBrace{j\omega_{i}t}\ \for{i\in\vBrace{1,2}}.$ Using one of many techniques (e.g. FFT, DTFT etc.), one may access the $h\rBrace{\theta,\omega_{i}}$ values and compute  
\begin{equation*}
    \resizebox{1\linewidth}{!}{
        \begin{split}
            \lBrace{h_{APP}\rBrace{\theta}} &\triangleq \lBrace{\rBrace{h^{-1}\rBrace{\theta,\omega_{1}}-h^{-1}\rBrace{\theta,\omega_{2}}}^{-1}}
            \\&=
            \lBrace{
            \frac
            {
            g^{2}\vBetaT_{1}\vd_{1}\vBetaT_{2}\vd_{2}
            }{
            g\vBetaT_{2}\vd_{2}e^{-j\dOmega\tau}-g\vBetaT_{1}\vd_{1}
            -g^{2}e^{-j\omega_{2}\tau}\rBrace{\vBetaT_{1}\vd_{1}\vAlphaT_{2}\vd_{2}-\vBetaT_{2}\vd_{2}\vAlphaT_{1}\vd_{1}}
            }
            },
        \end{split}
    }
\end{equation*}
which when setting $\vAlphaT_{i}=\vBetaT_{i},\    \vBeta_{2}=\vBrace{\hat{g},0,\hdots,0}^{T}$ degenerates to 
\begin{equation}
    \label{eqn_twoFreqApproach_h}
    \lBrace{h_{APP}\rBrace{\theta}} = \lBrace{\frac{g\vBetaT_{1}\vd_{1}}{\hat{g}^{-1}\vBetaT_{1}\vd_{1}exp\rBrace{j2\dOmega\tau}-1}}.
\end{equation}
One can identify that $h_{APP}$ closely resembles the single frequency system response (\ref{eqn:GeneralFeedbackTransferFunction}), with a very relaxed ($\dOmega\tau$ instead of $\omega\tau$) phase alignment sensitivity, allowing the user to control the width of the enhanced radial slice through setting $\dOmega$.
\par We simulate the suggested application in Fig.~\ref{}.
\subsection*{Simulations}